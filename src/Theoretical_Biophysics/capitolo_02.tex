%-------------------------------------------------------------------------------
% CAPITOLO 2 (Lezione del 04/03/2025)
%-------------------------------------------------------------------------------
\chapter{Lezione 2}
\label{chap:2}
\textit{Data: 04/03/2025}


\section{Equazione di Langevin}

\begin{equation}
m \frac{dv}{dt} = F_{ext}-\zeta v + \delta F
\label{eq:langevin_base}
\end{equation}

\begin{itemize}
    \item $F_{ext}$ è una forza deterministica nota
    \item $\zeta v$  è un termine di dissipazione che rappresenta l'attrito con il fluido circostante
    \item $\delta F$ è un termine di rumore casuale
\end{itemize}

Questa equazione descrive il moto di una particella in un fluido. Il fluido ha due effetti distinti:
\begin{itemize}
    \item Un effetto dissipativo, descritto dal coefficiente di attrito, che tende a smorzare il moto.
    \item Un effetto eccitante, dovuto agli impulsi casuali ricevuti nel tempo, che è rappresentato dal termine di rumore.
\end{itemize}

Dunque i due termini  $-\zeta v$ e $\delta F$ derivano dal fatto che il sistema si trova in un bagno termico nel caso di una particella browniana, o più in generale in un ambiente con certe sorgenti di rumore.

La prima parte dell'equazione ($m \frac{dv}{dt} = F_{ext}$), invece, è reversibile.

\subsection{Cosa accade dal punto di vista della fisica statistica?}

Quando descriviamo i sistemi nella meccanica statistica, suddividiamo l'universo in due parti: il sistema di interesse e il resto dell'universo, che agisce come un serbatoio termico. Integrando le variabili del serbatoio, si ottiene una descrizione efficace per il sistema di interesse.

Nel contesto della fisica statistica dell'equilibrio, si deriva l'ensemble canonico considerando l'universo come un sistema microcanonico isolato, con energia conservata. Partendo dalla distribuzione microcanonica dell'universo, si ottiene la distribuzione canonica per il sistema integrando le variabili del serbatoio:
$$ P(S_U) = \frac{\delta(H(S_U)-E)}{\Gamma(E)} $$
$$ S_U=(S_{sistema}, S_{reservoir}) $$
$$ P(S_s)=\sum_k P(S_k, S_r) $$
$$ P(S_s) = \frac{e^{-\beta H}}{z} $$

Nel caso descritto dall'Equazione (\ref{eq:langevin_base}), invece, l'approccio è diverso: si considera la distribuzione di probabilità a livello \textbf{dinamico}. Per tenere conto dell'ambiente, si introducono: il termine di rumore casuale e il termine di dissipazione.
Il termine di rumore esprime matematicamente l'effetto del serbatoio. In particolare, il serbatoio può scambiare energia con il sistema di interesse. Un esempio è il moto browniano: la particella riceve "colpi" dall'acqua circostante, il che si traduce in una descrizione stocastica del moto.

Questo approccio è utile perché è dinamico, permettendo di studiare il comportamento temporale del sistema. A lungo termine, se il sistema raggiunge uno stato di equilibrio con l'ambiente, si deve ottenere la stessa distribuzione di probabilità derivata dall'ensemble canonico.

Se il sistema è ergodico, ovvero esplora densamente lo spazio delle sue configurazioni microcanoniche, e se raggiunge uno stato di equilibrio con l'ambiente, allora la dinamica deve esplorare uniformemente lo spazio dei microstati. Ci sono vari regimi di diffusione che influenzano la velocità con cui questo avviene, ma non necessariamente la completezza dell'esplorazione.

Le equazioni che stiamo considerando non sono uniche. Ad esempio, per il moto di una particella:
\begin{align}
m\frac{d \mathbf{v}}{dt} &= \mathbf{F}_{ext}(\mathbf{x}) - \zeta \mathbf{v} + \delta \mathbf{F}, \label{eq:accoppiate_v} \\
\frac{d\mathbf{x}}{dt} &= \mathbf{v} \label{eq:accoppiate_x}
\end{align}
Queste due equazioni accoppiate, una per la velocità e una per la posizione, sono generalmente difficili da risolvere, tranne in casi semplici come forze lineari o elastiche.

Al posto di queste due equazioni differenziali di primo ordine, si può scrivere una sola equazione di secondo ordine per la posizione:
\begin{equation}
m\frac{d^2\mathbf{x}}{dt^2} = \mathbf{F}_{ext}(\mathbf{x}) - \zeta \frac{d\mathbf{x}}{dt} + \delta\mathbf{F}(t)
\label{eq:langevin_secondo_ordine}
\end{equation}
Questa è una formulazione alternativa, utile per simulazioni numeriche.

\subsection{Soluzione dell'equazione di Langevin assenza di forze esterne:}

Nella lezione precedente abbiamo risolto il caso più semplice: assumendo l'assenza di forze esterne ($F_{ext}=0$), l'equazione della velocità diventa indipendente dalla posizione e può essere risolta direttamente. La soluzione ottenuta è:
\begin{equation}
\mathbf{v}(t) = \mathbf{v}_0 e^{-\frac{\zeta}{m}t} + \frac{1}{m} \int_0^t e^{-\frac{\zeta}{m}(t-t')} \delta\mathbf{F}(t') dt'
\label{eq:soluzione_langevin_no_forze}
\end{equation}
Abbiamo osservato che il termine esponenziale agisce come un \textbf{memory kernel} per il sistema, pesando il contributo della condizione iniziale e del rumore nel tempo. Ciò significa che gli eventi del passato remoto hanno un peso minore rispetto a quelli del passato recente.

Il nucleo di memoria ha una sua scala temporale, un tempo caratteristico microscopico determinato dalla relazione:
\begin{equation}
\tau = \frac{m}{\zeta}
\label{eq:tempo_caratteristico}
\end{equation}
Questo nucleo di memoria è dovuto alla presenza di inerzia nel sistema. In effetti, si può ribaltare l'argomento e affermare che \textbf{l'inerzia è un modo per modellare la presenza della memoria}.

Molti biologi non gradiscono il termine \textit{inerzia}, poiché è un concetto tipicamente meccanico. Per i fisici, invece, è una nozione intuitiva e familiare. Tuttavia, i biologi comprendono bene il concetto di memoria. A livello filosofico, quindi, l'inerzia corrisponde alla memoria: è il tipo più semplice di memoria che si possa avere, in cui il passato viene pesato esponenzialmente.

\subsection{Limite overdamped}

Se si prende il limite in cui la massa tende a zero ($m \rightarrow 0$), oppure se si considerano scale temporali molto più grandi di $\tau$ ($t>>\tau$), è possibile trascurare il memory kernel. In questo caso, la soluzione dell'equazione del moto diventa:
\begin{equation}
\mathbf{v}(t) = \frac{1}{\zeta} \delta \mathbf{F}
\label{eq:limite_overdamped_v}
\end{equation}
Può essere riscritta anche come:
\begin{equation}
\frac{d\mathbf{x}}{dt} \zeta = \delta \mathbf{F}
\label{eq:limite_overdamped_x_no_ext}
\end{equation}
Questa porta all'\textbf{equazione generale di Langevin nel limite overdamped}:
\begin{equation}
\frac{d\mathbf{x}}{dt} \zeta = \delta \mathbf{F} + \mathbf{F}_{ext}
\label{eq:langevin_overdamped_generale}
\end{equation}
Questo limite è valido quando si considerano tempi molto più lunghi di $\tau$. Un esempio realistico è il moto browniano: per una particella in un fluido, il tempo caratteristico $\tau$ è dell'ordine di $10^{-12} s$, mentre i tempi di osservazione sperimentali sono molto più lunghi (millisecondi o secondi). Questo significa che \textbf{per il moto browniano si è sempre nel limite overdamped}.

\section{Teorema di fluttuazione-dissipazione}

Consideriamo il ruolo del parametro $B$, che caratterizza le fluttuazioni del rumore.
\begin{equation}
<\delta F_{\alpha}(t)\delta F_{\beta}(t')> = 2B \delta_{\alpha \beta} \delta(t-t')
\label{eq:fluttuazioni_rumore}
\end{equation}
Le variabili $\alpha$ e $\beta$ rappresentano le coordinate spaziali x, y, e z.

Supponiamo di avere una particella browniana immersa in un fluido. Dopo molte interazioni con le molecole del fluido, la particella raggiunge l'equilibrio termico in cui vale il teorema di equipartizione e in cui la sua energia cinetica media è data dalla relazione:
\begin{equation}
\frac{1}{2} m \langle v^2 \rangle_{eq} = \frac{1}{2} \; d \; k_B T
\label{eq:equipartizione_energia}
\end{equation}
Nel caso generale, questa relazione si generalizza a:
\begin{equation}
\frac{1}{2} m \sum_\alpha \langle v_\alpha^2 \rangle = \frac{1}{2} \; d \; k_B T
\label{eq:equipartizione_generale}
\end{equation}
Vogliamo calcolare la media di $v^2(t)$. Questa media dipende dal tempo, e per $t \to \infty$, ci aspettiamo che sia uguale al valore di equilibrio di $v^2$:
$$ \langle v^2 (t) \rangle \longrightarrow \langle v^2 \rangle_{eq} \; \; \; \; \; \; \; \; \;  \mathrm{per} \;  t \rightarrow +\infty $$
Questa media è calcolata utilizzando la distribuzione canonica di equilibrio. Effettuare una media in questo problema dinamico stocastico significa fare una media rispetto alla distribuzione del rumore, ovvero dobbiamo considerare le fluttuazioni del rumore stesso.

Assumiamo che la distribuzione del rumore sia gaussiana:
$$ P[\delta F] = \exp(-\frac{\int dt' \delta^2 F(t')}{2B}) $$
È un funzionale in quanto $\delta F$ dipende dal tempo.

Non è necessario conoscere l'intera distribuzione del rumore; per i nostri scopi, basta conoscere il valore medio e la sua varianza:
\begin{itemize}
    \item $<\delta F(t)> =0$
    \item $<\delta F_{\alpha}(t)\delta F_{\beta}(t')> = 2B \delta_{\alpha \beta} \delta(t-t')$
\end{itemize}
Poiché ogni coordinata si comporta in modo indipendente, possiamo esaminarle singolarmente. Quindi consideriamo dapprima il problema in una dimensione.
\begin{equation}
\begin{aligned}
\langle v_\alpha^2(t) \rangle &= v_{0 \alpha}^2 e^{-\frac{2 \zeta}{m} t} \\
&\quad + v_{0 \alpha} e^{-\frac{ \zeta}{m} t} \int_0^t dt' \, e^{-\frac{ \zeta}{m} (t-t')} \frac{\langle \delta F_{\alpha}(t') \rangle}{m} \\
&\quad +  \int_0^t dt' \, e^{-\frac{ \zeta}{m} (t-t')} \int_0^t dt'' \, e^{-\frac{ \zeta}{m} (t-t'')} \frac{\langle \delta F_{\alpha}(t') \; \delta F_{\alpha}(t'')\rangle}{m \cdot m}
\end{aligned}
\label{eq:media_v_alpha_quadro_start}
\end{equation}
\begin{itemize}
    \item Il primo termine è una costante e non dipende dal rumore.
    \item Il secondo termine è nullo poiché stiamo considerando per ipotesi che il valor medio del rumore sia pari a 0.
    \item Il terzo termine è l'unico di cui devo effettivamente calcolare il valore medio; questo è dato semplicemente dalla varianza del rumore (risultato già noto).
\end{itemize}
\begin{equation}
\begin{aligned}
\langle v_\alpha^2(t) \rangle &= v_{0 \alpha}^2 e^{-\frac{2 \zeta}{m} t} \\
&\quad + \int_0^t dt' \, e^{-\frac{ \zeta}{m} (t-t')} \int_0^t dt'' \, e^{-\frac{ \zeta}{m} (t-t'')} \frac{2B\delta(t'-t'')}{m^2}
\end{aligned}
\label{eq:media_v_alpha_quadro_mid}
\end{equation}
Poiché sia l'intervallo di integrazione di $t''$ che quello di $t'$ vanno da 0 a $t$, possiamo sfruttare la funzione a delta per eliminare uno degli integrali.
\begin{equation}
\begin{aligned}
\langle v_\alpha^2(t) \rangle &= v_{0 \alpha}^2 e^{-\frac{2 \zeta}{m} t} + \frac{2B}{m^2} \int_0^t dt' \, e^{-\frac{ 2\zeta}{m} (t-t')} \\ &= v_{0 \alpha}^2 e^{-\frac{2 \zeta}{m} t} + \frac{2B}{m^2} \; e^{-\frac{2 \zeta}{m} t} \;  \frac{m}{2 \zeta} (e^{+\frac{2 \zeta}{m} t}-1)
\end{aligned} 
\label{eq:media_v_alpha_quadro_int}
\end{equation}
\begin{equation}
\langle v_\alpha^2(t) \rangle = v_{0 \alpha}^2 e^{-\frac{2 \zeta}{m} t} + \frac{B}{m \zeta} \; (1-e^{-\frac{2 \zeta}{m} t})
\label{eq:media_v_alpha_quadro_final}
\end{equation}
Trovata la soluzione per una dimensione, bisogna sommare su tutti gli $\alpha$.
\begin{equation}
\langle v^2(t) \rangle = v_0^2 e^{-\frac{2 \zeta}{m} t} + \frac{d B}{m \zeta} \; (1-e^{-\frac{2 \zeta}{m} t})
\label{eq:media_v_quadro_total}
\end{equation}
Consideriamo il comportamento a lungo termine, nella condizione di equilibrio nel limite per $t \to \infty$. Quando $t$ tende a infinito, il primo termine tende a zero, così come l'altro termine esponenziale nella parentesi. Quindi si ha:
\begin{equation}
\langle v^2 \rangle_{eq} = \frac{d B}{m \zeta} 
\label{eq:v_quadro_equilibrio}
\end{equation}
Sfruttando la relazione (\ref{eq:equipartizione_energia}):
\begin{equation}
\langle v^2 \rangle_{eq} = \frac{d B}{m \zeta}  =  \frac{d \; k_b T}{m}
\label{eq:confronto_equilibrio}
\end{equation}
Si ritrova così l'equazione del \textbf{Teorema di fluttuazione-dissipazione:}
\begin{equation}
B = k_b T \; \zeta
\label{eq:fdt}
\end{equation}
Questo teorema mette in relazione le proprietà del rumore stocastico (che guida la dinamica) con la temperatura. Mette in relazione l'ampiezza del rumore stocastico B anche con il coefficiente di attrito; esprime la relazione che esiste tra la frizione (che tende a fermare il moto della particella) e il rumore (che tende a mantenere il sistema in moto).

\subsection{Equazione di Stokes-Einstein}

Ora vogliamo trovare la relazione che sussiste con il coefficiente di diffusione.

L'espressione per il coefficiente di diffusione è data da: $\langle \delta x^2 \rangle = 2Dt$
\begin{equation}
D = \lim_{t \to \infty} \frac{\langle \delta x^2 \rangle}{2t}
\label{eq:coeff_diffusione}
\end{equation}
Bisogna, quindi, trovare il valore medio di $\delta x^2$.

Poiché per ipotesi $\langle x \rangle =0$, si ha:
\begin{equation}
\begin{aligned}
\langle \delta x^2 \rangle &= \sum_{\alpha} \langle (x_{\alpha} (t) - x_{\alpha} (0)) ^2 \rangle \\ &= \sum_{\alpha} \langle x_{\alpha} (t)^2 \rangle
\end{aligned}
\label{eq:delta_x_quadro}
\end{equation}
Dove è stato sfruttato il fatto che abbiamo ipotizzato di porre nell'istante iniziale il granello di polline nell'origine, per cui $x_{\alpha} (t=0) =0$.

Usando l'equazione nel \textbf{limite overdamped} si ha:
\begin{equation}
\frac{d\mathbf{x}}{dt} = \frac{1}{\zeta} \delta \mathbf{F} \quad \longrightarrow \quad \mathbf{x}(t)= \mathbf{x}_0 + \int_0^t dt' \frac{\delta \mathbf{F}(t')}{\zeta} =  \int_0^t dt' \frac{\delta \mathbf{F}(t')}{\zeta} 
\label{eq:soluzione_x_overdamped}
\end{equation}
Quindi dobbiamo calcolare:
\begin{equation}
\begin{aligned}
\langle x_{\alpha} (t)^2 \rangle &= \langle \int_0^t dt' \frac{\delta \bar{F_{\alpha}}(t')}{\zeta}  \int_0^t dt'' \frac{\delta \bar{F_{\alpha}}(t'')}{\zeta} \rangle \\ &= \frac{1}{\zeta^2} \int_0^t dt' \int_0^t dt'' \langle \delta F_{\alpha}(t') \; \delta F_{\alpha}(t'')\rangle \\ &= \frac{1}{\zeta^2} \int_0^t dt' \int_0^t dt'' 2 k_b T \zeta \delta (t'-t'') \\ &= \frac{2 \; k_b \; T \; t}{\zeta}
\end{aligned}
\label{eq:media_x_alpha_quadro}
\end{equation}
Trovata la soluzione per una dimensione, bisogna sommare su tutti gli $\alpha$.
\begin{equation}
\begin{aligned}
\langle \delta x^2 \rangle &=  \sum_{\alpha} \langle x_{\alpha} (t)^2 \rangle = \sum_{\alpha}^d \frac{2 \; k_b \; T \; t}{\zeta} \\ &= \frac{2 \; d \; k_b \; T \; t}{\zeta}
\end{aligned} 
\label{eq:media_delta_x_quadro_total}
\end{equation}
Usando la relazione (\ref{eq:coeff_diffusione}), si trova:
\begin{equation}
D= \frac{ d \; k_b \; T }{  \zeta}
\label{eq:stokes_einstein_zeta}
\end{equation}
L'espressione di $\zeta$ è determinata dalla forma dell'oggetto e dalle proprietà del fluido. In generale nel caso di una particella sferica in contatto con un fluido con viscosità $\eta$ si ha: $\zeta= 6 \pi R \eta$.

Se indichiamo con $A$ il fattore geometrico, si trova: $\zeta= A \eta$

Si ottiene così l'\textbf{Equazione di Stokes-Einstein :}
\begin{equation}
D= \frac{ d \; k_b \; T }{  A \eta}
\label{eq:stokes_einstein_final}
\end{equation}

\section{Processo di Wiener}

Consideriamo l'equazione nel caso overdamped.
\begin{equation}
\frac{d\mathbf{x}}{dt} \zeta = \delta \mathbf{F}
\label{eq:overdamped_wiener_start}
\end{equation}
Per rendere l'equazione più semplice si può porre il coefficiente d'attrito uguale a 1; in questo modo si ottiene un'equazione più semplice, ma tramite un metodo formalmente non corretto.

Per eliminare il parametro in modo matematicamente corretto, si deve riscalare tutto. Consideriamo la relazione:
\begin{equation}
<\delta F(t_1)\delta F(t_2)> = 2B  \delta(t_1-t_2)
\label{eq:rumore_relazione_B}
\end{equation}
Ora riscaliamo il tempo $t' = \frac{t}{\zeta}$, e riscriviamo la (\ref{eq:overdamped_wiener_start}) in funzione di $t'$:
\begin{equation}
\frac{dx (t)}{dt'}  = \delta F(t)
\label{eq:riscalata_t_primo}
\end{equation}
Definiamo delle nuove variabili:
\begin{align}
\tilde{x}(t) &= x(t) = x(\zeta t') \label{eq:nuova_x} \\
\xi (t')&=\delta F(\zeta t') \label{eq:nuova_xi}
\end{align}
Così facendo otteniamo:
\begin{equation}
\frac{d \tilde{x}(t')}{dt'}  = \xi(t')
\label{eq:equazione_riscalata_finale}
\end{equation}
Verifichiamo che così facendo anche la varianza del rumore non dipende più dall'attrito:
\begin{equation}
<\xi(t'_1) \xi(t'_2)> = <\delta F(\zeta t'_1)\delta F(\zeta t'_2)> = 2 k_b T \zeta   \delta(\zeta(t'_1-t'_2))
\label{eq:varianza_xi_start}
\end{equation}
Usiamo la proprietà della delta di Dirac:
\begin{equation}
<\xi(t'_1) \xi(t'_2)> = 2 k_b T \zeta   \frac{\delta(t'_1-t'_2)}{\zeta} = 2 k_b T \delta(t'_1-t'_2)
\label{eq:varianza_xi_final}
\end{equation}
In questo modo abbiamo eliminato l'attrito sia dall'equazione di Langevin che dalla varianza del rumore. Matematicamente, ciò equivale a ridefinire l'unità microscopica di misura del tempo.

Da qui in poi per alleggerire la notazione al posto di $t'$ usiamo semplicemente $t$.

Passiamo ora al \textbf{processo di Wiener}. La relazione $\frac{d x(t)}{dt}  = \xi(t)$ in matematica, viene scritta come:
\begin{align}
dx&=dW \label{eq:wiener_dx} \\
W &= \xi (t) dt \label{eq:wiener_W}
\end{align}
Questo perché, scritta nel primo modo, l'equazione non è differenziabile. Consideriamo il valore medio di $dx^2$ su un intervallo di tempo infinitesimo:
\begin{equation}
\begin{aligned}
<(dx)^2> &= <(x(t+dt)-x(t))^2> \\
&= \int_t^{t+dt} dt' \int_t^{t+dt} dt'' <\xi(t') \xi(t'')> = 2 k_b T \delta(t'-t'')
\end{aligned}
\label{eq:media_dx_quadro}
\end{equation}
Quindi si ha:
$$ dx^2 \simeq dt \quad \rightarrow \quad dx \simeq dt^{1/2} $$
$$ \frac{dx}{dt} \simeq \frac{1}{dt^{1/2}} \longrightarrow + \infty \quad \mathrm{per} \; \; dt \to 0 $$
Quindi mandando $dt \to 0$, il risultato diverge, il che implica che l'equazione non è differenziabile.

Risulta più corretto da un punto di vista formale, pertanto, usare la notazione del processo di Wiener:
$$
\begin{cases}
dx&=dW \\
W &= \xi (t) dt
\end{cases}
$$

