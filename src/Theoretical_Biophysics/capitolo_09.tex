%-------------------------------------------------------------------------------
% CAPITOLO 9 (Lezione del 28/03/2025)
%-------------------------------------------------------------------------------
\chapter{Lezione 9}
\label{capitolo_09}
\textit{Data: 28/03/2025}


\section{Dinamica chemiotattica}

Il batterio è sensibile ai gradienti e non al valore assoluto della concentrazione. Questo concetto è ben rappresentato dal grafico sottostante.

Nel primo caso, inizialmente il batterio si trova in un ambiente in cui la concentrazione è costante, ad un certo punto, per un breve periodo, la concentrazione viene aumentata e poi è riportata ai valori iniziali. Nel secondo caso, invece, la concentrazione viene aumentata e poi lasciata ad un valore più elevato. In entrambi i casi il grafico che mostra l'andamento della probabilità di rotazione antioraria dei flagelli è analogo: inizialmente aumenta, per poi tornare ad un valore costante. Si evince, quindi, che ciò che guida il comportamento chimico è la variazione della concentrazione del nutriente, e non il solo valore assoluto.

\begin{figure}[h!]
    \centering
    \includegraphics[width=0.5\textwidth]{pics/09_1.jpeg}
\end{figure}

\begin{enumerate}
    \item Se la concentrazione non è cambiata dall'esterno nel tempo, ma la distribuzione del nutriente è disomogenea nello spazio, allora si ha:
    \begin{equation}
        \dot{c} = \nabla c \cdot \bar{v}
        \label{eq:chap9_1}
    \end{equation}
    In questo caso c'è una variazione temporale nel nutriente indotta dal fatto che il materiale si muove e, quindi, esplora le regioni con diversa concentrazione.

    \item Se, invece, la concentrazione viene cambiata dall'esterno nel tempo, si ha:
    \begin{equation}
        \dot{c} = \frac{dc}{dt}
        \label{eq:chap9_2}
    \end{equation}
\end{enumerate}

Quando la concentrazione di nutriente aumenta, la probabilità di rotazione antioraria aumenta, ciò implica che la lunghezza della corsa aumenta e che il \textbf{tumbling rate} diminuisce.

Il tumbling rate è definito come:
\begin{equation}
    \alpha(\dot{c}) = \frac{1}{\tau}
    \label{eq:chap9_3}
\end{equation}
dove $\tau$ è la durata media della corsa.

Il tumbling rate è una funzione \textbf{decrescente} di $\dot{c}$. Quindi se $\dot{c}$ aumenta, $\alpha$ diminuisce, e viceversa.

Supponiamo di osservare una popolazione di batteri in una dimensione $x$. Sia $P(x,t) dx$ la probabilità di trovare un batterio in $x$ al tempo $t$. Questa probabilità può anche essere espressa come il rapporto del numero di batteri in $x$ al tempo $t$ per il numero totale di batteri (che consideriamo essere fisso): $\frac{N(x,t)}{N_{tot}}$

L'obiettivo è scrivere un'equazione per $P(x,t)$ e capire come questa distribuzione dipende dalla concentrazione del nutriente.

Discretizziamo la retta e assumiamo che la concentrazione di gradiente aumenti spostandosi verso destra dalla coordinata $x$.

\begin{verbatim}
|-------|-------|-------|-------|-------|-------|------->
            - <--   x  --> +
\end{verbatim}

Dividiamo la popolazione di batteri in due sub-popolazioni, in modo che:
\begin{itemize}
    \item $P^+(x,t)$ descrive la popolazione di batteri che si muovono verso destra
    \item $P^-(x,t)$ descriva la popolazione di batteri che si muovono verso sinistra
\end{itemize}

Poichè in questo caso la distribuzione del nutriente è disomogenea nello spazio, l'espressione del tumbling rate $\alpha$ è: $\alpha(\dot{c}) = \alpha(\nabla c \cdot \bar{v})$

Distinguiamo anche per $\alpha$ il caso in cui si va a destra e quello in cui si va a sinistra:
\begin{itemize}
    \item $\alpha^+(\dot{c}) = \alpha(\nabla c \cdot \bar{v}) = \alpha \left( \frac{dc}{dx} \; v \right)$
    \item $\alpha^-(\dot{c}) = \alpha \left( -\frac{dc}{dx} \; v \right)$
\end{itemize}

Nel caso della popolazione $+$ la velocità è sempre orientata verso destra per definizione, quindi ha segno positivo. Nella popolazione $-$ la velocità è orientata verso sinistra, quindi ha segno negativo.

Discretizziamo l'asse e scegliamo la lunghezza dell'intervallo $d x$ in modo tale da poter assumere che i batteri si muovano a velocità fissa: $dx=v \;dt$

Il cambiamento di $P^+(x,t)$ in un piccolo intervallo di tempo è dato da:
\begin{equation}
    \delta P^+(x,t)=P^+(x-dx)-P^+(x)-\frac{\alpha^+}{2}P^+(x,t)dt + \frac{\alpha^-}{2}P^-(x,t)dt
    \label{eq:chap9_4}
\end{equation}

I primi due termini indicano rispettivamente i batteri che entrano in $x$ e quelli che escono da $x$.

Bisogna anche considerare che alcuni batteri eseguono una torsione e cambiano direzione. La probabilità di tumbling è omogenea in tutte le direzioni; in una dimensione ciò implica che la probabilità di effettuare un tumble verso destra è uguale alla probabilità di effettuare un tumble verso sinistra. Bisogna quindi considerare anche il contributo degli ultimi due termini che indicano rispettivamente i batteri che inizialmente si stavano muovendo verso destra e che fanno un tumble verso sinistra (vanno sottratti), e i batteri che inizialmente si stavano muovendo verso sinistra e che fanno un tumble verso destra (vanno sommati).

Dividiamo per $\delta t$ e consideriamo il caso in cui $\delta t$ è molto piccolo. In questo modo il termine a sinistra diventa una derivata rispetto al tempo.
\begin{equation}
    \frac{\partial P^+}{\partial t} = \frac{dx}{dt} \frac{ P^+(x-dx)-P^+(x)}{dx}-\frac{\alpha^+}{2}P^+(x,t) + \frac{\alpha^-}{2}P^-(x,t)
    \label{eq:chap9_5}
\end{equation}

Analogamente, possiamo fare esattamente la stessa cosa per la popolazione $-$. Alla fine, si ottengono due equazioni:
\begin{equation}
    \begin{cases}
        & \frac{\partial P^+}{\partial t} = -v \frac{ \partial P^+}{ \partial x}-\frac{\alpha^+}{2}P^+(x,t) + \frac{\alpha^-}{2}P^-(x,t) \\
        &\frac{\partial P^-}{\partial t} = +v \frac{ \partial P^-}{ \partial x}+\frac{\alpha^+}{2}P^+(x,t) - \frac{\alpha^-}{2}P^-(x,t)
    \end{cases}
    \label{eq:chap9_6}
\end{equation}

Vogliamo risolvere queste equazioni, ma risolverle dinamicamente è complicato, consideriamo quindi una \textbf{situazione stazionaria.} Lo stato stazionario è una situazione in cui la popolazione non cambia più. In tale stato, le quantità non dipendono dal tempo. Quindi, possiamo porre il lato sinistro delle equazioni pari a zero:
\begin{equation}
    \frac{\partial P^+}{\partial t} = \frac{\partial P^-}{\partial t} =0
    \label{eq:chap9_7}
\end{equation}

Possiamo riscrivere le due equazioni nel seguente modo:
\begin{equation}
    v \frac{ d P^+}{ dx}=-\frac{\alpha^+}{2}P^+(x,t) + \frac{\alpha^-}{2}P^-(x,t)
    \label{eq:chap9_8}
\end{equation}
\begin{equation}
    v \frac{ d P^-}{ d x} = -\frac{\alpha^+}{2}P^+(x,t) + \frac{\alpha^-}{2}P^-(x,t)
    \label{eq:chap9_9}
\end{equation}

Poiché nello stato stazionario $P^+$ e $P^-$ dipendono solo da $x$, la derivata parziale è stata sostituito con una totale.

Sottraiamo la \eqref{eq:chap9_9} alla \eqref{eq:chap9_8}. I termini con $\alpha^+$ e $\alpha^-$ si cancellano a vicenda e otteniamo:
\begin{equation}
    v \frac{d(P^+ - P^-)}{dx} = 0 \; \; \longrightarrow \; \; P^+ - P^- = A=\text{costante}
    \label{eq:chap9_10}
\end{equation}

Questa costante è determinata dalle condizioni al contorno. Se imponiamo condizioni tali per cui vi è assenza di corrente, allora la costante è pari a 0 e si ha:
\begin{equation}
    P^+ = P^- = \frac{P}{2}
    \label{eq:chap9_11}
\end{equation}
dove $P = P^+ + P^-$.

Ora sommiamo le equazioni \eqref{eq:chap9_9} e \eqref{eq:chap9_8}:
\begin{equation}
    v \left( \frac{dP^+}{dx} + \frac{dP^-}{dx} \right)= - \alpha ^+ P^++\alpha^- P^-
    \label{eq:chap9_12}
\end{equation}

Poiché $P^+ = P^- = \frac{P}{2}$, otteniamo:
\begin{equation}
    v \frac{dP}{dx} = - (\alpha ^+ -\alpha^- ) \frac{P}{2}
    \label{eq:chap9_13}
\end{equation}

Questa equazione descrive la variazione spaziale della densità di probabilità. Il termine a sinistra rappresenta quante particelle escono dalla regione $x$. I termini con $\alpha$ rappresentano il cambiamento dovuto al tumbling delle particelle. Stiamo considerando un comportamento medio: non analizziamo le fluttuazioni, che richiederebbero un'equazione stocastica per ciascun batterio.

\subsection{Analisi con l'espansione lineare}

Assumiamo che $\alpha(\dot{c})$ sia una funzione regolare e che le variazioni del nutriente siano piccole. Eseguiamo un'espansione lineare e fermiamoci al primo termine in modo che $\alpha$ dipenda linearmente da $\dot{c}$:
\begin{equation}
    \alpha(\dot{c}) \approx \alpha(0) + \left. \frac{d\alpha}{d\dot{c} } \right|_{0} \dot{c}+ \dots
    \label{eq:chap9_14}
\end{equation}

Ricordando che $\dot{c}= \pm v \frac{dc}{dx}$, si ha:
allora:
\begin{align}
    \alpha^+ &= \alpha(0) + \left. \frac{d\alpha}{d \dot{c}} \right|_{0} v \frac{dc}{dx} \label{eq:chap9_15} \\
    \alpha^- &= \alpha(0) - \left. \frac{d\alpha}{d\dot{c} } \right|_{0} v \frac{dc}{dx} \label{eq:chap9_16}
\end{align}

Sottraendo:
\begin{equation}
    \alpha^+ - \alpha^- = 2 \left. \frac{d\alpha}{d \dot{c}} \right|_{0} v \frac{dc}{dx}
    \label{eq:chap9_17}
\end{equation}

Quindi l'equazione \eqref{eq:chap9_13} diventa:
\begin{equation}
    \frac{1}{P} \frac{dP}{dx} = - \left. \frac{d\alpha}{d \dot{c}} \right|_{0} \frac{dc}{dx}
    \label{eq:chap9_18}
\end{equation}

Integriamo:
\begin{equation}
    \int_0^{P(x)} \frac{dP}{P} = - \left. \frac{d\alpha}{d \dot{c}} \right|_{0} \int_0^{c(x)} dc
    \label{eq:chap9_19}
\end{equation}

Abbiamo portato $\left. \frac{d\alpha}{d \dot{c}} \right|_{0}$ fuori dall'integrale perchè è un numero.

Integrando si trova:
\begin{equation}
    \ln \left( \frac{P(x)}{P(0)} \right) = -\left. \frac{d\alpha}{d \dot{c}} \right|_{0} \left( c(x) - c(0) \right)
    \label{eq:chap9_20}
\end{equation}

Ovvero:
\begin{equation}
    P(x) = A \; \exp\left( -\left. \frac{d\alpha}{d \dot{c}} \right|_{0} c(x) \right)
    \label{eq:chap9_21}
\end{equation}
dove A é una costante.

Questa distribuzione è della forma Gibbs-Boltzmann.

Se $\dot{c}$ aumenta, $\alpha$ diminuisce, quindi $\frac{d\alpha}{d\dot{c}} < 0$. L'esponente della \eqref{eq:chap9_21} quindi è positivo, segue che \textbf{i massimi di $P(x)$ coincidono con i massimi della concentrazione del nutriente}. A lungo termine perciò la popolazione si concentra dove il nutriente è maggiore, cioè nei massimi di $c(x)$.

\subsection{Analisi senza espansione lineare}

Supponiamo ora di non voler assumere linearità. Allora:
\begin{equation}
    \alpha^+ - \alpha^- = \alpha\left(v \frac{dc}{dx}\right) - \alpha\left(-v \frac{dc}{dx}\right) \equiv f(x)
    \label{eq:chap9_22}
\end{equation}

Sostituiamo nella \eqref{eq:chap9_13}:
\begin{equation}
    v \frac{dP(x)}{dx} = - f(x)\frac{P(x)}{2}
    \label{eq:chap9_23}
\end{equation}

Integriamo:
\begin{equation}
    \int_0^{P(x)} \frac{dP}{P} = - \frac{1}{2v}\int_0^{x} f(x') dx'
    \label{eq:chap9_24}
\end{equation}

Integrando si trova:
\begin{equation}
    P(x) = p(0) \exp\left( - \frac{1}{2v} \int_0^x f(x') dx' \right)
    \label{eq:chap9_25}
\end{equation}

Questa è ancora una distribuzione di Boltzmann $P(x) \approx e^{- \beta V(x)}$
con un \textbf{potenziale efficace} $V(x)$ dato da:
\begin{equation}
    V(x) = \frac{1}{2v} \int_0^x f(x') dx'
    \label{eq:chap9_26}
\end{equation}

\subsubsection{Massimi della distribuzione}

I massimi di $P(x)$ si trovano nei minimi di $V(x)$.

Troviamo i punti stazionari di $V(x)$ imponendo che $\frac{dV}{dx} = 0$:
\begin{equation}
    \frac{1}{2v} f(x) = 0 \; \; \longrightarrow \; \; \alpha\left(v \frac{dc}{dx} \right) = \alpha\left(-v \frac{dc}{dx} \right)
    \label{eq:chap9_27}
\end{equation}

Poiché $\alpha$ è una funzione monotona, per ottenere la condizione espressa dalla \eqref{eq:chap9_27}, l'argomento deve essere zero, ovvero:
\begin{equation}
    \frac{dc}{dx}=0
    \label{eq:chap9_28}
\end{equation}

Quindi i punti stazionari di $V(x)$ sono anche punti stazionari della concentrazione del nutriente.

Per capire se sono massimi o minimi, calcoliamo la derivata seconda e imponiamo che sia maggiore di zero:
\begin{equation}
    \left.\frac{d^2V}{dx^2} \right|_{min} >0
    \label{eq:chap9_29}
\end{equation}

La derivata seconda di $V$ rispetto a $x$ è data da $f'(x)$, che si ottiene derivando $\alpha^+ - \alpha^-$:
\begin{equation}
    \frac{d\alpha^+}{dx} - \frac{d\alpha^-}{dx} >0
    \label{eq:chap9_30}
\end{equation}

Usiamo la relazione $\alpha^{\pm}(\dot{c}) = \alpha \left( \pm \frac{dc}{dx} \; v \right)$ e la regola di derivazione della funzione composta:
\begin{equation}
    \left. \frac{d\alpha}{d \dot{c}} \right|_{min} \frac{d}{dx} \left( v\frac{dc}{dx}\right) - \left. \frac{d\alpha}{d \dot{c}} \right|_{min} \frac{d}{dx} \left( -v\frac{dc}{dx}\right) >0
    \label{eq:chap9_31}
\end{equation}
\begin{equation}
    \left. \frac{d\alpha}{d \dot{c}}\right|_{min} v \;\frac{d^2c}{dx^2} >0
    \label{eq:chap9_32}
\end{equation}

Se $\dot{c}$ aumenta, $\alpha$ diminuisce, quindi $\frac{d\alpha}{d\dot{c}} < 0$. Affinchè la \eqref{eq:chap9_32} sia verificata, perciò:
\begin{equation}
    \frac{d^2c}{dx^2} <0
    \label{eq:chap9_33}
\end{equation}

Ovvero il punto considerato è un punto di massimo per $c$.

Anche con questo approccio abbiamo trovato che \textbf{i massimi di $P(x)$ coincidono con i massimi della concentrazione del nutriente $c(x)$}. Quindi, la dinamica chemiotattica porta i batteri verso i massimi della concentrazione del nutriente.

\section{Linear Response Theory}

Si vuole ora giustificare perché l'assunzione che $\alpha(\dot{c})$ dipenda linearmente da $\dot{c}$ sia ragionevole. Per farlo, si introduce il concetto di risposta di un sistema a uno stimolo esterno.
Un segnale esterno $c(x)$ (input) influenza una variabile interna del sistema, $\alpha$ (output).

Quindi l'espressione $\alpha(\dot{c}) = \alpha(0) + A \;\dot{c}$ descrive il comportamento di risposta del sistema ad un segnale di ingresso.


Usiamo la Linear Response Theory e chiamiamo in generale $x$ la variabile di interesse (che in questo caso specifico era $\alpha$).

Consideriamo il sistema descritto dalla seguente equazione di Langevin:
\begin{equation}
    \zeta \frac{dx}{dt} = F(x) + \delta F
    \label{eq:chap9_34}
\end{equation}

Dove $F(x) = - \frac{dU(x)}{dx}$

Perturbiamo il sistema in modo tale che: $U(x) \rightarrow U(x) + x \;f$ e studiamo la risposta del sistema nel caso in cui $U(x) = \frac{1}{2} kx^2$

In questo caso: $F(x) = -kx + f = F_0(x) + f$

L'equazione di Langevin diventa:
\begin{equation}
    \zeta \frac{dx}{dt} = -kx + \delta F +f
    \label{eq:chap9_35}
\end{equation}

\begin{itemize}
    \item Se considero il sistema \textbf{imperturbato} ($f=0$):

    Consideriamo $\zeta = 1$
    \begin{equation}
        x(t) = x_0 e^{-k t} + \int_0^t dt' e^{-k(t - t')} \delta F(t')
        \label{eq:chap9_36}
    \end{equation}
    \begin{equation}
        \langle x(t) \rangle_0 = x_0 e^{-k t} \longrightarrow 0 \; \; \; \; \mathrm{per } \; \;t \to \infty
        \label{eq:chap9_37}
    \end{equation}

    \item Se il sistema è \textbf{perturbato}, invece, si ha: $\langle x \rangle_f \neq \langle x \rangle_0$
    \begin{equation}
        x(t) = x_0 e^{-k t} + \int_0^t dt' e^{-k(t - t')} \left( \delta F(t') +f \right)
        \label{eq:chap9_38}
    \end{equation}
    Sfruttando il fatto che la media del rumore è nulla, si trova:
    \begin{equation}
        \langle x(t) \rangle_f = x_0 e^{-k t} + \int_0^t dt' e^{-k(t - t')} f(t') = \langle x(t) \rangle_0 + \int_0^t dt' e^{-k(t - t')} f(t')
        \label{eq:chap9_39}
    \end{equation}

    Nel formalismo della teoria della risposta lineare:
    \begin{equation}
        \langle x(t) \rangle_f = \langle x(t) \rangle_0 + \int_0^t dt' R(t - t') f(t')
        \label{eq:chap9_40}
    \end{equation}

    Confrontando con la formula precedente, otteniamo che il \textbf{Response Kernel} è:
    \begin{equation}
        R(t - t') = e^{-k(t - t')}
        \label{eq:chap9_41}
    \end{equation}
\end{itemize}

Calcoliamo ora la \textbf{funzione di correlazione} in assenza di perturbazione:
\begin{equation}
    \begin{aligned}
        C(t, t') &= \langle \delta x(t) \delta x(t') \rangle \\
        &= \langle (x(t) - \langle x \rangle) (x(t') - \langle x \rangle) \rangle
    \end{aligned}
    \label{eq:chap9_42}
\end{equation}

Sfruttando la \eqref{eq:chap9_36} e la \eqref{eq:chap9_37}, si ha:
\begin{equation}
    \begin{aligned}
        C(t, t') &= \langle \int_0^t dt_1 e^{-k(t - t_1)} \delta F(t_1) \; \; \int_0^{t'} dt_2 e^{-k(t - t_2)} \delta F(t_2) \; \rangle \\
        &= \int_0^t dt_1 \int_0^{t'} dt_2 e^{-k(t+t'-t_1 - t_2)} \langle \delta F(t_1) \delta F(t_2) \rangle \\
        &= \int_0^t dt_1 \int_0^{t'} dt_2 e^{-k(t+t'-t_1 - t_2)} 2 K_b T \delta(t_1 -t_2)
    \end{aligned}
    \label{eq:chap9_43}
\end{equation}

Nell'ultimo passaggio è stato sfruttato il fatto che $\langle \delta F(t_1) \delta F(t_2) \rangle = 2K_b T \zeta \delta (t_1-t_2)$ e che in questo caso stiamo assumendo $\zeta=1$.

Poiché $\delta(t_1 - t_2)$ impone $t_1 = t_2$, è utile iniziare con l'integrazione sulla variabile con l'estremo superiore più grande. Supponiamo quindi che t > t'. Allora:
\begin{itemize}
    \item $t_1 \in [0, t]$
    \item $t_2 \in [0, t']$
\end{itemize}
Se integriamo prima su $t_1$, il range include sempre $t_2$, permettendo di implementare facilmente la delta. Integrare prima $t_2$ richiederebbe più attenzione, perché ci sono valori di $t_1 > t'$ per cui la delta vale zero. Assumiamo quindi che $t>t'$ e integriamo prima in $dt_1$ in modo da non avere problemi con i domini di integrazione:
\begin{equation}
    \begin{aligned}
        C(t, t') &= \int_0^{t'} dt_2 e^{-k(t+t'- 2t_2)} \; 2 K_b T \\
        &= e^{-k(t+t')} \int_0^{t'} dt_2 \; e^{ 2 k t_2} \; 2 K_b \\
        &= e^{-k(t+t')} \; \frac{1}{2k} \; (e^{ 2 k t'} -1) \; 2 K_b T
    \end{aligned}
    \label{eq:chap9_44}
\end{equation}

Quindi:
\begin{equation}
    C(t, t') = \frac{K_b T}{k} \; (e^{-k(t-t')} - e^{-k(t+t')})
    \label{eq:chap9_45}
\end{equation}

Per $t, t' \to \infty$ il secondo termine ($e^{-k(t + t')}$) tende a zero, mentre il primo rimane finito, quindi:
\begin{equation}
    C(t, t') = \frac{K_b T}{k} \; e^{-k(t-t')} \; \; \; \; \mathrm{per} \; \; t, t' \to \infty
    \label{eq:chap9_46}
\end{equation}

Questa funzione di correlazione dipende solo dalla differenza dei tempi: è un segno della \textbf{invarianza per traslazione temporale}, caratteristica dei sistemi in equilibrio.

Abbiamo ora la funzione di risposta $R(t - t')$ e la funzione di correlazione $C(t - t')$. In un \textbf{sistema in equilibrio} valgono le relazioni del \textbf{teorema di fluttuazione-dissipazione}:
\begin{equation}
    R(t - t') = -\beta \frac{\partial }{\partial t} C(t - t')
    \label{eq:chap9_47}
\end{equation}
\textbf{Il teorema afferma che in sistemi all'equilibrio, la risposta a una perturbazione è legata alle fluttuazioni spontanee del sistema.}

Supponiamo di applicare al tempo $t=0$ una perturbazione esterna costante $f(t) = f_0$.


La variazione nella variabile osservabile $\delta \langle x(t) \rangle$ sarà:
\begin{equation}
    \delta \langle x(t) \rangle = \int_0^t dt' \, R(t - t') \cdot f_0 = f_0 \int_0^t R(t - t') dt'
    \label{eq:chap9_48}
\end{equation}

Per $t \to \infty$, otteniamo la risposta statica:
\begin{equation}
    \delta \langle x \rangle \approx f_0 \int_0^{\infty} R(t-t') dt' = f_0 \cdot \chi
    \label{eq:chap9_49}
\end{equation}

Dove $\chi$ è la \textbf{suscettibilità statica}. In sistemi fisici dissipativi classici (es. particella in potenziale quadratico), questa risposta tende a un valore finito.

\begin{figure}[h!]
    \centering
    \includegraphics[width=0.5\textwidth]{pics/09_3.png}
    \caption{Andamento della risposta nei sistemi fisici classici.}
\end{figure}

Nel caso di un sistema come un batterio, abbiamo osservato invece un comportamento diverso: il sistema risponde alla perturbazione, ma poi torna al valore iniziale. Questo fenomeno è detto \textbf{adattamento}.

\begin{figure}[h!]
    \centering
    \includegraphics[width=0.5\textwidth]{pics/09_4.png}
    \caption{Andamento della risposta nei batteri.}
\end{figure}

Affinché ciò avvenga, l'integrale del kernel deve essere nullo:
\begin{equation}
    \int_{-\infty}^{\infty} R(t) dt = 0
    \label{eq:chap9_50}
\end{equation}

Cioè il \textbf{kernel di risposta} deve avere lobi positivi e negativi, in modo tale che l'effetto della perturbazione si annulli nel lungo termine.

\begin{figure}[h!]
    \centering
    \includegraphics[width=0.5\textwidth]{pics/09_5.png}
    \caption{Andamento del kernel di risposta nei batteri.}
\end{figure}

Nei batteri, si misura sperimentalmente la funzione di risposta per la variabile $\alpha(t)$.

Questa risposta mostra una struttura coerente con l'adattamento: un primo lobo positivo seguito da uno negativo. Questo comportamento è compatibile con una dipendenza lineare tra $\alpha$ e $\dot{c}$, e dimostra che, nonostante la linearità, l'adattamento è possibile grazie alla forma specifica del kernel di risposta.