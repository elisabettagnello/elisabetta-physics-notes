%-------------------------------------------------------------------------------
% CAPITOLO 6 (Lezione del 18/03/2025)
%-------------------------------------------------------------------------------
\chapter{Lezione 6}
\label{capitolo_06}
\textit{Data: 18/03/2025}



\section{Comportamento dei Batteri}


Consideriamo un esperimento classico, ad esempio quello di Adler, condotto su una piastra di Petri.
Vengono predisposte due situazioni:

\begin{enumerate}
    \item \textbf{Chemiotassi}: Al centro della piastra viene posta una regione ricca di nutrienti (es. glucosio). Si osserva che i batteri, come \textit{E. coli}, si muovono seguendo il gradiente di concentrazione del nutriente, dirigendosi verso la zona a concentrazione più alta.
    \item \textbf{Diffusione del nutriente}: Il nutriente è distribuito in modo casuale e uniforme nella piastra, ma a una concentrazione molto bassa (es. $10^{-6}$ molare). In questo caso non c'è un gradiente che possa guidare il movimento.
\end{enumerate}

Iniziamo a studiare il comportamento di un singolo batterio in un ambiente uniforme, utilizzando un approccio fisico-matematico. Il punto di partenza è l'equazione di Langevin.

\subsubsection{L'Equazione di Langevin}

L'equazione di Langevin descrive il moto di una particella in un fluido, soggetta a una forza di attrito viscoso e a una forza stocastica che rappresenta gli urti casuali con le molecole del fluido.
\begin{equation}
\label{eq:chap6_1}
m\frac{d\vec{v}}{dt}=-\xi\vec{v}+\delta\vec{F}
\end{equation}
Ricordiamo la soluzione formale di questa equazione:
\begin{equation}
\label{eq:chap6_2}
\vec{v}(t)=\vec{v}_{0}e^{-\frac{t}{\tau}}+\int_{0}^{t} dt^{\prime}e^{-\frac{(t-t')}{\tau}}\frac{\delta \vec{F}(t^{\prime})}{m}
\end{equation}
dove $\tau$ è il tempo di rilassamento della velocità, definito come:
\begin{equation}
\label{eq:chap6_3}
\tau=\frac{m}{\xi}
\end{equation}

\subsubsection{Il Regime Sovrasmorzato (Overdamped)}

Ora, dimostriamo che per un batterio in acqua, il termine inerziale ($m\frac{d\vec{v}}{dt}$) è trascurabile. Questo è noto come limite o regime sovrasmorzato (\textit{overdamped}).

In questo caso, l'equazione di Langevin si semplifica notevolmente:
\begin{equation}
\label{eq:chap6_4}
\frac{d\vec{v}}{dt} \approx 0 \rightarrow \xi\vec{v}=\delta\vec{F}
\end{equation}
Per giustificare questa approssimazione, calcoliamo il valore di $\tau$ per un batterio. Assumiamo che il batterio sia una sfera di raggio $r$.

\begin{itemize}
    \item \textbf{Raggio del batterio}: $r \approx 1 \mu m = 10^{-6} m = 10^{-4} cm$
    \item \textbf{Densità del batterio}: $\rho \approx 1 g/cm^3$
    \item \textbf{Viscosità dell'acqua (a 20°C)}: $\eta_{H_2O} = 1 \text{cp} = 10^{-2} \text{P} = 10^{-2} g \cdot cm^{-1} \cdot s^{-1}$
\end{itemize}

Per una sfera, la massa è $m = \frac{4}{3}\pi r^3 \rho$ e il coefficiente di attrito di Stokes è $\xi = 6\pi\eta r$.

Sostituendo questi valori nella definizione di $\tau$:
\begin{equation}
\label{eq:chap6_5}
\tau = \frac{m}{\xi} = \frac{\frac{4}{3}\pi r^3 \rho}{6\pi\eta r} = \frac{2\rho r^2}{9\eta}
\end{equation}
Inserendo i valori numerici (in unità CGS):
\begin{equation}
\label{eq:chap6_6}
\tau \approx \frac{2 \cdot (1 \frac{g}{cm^3}) \cdot (10^{-4} cm)^2}{9 \cdot (10^{-2} g \cdot cm^{-1} \cdot s^{-1})} \approx 10^{-7} s
\end{equation}
Questo risultato ci dice che il tempo caratteristico su cui la velocità del batterio "dimentica" il suo valore iniziale è estremamente piccolo. Su qualsiasi scala temporale di nostro interesse, possiamo considerare che la velocità si adatti istantaneamente alle forze agenti.

Pertanto, \textbf{siamo nel limite sovrasmorzato}. Il moto del batterio è dominato dagli effetti viscosi del fluido e non dalla sua inerzia.

\subsection{Ipotesi 1: Particella Passiva}

Se il moto è dominato dal bagno termico, possiamo trattare il batterio come una grande particella colloidale che si muove passivamente per diffusione (moto Browniano).

Lo spostamento quadratico medio in questo regime è dato da:
\begin{equation}
\label{eq:chap6_7}
\langle \delta\vec{x}^{2}\rangle=2Dt
\end{equation}
dove $\delta\vec{x} = \vec{x}(t) - \vec{x}_0$ è lo spostamento e $D$ è il coefficiente di diffusione, dato dalla relazione di Einstein-Stokes:
\begin{equation}
\label{eq:chap6_8}
D=\frac{k_{B}T}{\xi}
\end{equation}
Calcoliamo il valore di D e la distanza tipica percorsa.

A temperatura ambiente ($T \approx 300 K$), $k_B T \approx 4.1 \cdot 10^{-21} J$

Il coefficiente di attrito $\xi = 6\pi\eta r \approx 6\pi(10^{-3} Pa \cdot s)(10^{-6} m) \approx 1.88 \cdot 10^{-8} N \cdot s \cdot m^{-1}$
\begin{equation}
\label{eq:chap6_9}
D \approx \frac{4.1 \cdot 10^{-21} J}{1.88 \cdot 10^{-8} N \cdot s \cdot m^{-1}} \approx 2 \cdot 10^{-13} m^2/s = 0.2 \mu m^2/s
\end{equation}
La distanza tipica percorsa in un tempo $t$ è la radice dello spostamento quadratico medio:
\begin{equation}
\label{eq:chap6_10}
\sqrt{\langle\delta x^{2}\rangle} \sim \sqrt{D} \cdot t^{1/2} \approx 1 \cdot t^{1/2} \mu m
\end{equation}
Se consideriamo un tempo di $t=1s$, la distanza percorsa sarebbe dell'ordine di $1 \mu m$

\textbf{Questo risultato non è sperimentalmente vero.} Si osserva che i batteri possono coprire distanze molto maggiori in un secondo (decine di micrometri).

La conclusione è che la nostra ipotesi è sbagliata: i batteri non sono particelle passive. Essi sono agenti attivi, dotati di "motori" (i \textbf{flagelli}) che usano per autopropellersi.

\subsection{Ipotesi 2: Il Modello "Run and Tumble"}

Un modello più realistico per il moto batterico in assenza di gradienti di nutrienti è il cosiddetto \textbf{Run and Tumble} (corsa e riorientamento).

Questo moto è una combinazione di due modalità:

\begin{enumerate}
    \item \textbf{RUN (Corsa)}: I diversi flagelli del batterio ruotano in modo sincrono (in \textbf{senso antiorario}) e si raccolgono in un unico fascio che funziona come un'elica. Questo spinge il batterio in avanti lungo una traiettoria pressoché \textbf{rettilinea}.
    \item \textbf{TUMBLE (Riorientamento)}: I motori flagellari invertono il loro senso di rotazione (\textbf{senso orario}). I flagelli non sono più sincronizzati, si separano e si muovono caoticamente. Questo non produce una spinta netta, ma induce un \textbf{riorientamento casuale} del batterio. Dopodiché, i motori tornano a ruotare in modo sincrono e inizia un nuovo "run" in una nuova direzione.
\end{enumerate}

In questo modello, il moto è dominato dalla meccanica dei flagelli, e l'effetto degli urti casuali con le molecole d'acqua è meno influente sulla traiettoria complessiva.

Assumiamo che non ci sia un gradiente di nutriente.

\begin{itemize}
    \item Caratteristiche dei "Runs"
    \begin{itemize}
        \item \textbf{Durata media}: $\tau_{RUN} \sim 1s$
        \item \textbf{Lunghezza media}: $l_{RUN} \sim 10-20 \mu m$
        \item il che implica una velocità $v_0 \sim 10-20 \mu m/s$
    \end{itemize}
    \item Caratteristiche dei "Tumbles"
    \begin{itemize}
        \item \textbf{Durata media}: $\tau_{TUMBLE} \sim 0.1s$ (molto breve rispetto al run).
        \item \textbf{Spostamento}: Durante un tumble, lo spostamento del batterio è trascurabile. Il suo effetto principale è un cambio di direzione.
    \end{itemize}
\end{itemize}

\subsubsection{Statistica dei Tumble}

Si assume che gli eventi di "tumble" seguano una \textbf{statistica di Poisson}. 

Definiamo $\alpha$ come il rate con cui terminano i "run" e iniziano i "tumble". 

Sperimentalmente, $\alpha = (\tau_{RUN})^{-1} \approx 1s^{-1}$.

La probabilità che un "run" abbia una durata esatta $t$ (cioè, che non avvengano "tumble" per un tempo $t$, e che un "tumble" avvenga nell'intervallo $[t, t+dt]$) è data da:

$P(\text{0 tumble in } t) \cdot P(\text{1 tumble in } dt)$

Usando la distribuzione di Poisson $P(k,t) = \frac{(\alpha t)^k e^{-\alpha t}}{k!}$:

\begin{itemize}
    \item $P(0, t) = e^{-\alpha t}$
    \item $P(1, dt) = \alpha dt e^{-\alpha dt} \approx \alpha dt$ (per $dt$ piccolo)
\end{itemize}

Quindi, la densità di probabilità per la durata di un run è:
\begin{equation}
\label{eq:chap6_11}
P(t_{RUN}=t) = \alpha e^{-\alpha t}
\end{equation}
Questa è una distribuzione esponenziale. La durata media di un run è:
\begin{equation}
\label{eq:chap6_12}
\langle t_{RUN} \rangle = \int_0^\infty t P(t) dt = \int_0^\infty t \alpha e^{-\alpha t} dt = \frac{1}{\alpha} = \tau_{RUN}
\end{equation}
Se la velocità durante un run è costante e pari a $v_0$, possiamo trovare la distribuzione delle lunghezze dei run, $l = v_0 t$:
\begin{equation}
\label{eq:chap6_13}
P(l_{RUN}) dl = P(t) dt \rightarrow P(l) = P(t) \frac{dt}{dl} = (\alpha e^{-\alpha t}) \frac{1}{v_0} = \frac{\alpha}{v_0} e^{-\frac{\alpha l}{v_0}}
\end{equation}
La lunghezza media di un run è:
\begin{equation}
\label{eq:chap6_14}
\langle l \rangle = v_0 \langle t \rangle = \frac{v_0}{\alpha}
\end{equation}
La distribuzione esponenziale implica che la maggior parte dei run sono brevi, ma c'è una "coda" di run occasionalmente molto lunghi.

\subsubsection{Distribuzione degli Angoli di Tumble}

Come si distribuisce l'angolo $\theta$ tra la direzione del run prima del tumble e quella dopo?

Un modello semplice assume che la nuova direzione sia scelta in modo completamente casuale e uniforme sulla superficie di una sfera.

La probabilità di orientarsi in un elemento di angolo solido $d\Omega = \sin\theta d\theta d\phi$ è costante.
\begin{equation}
\label{eq:chap6_15}
P(\theta, \phi) d\theta d\phi = C \sin\theta d\theta d\phi
\end{equation}
Per trovare la distribuzione del solo angolo polare $\theta$, integriamo sull'angolo azimutale $\phi$ (da 0 a $2\pi$):
\begin{equation}
\label{eq:chap6_16}
P(\theta)d\theta = \left( \int_0^{2\pi} C d\phi \right) \sin\theta d\theta = C' \sin\theta d\theta
\end{equation}
Normalizzando la probabilità ($\int_0^\pi P(\theta) d\theta = 1$), si ottiene $C' = 1/2$.
\begin{equation}
\label{eq:chap6_17}
P(\theta) = \frac{1}{2}\sin\theta
\end{equation}
Questo modello teorico predice che l'angolo di riorientamento più probabile è $\theta=90^\circ$.

Tuttavia, i \textbf{dati sperimentali} mostrano una distribuzione diversa, con un picco intorno ai \textbf{70°}.

Questo indica che il riorientamento non è perfettamente isotropo; c'è una certa "memoria" della direzione precedente.

\subsubsection{Effetto della Diffusione Rotazionale}

Durante un "run", che dura circa 1 secondo, il batterio non si muove in linea perfettamente retta, perché gli urti termici del fluido inducono piccole fluttuazioni anche nella sua orientazione (diffusione rotazionale). Questo processo è descritto da:
\begin{equation}
\label{eq:chap6_18}
\langle \delta\theta^2 \rangle \sim 2D_{rot}t
\end{equation}
Tuttavia, il tempo caratteristico per la diffusione rotazionale, $t_B \sim 1/D_{rot}$, è molto più grande della durata media di un run, $\tau_{RUN}$.

Quindi, l'effetto della diffusione rotazionale esiste, ma è molto piccolo e può essere considerato trascurabile prima che avvenga il successivo "tumble".

\subsection{Giustificazione Fluidodinamica del Regime Sovrasmorzato}

Un modo più fondamentale per dimostrare che l'inerzia è trascurabile è analizzare le \textbf{equazioni di Navier-Stokes} che governano il flusso del fluido attorno al batterio. L'equazione è:
\begin{equation}
\label{eq:chap6_19}
\rho \left( \frac{\partial\vec{v}}{\partial t} + (\vec{v}\cdot\nabla)\vec{v} \right) = -\nabla p + \eta\nabla^2\vec{v}
\end{equation}
dove il termine a sinistra rappresenta l'inerzia del fluido e i termini a destra sono le forze di pressione e viscose.

Adimensionalizziamo l'equazione usando scale caratteristiche per il nostro problema:
\begin{itemize}
    \item Lunghezza: $l$ (raggio del batterio)
    \item Velocità: $v_0$ (velocità del batterio)
    \item Tempo: $t_0 = l/v_0$
\end{itemize}
Riscrivendo l'equazione con variabili adimensionali ($\tilde{v} = v/v_0$, etc.), emerge un numero adimensionale fondamentale, il \textbf{numero di Reynolds (Re)}, che moltiplica il termine inerziale:
\begin{equation}
\label{eq:chap6_20}
Re \left[ \frac{\partial\vec{\tilde{v}}}{\partial \tilde{t}} + (\vec{\tilde{v}}\cdot\tilde{\nabla})\vec{\tilde{v}} \right] = -\tilde{\nabla} \tilde{P} + \tilde{\nabla}^2\vec{\tilde{v}}
\end{equation}
Il numero di Reynolds è definito come il rapporto tra le forze inerziali e le forze viscose:
\begin{equation}
\label{eq:chap6_21}
Re = \frac{\rho v_0 l}{\eta}
\end{equation}
Calcoliamolo per un batterio:
\begin{equation}
\label{eq:chap6_22}
Re  \approx 10^{-4}
\end{equation}
Poiché \textbf{Re $<< 1$}, il termine inerziale nell'equazione di Navier-Stokes è completamente trascurabile. L'equazione si riduce all'\textbf{equazione di Stokes}, che bilancia solo le forze di pressione e di viscosità. Questo conferma, da un punto di vista fluidodinamico, che il mondo di un batterio è dominato dalla viscosità e l'inerzia non gioca un ruolo rilevante.

