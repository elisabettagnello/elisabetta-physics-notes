%-------------------------------------------------------------------------------
% CAPITOLO 4 (Lezione del 11/03/2025)
%-------------------------------------------------------------------------------
\chapter{Lezione 4}
\label{capitolo_04}
\textit{Data: 11/03/2025}


\section{Potenziale di Membrana}

Nella lezione precedente, abbiamo cercato di stimare il potenziale di membrana utilizzando la Legge di Fick e l'equazione di Fokker-Planck. La corrente di probabilità $J$ è data da:
\begin{equation}
\label{eq:chap4_1}
J = -D \frac{\partial P}{\partial x} + \frac{{F}_{ext}}{\zeta} P
\end{equation}
Possiamo identificare la densità di probabilità $P$ con la concentrazione $C$. L'equazione diventa quindi l'equazione di Nernst-Planck:
\begin{equation}
\label{eq:chap4_2}
J = -D \frac{\partial C}{\partial x} + \frac{qE}{\zeta} C
\end{equation}
Utilizzando la relazione di Einstein, $D = \frac{k_B T}{\zeta}$, possiamo riscriverla come:
\begin{equation}
\label{eq:chap4_3}
J = D \left[ -\frac{\partial C}{\partial x} + \frac{qE}{k_B T} C \right]
\end{equation}
Considerando una membrana permeabile che separa due compartimenti, uno interno (int) e uno esterno (ext), con concentrazioni $C_{int}$ e $C_{ext}$ e potenziali $V_{int}$ e $V_{ext}$ , abbiamo raggiunto la condizione di equilibrio quando la forza elettrica bilancia quella chimica (diffusiva). Il risultato trovato era:
\begin{equation}
\label{eq:chap4_4}
\ln\left(\frac{C_{ext}}{C_{int}}\right) = -q \frac{\Delta V}{k_B T}
\end{equation}
Questa relazione può essere riscritta per evidenziare una distribuzione di tipo Boltzmann:
\begin{equation}
\label{eq:chap4_5}
\frac{C_{int}}{C_{ext}} = e^{-q \frac{\Delta V}{k_B T}} = e^{-\beta q \Delta V}
\end{equation}
Questa è una distribuzione di Boltzmann, e ci chiediamo se sia una proprietà generale.

\section{Distribuzione di Boltzmann}

Torniamo all'equazione di Fokker-Planck (FP) in una dimensione:
\begin{equation}
\label{eq:chap4_6}
\frac{\partial P}{\partial t} =  D \frac{\partial^2 P}{\partial x^2} - \frac{1}{\zeta} \frac{\partial}{\partial x}[{F}_{ext} P]
\end{equation}
Questa è un'equazione di continuità $\frac{\partial P}{\partial t} + \frac{\partial J}{\partial x} = 0$.

Se la forza è conservativa, può essere derivata da un potenziale $U(x)$:
\begin{equation}
\label{eq:chap4_7}
{F}_{ext} = -\frac{d}{dx}U(x)
\end{equation}
\textbf{Vogliamo verificare se la distribuzione di equilibrio è una distribuzione di Boltzmann.}

La corrente $J(x)$ può essere riscritta in una forma conveniente:
\begin{equation}
\label{eq:chap4_8}
J(x) = -D e^{-\beta U} \frac{\partial}{\partial x} (e^{\beta U} P)
\end{equation}
Proponiamo come soluzione stazionaria la distribuzione di Boltzmann:
\begin{equation}
\label{eq:chap4_9}
P(x) = \frac{e^{-\beta U(x)}}{Z}
\end{equation}
dove $Z$ è una costante di normalizzazione. Sostituendo questa forma nella corrente otteniamo:
\begin{equation}
\label{eq:chap4_10}
J(x) = -D e^{-\beta U} \frac{\partial}{\partial x} \left( e^{\beta U} \frac{e^{-\beta U}}{Z} \right) = -D e^{-\beta U} \frac{\partial}{\partial x} \left( \frac{1}{Z} \right) = 0
\end{equation}
Poiché la corrente di probabilità è zero ovunque ($J=0$), il movimento netto di probabilità è nullo. Dall'equazione di continuità, se $J=0$, allora $\frac{\partial P}{\partial t} = 0$. Questo significa che la distribuzione è stazionaria, cioè non cambia nel tempo.

\textbf{Nota importante}: Abbiamo dimostrato che una distribuzione di Boltzmann è \textit{una} soluzione di equilibrio per l'equazione di FP quando la forza è conservativa. Non abbiamo dimostrato come il sistema raggiunga questo equilibrio, né che questa sia l'unica soluzione possibile.

\textbf{Ogni volta che la forza deriva da un potenziale, il sistema raggiungerà uno stato di equilibrio descritto da una distribuzione di Boltzmann.} Questo stato rappresenta un bilancio tra un contributo energetico (che spinge le particelle verso il minimo del potenziale) e un contributo entropico (dovuto alla temperatura, che tende a sparpagliare le particelle). L'energia domina solo a basse temperature. Se aspettiamo un tempo sufficientemente lungo, il sistema "termalizza", raggiungendo una distribuzione di Boltzmann, proprio come in meccanica statistica.

\section{L'Oscillatore Armonico Stocastico (Ornstein Uhlenbeck)}

Consideriamo ora un nuovo esempio: un oscillatore armonico stocastico. Si tratta di una particella soggetta a una forza elastica, che si muove in un potenziale quadratico.
\begin{equation}
\label{eq:chap4_11}
U(x) = \frac{1}{2} K x^2 \quad \Rightarrow \quad F(x) = -Kx
\end{equation}
Vogliamo studiare questo sistema immerso in un bagno termico, quindi dobbiamo aggiungere un termine di frizione (dissipazione) e un termine di rumore (forza stocastica). L'equazione di Langevin per la particella è:

\begin{equation}
m \frac{d^2 x}{dt^2} = -Kx - \zeta \frac{dx}{dt} + \delta F
\label{eq:chap4_12}
\end{equation}

Essendo un'equazione lineare nella variabile $x$, possiamo risolverla analiticamente.

\subsection{Soluzione dell'Equazione Omogenea}

Iniziamo risolvendo l'equazione omogenea associata ($\delta F = 0$):
\begin{equation}
\label{eq:chap4_13}
\left[ m \frac{d^2}{dt^2} + \zeta \frac{d}{dt} + K \right] x(t) = 0
\end{equation}
Cercando soluzioni della forma $x(t) = A e^{i\omega t}$, otteniamo l'equazione caratteristica per $\omega$:
\begin{equation}
\label{eq:chap4_14}
-m\omega^2 + i\zeta\omega + K = 0
\end{equation}
Dividendo per $-m$ e introducendo la pulsazione propria $\omega_0^2 = K/m$ e introducendo la scala temporale $\tau = m/\zeta$, otteniamo:
\begin{equation}
\label{eq:chap4_15}
\omega^2 - i\frac{1}{\tau}\omega - \omega_0^2 = 0
\end{equation}
Le soluzioni sono:
\begin{equation}
\label{eq:chap4_16}
\omega_{\pm} = \frac{i}{2\tau} \pm \frac{1}{2} \sqrt{4\omega_0^2 - \frac{1}{\tau^2}} = \frac{i}{2\tau} \pm \frac{1}{2\tau} \sqrt{4\omega_0^2\tau^2 - 1}
\end{equation}
Introduciamo il parametro adimensionale $\epsilon = \omega_0^2 \tau^2$. Questo parametro distingue due regimi:
\begin{enumerate}
    \item \textbf{Regime sovrasmorzato (overdamped)}: $4\epsilon < 1$
    
    L'argomento della radice è negativo, le soluzioni $\omega_\pm$ sono puramente immaginarie, il che corrisponde a decadimenti esponenziali reali, senza oscillazioni. L'attrito domina e "distrugge" l'inerzia prima che il sistema possa compiere un'oscillazione.
    
    \item \textbf{Regime sottosmorzato (underdamped)}: $4\epsilon > 1$
    
    L'argomento della radice è positivo, $\omega_\pm$ hanno una parte reale e una immaginaria, portando a oscillazioni smorzate.
\end{enumerate}
Per semplicità, analizziamo in dettaglio il \textbf{regime sovrasmorzato} con $\epsilon \ll 1$. In questo limite, possiamo approssimare le radici $\omega_\pm$:
\begin{equation}
\label{eq:chap4_17}
\omega_+ \approx \frac{i}{\tau}(1-\epsilon), \quad \omega_- \approx \frac{i\epsilon}{\tau}
\end{equation}
La soluzione omogenea generale è:
\begin{equation}
\label{eq:chap4_18}
x_{om}(t) = a e^{i\omega_+ t} + b e^{i\omega_- t}
\end{equation}
Imponendo le condizioni iniziali $x(0)=0$ e $v(0)=v_0$, si possono determinare le costanti. Il risultato per la posizione media è:
\begin{equation}
\label{eq:chap4_19}
\langle x(t) \rangle = x_{om}(t) = \frac{v_0 \tau}{1-2\epsilon} \left[ e^{-\frac{t\epsilon}{\tau}} - e^{-\frac{t(1-\epsilon)}{\tau}} \right]
\end{equation}

\subsection{Soluzione completa con il metodo del propagatore}

Per trovare la soluzione completa, usiamo il metodo del propagatore (o funzione di Green). La soluzione per $x(t)$ è data dalla somma della soluzione omogenea e di una convoluzione tra il propagatore $G(t-t')$ e la forza stocastica $\delta F(t')$.
\begin{equation}
\label{eq:chap4_20}
\left(m \frac{d^2}{dt^2} + \zeta \frac{d}{dt} + k\right) G(t) = \delta F(t) 
\end{equation}
$G(t)$ e la sua trasformata di Fourier $G(\omega)$ sono legate dalla seguenti relazioni:
\begin{align}
\label{eq:chap4_21}
G(t)&=\int_{-\infty}^{+\infty} \frac{d\omega}{2\pi} e^{i \omega t} G(\omega) \\
G(\omega)&=\int_{-\infty}^{+\infty} dt e^{-i \omega t} G(t)
\end{align}
Passiamo allo spazio di Fourier:
\begin{equation}
\label{eq:chap4_22}
\hat{G}(\omega) = \frac{1}{-m (\omega - \omega_+)(\omega - \omega_-)} 
\end{equation}
La soluzione particolare è data da:
\begin{equation}
\label{eq:chap4_23}
x_{\text{particolare}}(t) = \int_{-\infty}^{+\infty} G(t - t') \delta F(t') dt'
\end{equation}
Troviamo G(t) applicando il teorema dei residui:
\begin{equation}
\label{eq:chap4_24}
G(t) = 2\pi i \sum \text{Res}[\omega_i] = \frac{2\pi i}{2 \pi} \left(-\frac{1}{m}\right) \left[\frac{e^{i\omega_+t}}{(\omega_+ - \omega_-)} + \frac{e^{i\omega_-t}}{(\omega_- - \omega_+)}\right]
\end{equation}
La soluzione particolare quindi è:
\begin{equation}
\label{eq:chap4_25}
x_{\text{particolare}} = \frac{\tau}{m(2\epsilon - 1)} \int_0^t dt' \left[e^{-(1-\epsilon)\frac{(t - t')}{\tau}} - e^{-\frac{\epsilon(t - t')}{\tau}}\right] \delta F(t')
\end{equation}
L'equazione completa per la posizione quindi è:
\begin{equation}
\label{eq:chap4_26}
\begin{aligned}
x(t) =&\frac{v_0 \tau}{(2\epsilon - 1)} \left[e^{-(1-\epsilon)\frac{t}{\tau}} - e^{-\frac{\epsilon t}{\tau}}\right]  + \\
&\frac{\tau}{m(2\epsilon - 1)} \int_0^t dt' \left[e^{-(1-\epsilon)\frac{(t - t')}{\tau}} - e^{-\frac{\epsilon(t - t')}{\tau}}\right] \delta F(t')
\end{aligned}
\end{equation}
Poiché $\langle \delta F(t)  \rangle = 0$, e $\langle x(t)  \rangle = x_{homo}$ per studiare l’evoluzione del moto, considerando agli gli effetti della diffusione, consideriamo $\langle x^2(t)  \rangle$.

\subsection{Spostamento Quadratico Medio}
Lo spostamento quadratico medio (MSD)  , $\langle x^2(t) \rangle$, è la quantità che descrive l'esplorazione dello spazio da parte della particella.
\begin{equation}
\label{eq:chap4_27}
\begin{aligned}
\langle x^2 (t) \rangle = &x_{homo}^2 + \\
&\frac{\tau^3 2K_b T \zeta }{m^2 (2 \epsilon -1)^2} \left[ \frac{1}{2 \epsilon} (1-e^{-2 \epsilon \frac{t}{\tau}}) - 2 (1-e^{- \frac{t}{\tau}}) + \frac{1 }{ 2( 1-\epsilon)} (1-e^{-2 (1-\epsilon) \frac{t}{\tau}})  \right]
\end{aligned}
\end{equation}
Dove il secondo termine è stato ottenuto facendo un doppio integrale in $dt$ e $dt'$ e usando la relazione $<\delta F(t)\delta F(t')> = 2 K_b T \zeta  \delta(t-t')$.

La fisica del fenomeno è la seguente:
\begin{enumerate}
    \item Agli istanti iniziali, la particella si muove in \textbf{modo balistico}, a causa della sua velocità iniziale $v_0$
    \item Successivamente, l'attrito smorza il moto, distruggendo la memoria della condizione iniziale.
    \item La particella inizia a muoversi per diffusione, senza ancora "sentire" l'effetto del potenziale confinante.
    \item Quando la particella si allontana troppo dall'origine, la forza di richiamo del potenziale diventa significativa e la "respinge" indietro.
    \item Infine, il sistema termalizza all'interno del potenziale. Lo spostamento quadratico medio non cresce più all'infinito (come nella diffusione libera) ma si satura a un valore finito, dettato dall'equilibrio tra energia potenziale e energia termica, in accordo con il teorema di equipartizione dell'energia.
\end{enumerate}
Analizziamo i diversi regimi temporali per $\langle x^2(t) \rangle$:

\subsubsection{Regime balistico ($t \ll \tau$)}
\begin{equation}
\label{eq:chap4_28}
\langle x^2 (t) \rangle = x_{homo}^2 +  \Omega^2
\end{equation}
\begin{itemize}
    \item $\Omega^2 \simeq \tau^3$
    \begin{equation}
    \label{eq:chap4_29}
    \begin{aligned}
    x_{\text{homo}} & = \frac{v_0 \tau}{(2\epsilon - 1)} \left[e^{-(1-\epsilon)\frac{t}{\tau}} - e^{-\frac{\epsilon t}{\tau}}\right] \simeq C \left[ 1- (1-\epsilon)\frac{t}{\tau} - 1 + \frac{\epsilon t}{\tau}\right] \\ & \simeq C \left[ (2 \epsilon -1)\frac{t}{\tau} \right]
    \end{aligned}
    \end{equation}
    \begin{equation}
    \label{eq:chap4_30}
    \begin{aligned}
    x_{\text{homo}}^2 &  \simeq \left( \frac{v_0 \tau}{(2\epsilon - 1)} \right)^2 (2 \epsilon -1)^2 \frac{t}{\tau}^2 = (v_0 t)^2
    \end{aligned}
    \end{equation}
    \item $x_{homo}^2 \simeq (v_0 t)^2$
\end{itemize}
L'MSD cresce quadraticamente con il tempo:
\begin{equation}
\label{eq:chap4_31}
\langle x^2(t) \rangle \approx v_0^2 t^2
\end{equation}
Il grafico log-log di $\langle x^2(t) \rangle$ VS $t$ ha una pendenza di 2.

\subsubsection{Regime diffusivo ($\tau \ll t \ll \tau/2\epsilon$)}
\begin{equation}
\label{eq:chap4_32}
\begin{aligned}
x_{\text{homo}} & \simeq C \left[e^{-\frac{t}{\tau}} e^{\epsilon \frac{t}{\tau}} - e^{-\frac{\epsilon t}{\tau}}\right] =  \frac{v_0 \tau}{(2\epsilon - 1)}  (0-1) = -\frac{v_0 \tau}{(2\epsilon - 1)}
\end{aligned}
\end{equation}
\begin{equation}
\label{eq:chap4_33}
\begin{aligned}
x_{\text{homo}}^2 &  \simeq \left( \frac{v_0 \tau}{(2\epsilon - 1)} \right)^2
\end{aligned}
\end{equation}
\begin{equation}
\label{eq:chap4_34}
\begin{aligned}
\Omega^2 &\simeq \frac{\tau^3 2K_b T \zeta }{m^2 (2 \epsilon -1)^2} \left[ \frac{1}{2 \epsilon}  (1-1+2 \epsilon \frac{t}{\tau} + o(t^2))- 2  + \frac{1 }{ 2( 1-\epsilon)}   \right] \\ &\simeq \frac{\tau^3 2K_b T \zeta }{m^2 (2 \epsilon -1)^2}  \frac{t}{\tau} = \frac{\ 2K_b T t}{\zeta (2 \epsilon -1)^2} = \frac{\ 2Dt}{(2 \epsilon -1)^2}
\end{aligned} 
\end{equation}
Nell’espressione di $\Omega^2$ si passa dalla prima alla seconda riga tenendo a mente che solo il termine proporzionale a $t$ è quello rilevante.

L'MSD cresce linearmente con il tempo, come in un moto browniano libero.
\begin{equation}
\label{eq:chap4_35}
\langle x^2(t) \rangle \approx 2Dt
\end{equation}
Il grafico log-log ha pendenza 1.

\subsubsection{Regime di equilibrio ($t \gg \tau/2\epsilon$)}
Tutti i termini esponenziali decadono a zero. L'MSD raggiunge un valore di saturazione costante, determinato dal confinamento del potenziale.
\begin{equation}
\label{eq:chap4_36}
\langle x^2 \rangle_{eq} = \frac{k_B T}{K}
\end{equation}
Questo risultato è esattamente ciò che predice il \textbf{teorema di equipartizione dell'energia} per un grado di libertà quadratico in un sistema a temperatura $T$. Il grafico log-log diventa a pendenza 0.

\begin{figure}[h!]
    \centering
    \includegraphics[width=0.7\textwidth]{pics/04_1.png}
    \caption {Mean-square displacement  per un processo di Ornstein-Uhlenbeck.}
    \label{fig:chap4_msd}
\end{figure}

\section* {Riassunto dei risultati trovati fino ad ora:}
\begin{itemize}
    \item Dinamica diffusiva con rumore
    \begin{equation}
    \label{eq:chap4_37}
    \langle \delta x^2 \rangle = \frac{2 \; d \; k_b \; T \; t}{\zeta} = 2Dt
    \end{equation}
    \begin{equation}
    \label{eq:chap4_38}
    \langle v^2 \rangle_{eq} = \frac{d B}{m \zeta}  =  \frac{d \; k_b T}{m}
    \end{equation}
    \item Diffusion drift (forza costante)
    \begin{equation}
    \label{eq:chap4_39}
    \langle \bar{v}(t) \rangle = v_0 e^{- \frac{\zeta}{m}t} + \frac{\bar{F}_{ext}}{\zeta} \left( 1-e^{-\frac{\zeta}{m} t }\right)
    \end{equation}
    \begin{equation}
    \label{eq:chap4_40}
    \langle \bar{v} \rangle_{\infty} =  \frac{\bar{F}_{ext}}{\zeta}
    \end{equation}
    \begin{equation}
    \label{eq:chap4_41}
    \bar{J} = -D  \; \nabla \cdot P + \frac{\bar{F}_{ext}}{\zeta} P
    \end{equation}
    \begin{equation}
    \label{eq:chap4_42}
    \frac{\partial P}{\partial t} = D  \nabla^2  P - \nabla \left( \frac{\bar{F}_{ext}}{\zeta} P \right)
    \end{equation}
    \item Diffusione con potenziale confinante
    \begin{equation}
    \label{eq:chap4_43}
    \begin{aligned}
    \langle x^2 (t) \rangle = &x_{homo}^2 +  \Omega^2
    \end{aligned}
    \end{equation}
    \begin{equation}
    \label{eq:chap4_44}
    \langle x^2 (t) \rangle_{eq} = \frac{k_B T}{k}
    \end{equation}
\end{itemize}

