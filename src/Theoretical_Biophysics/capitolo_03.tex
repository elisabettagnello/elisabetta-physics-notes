%-------------------------------------------------------------------------------
% CAPITOLO 3 (Lezione del 07/03/2025)
%-------------------------------------------------------------------------------
\chapter{Lezione 3}
\label{capitolo_03}
\textit{Data: 07/03/2025}

\section{Equazione di diffusione}

Ripartiamo dall’equazione di Langevin nel caso overdamped:

\begin{equation}
\frac{d\bar{x}}{dt} \zeta = \delta \bar{F}
\label{eq:chap3_langevin_overdamped}
\end{equation}

Abbiamo visto che quello descritto da questa equazione è un processo stocastico non differenziabile. La descrizione matematica corretta si ottiene riscalando:

\begin{align} 
dx&=dW \label{eq:chap3_wiener_dx} \\ 
<dW^2>&=2Ddt \label{eq:chap3_wiener_dw2}
\end{align}

$W^2$ è una quantità casuale che scala con l'intervallo di tempo.

Suddividiamo l'asse temporale in piccoli intervalli, ognuno dei quali è dato da $dt$, e indichiamo con $dx$ lo spostamento effettuato in ogni intervallino. La distanza totale percorsa in un tempo t è:

\begin{equation} 
x(t) = \sum^t_{\tau=0} dx 
\label{eq:chap3_distanza_totale}
\end{equation}

Questo è esattamente lo stesso risultato ottenuto con il formalismo del random walk discreto. 

Grazie al teorema del limite centrale, possiamo concludere che la distribuzione della distanza percorsa in un tempo $t$ segue una distribuzione Gaussiana:

\begin{equation} 
P(x,t) = \frac{e^{-\frac{x^2}{4Dt}}}{\sqrt{4\pi Dt}}  
\label{eq:chap3_gaussiana}
\end{equation}

Questa distribuzione (nel caso multi-dimensionale) soddisfa la seguente equazione differenziale:

\begin{equation} 
\frac{\partial P}{\partial t} = D \; \nabla^2 P 
\label{eq:chap3_equazione_diffusione}
\end{equation}

che è nota come \textbf{equazione della diffusione}.

L'equazione della diffusione può essere riscritta come equazione di continuità:

\begin{equation} 
\frac{\partial P}{\partial t} + \nabla \cdot \bar{J} = 0 
\label{eq:chap3_continuita}
\end{equation}

Dove $\bar{J}$ è la \textbf{corrente di probabilità}, data dalla \textbf{legge di Fick}:

\begin{equation} 
\bar{J} = -D  \; \nabla \cdot P 
\label{eq:chap3_fick}
\end{equation}

Questa relazione implica che esiste una corrente che è diretta dalle regioni con maggiore concentrazione verso quelle con minore concentrazione.

\subsection{Derivazione della legge di Fick e d8 diffusione}

Consideriamo il caso unidimensionale e discretizziamo lo spazio in intervalli di lunghezza $\delta x$, in modo che:

\begin{align} 
<\delta x^2> &= 2D \delta t \label{eq:chap3_dx2_medio} \\  
\delta x &= \sqrt{2D \delta t } \label{eq:chap3_dx_medio}
\end{align}

La (\ref{eq:chap3_dx_medio}) rappresenta il comportamento medio, non il comportamento reale di una singola traiettoria.

Supponiamo di avere una popolazione che al tempo zero si trova nell’origine. Dalla posizione iniziale, dopo un certo tempo $t$, la popolazione si muoverà verso valori più alti dell'asse. Consideriamo la posizione $x$, $x - \delta x$ a sinistra e $x + \delta x$ a destra. Ci saranno più particelle a sinistra e meno a destra. 

Vogliamo calcolare la corrente sulla superficie di separazione $x$. \textbf{La corrente è definita come il numero di particelle che attraversano una superficie di separazione in unità di tempo, diviso per la superficie.} In questo caso, la superficie è solo un punto, quindi non è necessario dividere.

Poiché il movimento è casuale, metà delle particelle che si trovano su $x$ andrà a sinistra e metà andrà a destra (si muovono in direzioni opposte, quindi dobbiamo sottrarre un numero all'altro). 

\begin{equation} 
J(x, t) = \frac{1}{2}[N(x- \delta x)- N(x)] \cdot \frac{1}{\delta t} 
\label{eq:chap3_corrente_discreta}
\end{equation}

Il numero di particelle in un dato intervallo è definito come: 

\begin{equation}  
N(x,t) = P(x, t) \delta x 
\label{eq:chap3_numero_particelle}
\end{equation} 

dove $P(x, t)$ è la densità di probabilità.

Segue:

\begin{equation} 
\begin{aligned} J(x, t) &= \frac{1}{2}[P(x- \delta x)- P(x)] \cdot \frac{\delta x}{\delta t} \\ 
&= -\frac{1}{2}[-P(x- \delta x)+ P(x)] \cdot \frac{1}{\delta x} \frac{\delta x^2}{\delta t} \\  
&= -\frac{1}{2} \frac{\partial P}{\partial x} 2D \\ 
&= -D \frac{\partial P}{\partial x} 
\end{aligned} 
\label{eq:chap3_derivazione_fick}
\end{equation}

Dove abbiamo considerato il limite di $\delta x$ che tende a zero, in modo da poter definire la derivata. Attraverso questo processo, abbiamo derivato la legge di Fick.

Ritroviamo ora l’equazione di diffusione:

\begin{equation}  
\delta N(x,t) = [J(x -\delta x, t)-J(x,t) ]\delta t  
\label{eq:chap3_delta_N}
\end{equation} 

\begin{equation} 
\begin{aligned} 
N(x,t+\delta t)-N(x,t)  &= \delta x [P(x, t+\delta t)-P(x,t)]\\  
&= - \delta t [J(x,t)-J(x -\delta x, t)] 
\end{aligned} 
\label{eq:chap3_deriv_diff_1}
\end{equation}

\begin{equation}  
\frac{P(x, t+\delta t)-P(x,t)}{\delta t} = - \frac{J(x,t)-J(x -\delta x, t)}{\delta x}  
\label{eq:chap3_deriv_diff_2}
\end{equation} 

\begin{equation} 
\frac{\partial P}{\partial t} = - \frac{\partial J}{\partial x}  
\label{eq:chap3_deriv_diff_finale}
\end{equation}

\section{Metabolismo nei batteri}

Possiamo applicare le equazioni della diffusione anche ad un problema biologico, come il flusso di molecole di ossigeno verso un batterio in un lago. Schematizziamo il batterio come una sfera di raggio $R$. Ogni volta che una molecola di ossigeno, diffondendosi in acqua, arriva sulla superficie del nostro batterio, viene assorbita. La corrente di particelle che raggiunge la superficie del batterio è correlata alla concentrazione di ossigeno e alla dimensione del batterio. 

Il batterio ha bisogno di una certa quantità di ossigeno per unità di tempo per sopravvivere. Poiché si trova nel lago e le molecole di ossigeno continueranno a diffondersi, continueranno ad arrivare sulla superficie del batterio. Il fabbisogno (intake), la quantità di nutrienti di cui una cellula ha bisogno per sopravvivere, cresce con il volume della cellula stessa. Possiamo assumere che il fabbisogno sia una costante:

\begin{equation} 
I = \alpha R^3 
\label{eq:chap3_fabbisogno}
\end{equation}

Ciò che dobbiamo calcolare è il numero di particelle di ossigeno che arrivano sulla superficie del batterio per unità di tempo, ovvero la corrente di molecole che arriva sulla superficie del batterio. L'equazione che descrive il movimento delle molecole di ossigeno è precisamente l'equazione di diffusione. 

Formalizziamo il problema chiamando $c(x, t)$ la concentrazione di molecole di ossigeno nel lago (è l'analogo della probabilità P usata precedentemente). In assenza del batterio, possiamo aspettarci che questa distribuzione sia completamente uniforme. Tuttavia, una volta inserito il batterio nel lago, poiché inizierà ad assorbire l'ossigeno, modificherà la concentrazione. 

Poniamo l'origine del nostro sistema di riferimento al centro del batterio. Possiamo assumere che quando $x>>1$, ovvero a distanza molto grande dal batterio, la concentrazione sarà uguale alla concentrazione senza batterio: $c(x>>1,t)=c_0$. Mentre quando ci avviciniamo al batterio, ci sarà una modifica della concentrazione a causa del fatto che ogni volta che le molecole arrivano al confine, esse scompaiono.

L'equazione che descrive come la concentrazione varia nello spazio e nel tempo è l'equazione di diffusione:

\begin{equation} 
\frac{\partial c}{\partial t} + \nabla \cdot J = 0 
\label{eq:chap3_diffusione_concentrazione}
\end{equation}

Non siamo interessati alla dipendenza temporale della concentrazione. Dopo un certo tempo transitorio, il sistema infatti raggiungerà un nuovo stato stazionario, per cui possiamo supporre che la concentrazione non dipenda dal tempo $c(x,t) = c(x)$. Nello \textbf{stato stazionario} si ha:

\begin{equation} 
\frac{\partial c}{\partial t} = 0 \quad \longrightarrow \quad \nabla \cdot J = 0 
\label{eq:chap3_stato_stazionario}
\end{equation}

A questo punto si può applicare il teorema della divergenza. Se calcolo il flusso del vettore corrente attraverso una superficie chiusa, questo deve essere uguale a zero:

\begin{equation} 
\int \bar{J} \cdot \bar{n} dS =0  
\label{eq:chap3_divergenza}
\end{equation}

Quindi si può considerare una superficie chiusa in questo spazio tridimensionale e calcolare il flusso della corrente attraverso di essa. Poiché il batterio è sferico, anche il problema deve avere una \textbf{simmetria sferica}, quindi tutto è completamente \textbf{isotropo} rispetto agli angoli e dipende solo dalla distanza dall'origine: $\bar{J}(\bar{x})=J(r)$

Per applicare il teorema della divergenza, scelgo un volume costituito da due sfere concentriche. La sfera più piccola ha raggio $R_1$ e la sfera più grande ha raggio $R_2$.

Poiché $J$ ha una simmetria sferica, è un vettore orientato radialmente verso il centro del sistema di riferimento. Quindi la corrente entrerà perpendicolarmente al mio volume dall'esterno verso l'interno, e la direzione sarà entrante sulla sfera esterna e uscente sulla sfera interna.

Applicando il teorema della divergenza si trova:

\begin{align} 
&4\pi R_1^2J(R_1) - 4\pi R_2^2J(R_2) = 0 \quad ,\forall R_1,R_2 \label{eq:chap3_flusso_sfere} \\ 
&4\pi r^2J(r)=A=costante  \label{eq:chap3_corrente_costante} \\ 
&J(r) = \frac{A}{4\pi r^2} \label{eq:chap3_corrente_r}
\end{align}

Ora voglio sfruttare la relazione che collega la corrente alla concentrazione stessa. La definizione generale della corrente è:

\begin{equation} 
\bar{J} = -D  \; \nabla \cdot c 
\label{eq:chap3_fick_concentrazione}
\end{equation}

Poiché il problema ha una simmetria sferica, è conveniente riesprimere il gradiente in coordinate radiali. Sappiamo che le coordinate che dipendono dagli angoli sono zero; l'unica non nulla quindi è quella che dipende dalla distanza dall'origine:

\begin{equation} 
J(r) = -D \frac{d}{dr} c(r) 
\label{eq:chap3_corrente_radiale}
\end{equation}

È conveniente integrare dalla superficie del batterio (che ha raggio R) perché so che ogni volta che le molecole di ossigeno arrivano lì, scompaiono. Quindi, so che la concentrazione lì deve essere zero. Segue che:

\begin{equation} 
\begin{aligned} 
c(r) &= -\frac{1}{D} \int_R^r J(r') \, dr' +c(R)\\  
&= -\frac{1}{D} \int_R^r J(r') \, dr' \\ 
&= -\frac{1}{D} \int_R^r  dr' \frac{A}{4\pi r'^2} \\ 
&= \frac{A}{4 \pi D} (\frac{1}{r}-\frac{1}{R}) 
\end{aligned}  
\label{eq:chap3_concentrazione_r}
\end{equation}

Per $r \to + \infty$ la concentrazione deve tendere a $c_0$, quindi posso riscrivere la costante $A$ in termini della concentrazione stazionaria delle molecole di ossigeno nel lago:

\begin{equation} 
c(r \to \infty) = c_0 = - \frac{A}{4 \pi D R} \quad \longrightarrow \quad A= - 4 \pi c_0 D R 
\label{eq:chap3_costante_A}
\end{equation}

Sostituiamo nella (\ref{eq:chap3_corrente_r}) e nella (\ref{eq:chap3_concentrazione_r}):

\begin{align} 
J(r)  &= - \frac{DRc_0}{r^2} \label{eq:chap3_corrente_r_finale} \\ 
c(r) &= c_0 \left(1-\frac{R}{r}\right) \label{eq:chap3_concentrazione_r_finale}
\end{align}

La (\ref{eq:chap3_concentrazione_r_finale}) è un'espressione molto intuitiva, poiché ci dice che quando $r$ tende all'infinito otteniamo $c_0$, e che la concentrazione diminuisce man mano che ci avviciniamo al batterio.

A questo punto dobbiamo calcolare il numero di particelle che arrivano sulla superficie del batterio per unità di tempo, e che conseguentemente vengono assorbite. Dobbiamo calcolare $J(R)$ moltiplicato per la superficie del batterio:

\begin{equation} 
J(R) \cdot 4\pi R^2 = \frac{DRc_0}{R^2} \cdot 4\pi R^2 = 4\pi D R c_0
\label{eq:chap3_flusso_batterio}
\end{equation}

Imponiamo la condizione di sopravvivenza, per cui \textbf{il batterio sarà in grado di sopravvivere se il numero di molecole di ossigeno che arriva sulla superficie è maggiore del suo fabbisogno}:

\begin{equation} 
4\pi D R c_0 \geq \alpha R^3 
\label{eq:chap3_condizione_sopravvivenza}
\end{equation}

Ricaviamo così la \textbf{dimensione critica del batterio} (batteri più piccoli di questa dimensione critica possono sopravvivere e batteri più grandi non possono):

\begin{equation} 
R \leq \left( \frac{4\pi D c_0}{\alpha} \right)^{\frac{1}{2}} 
\label{eq:chap3_dimensione_critica}
\end{equation}

\section{Equazione di Langevin in presenza di forze esterne}

Forma generale dell’equazione di Langevin:

\begin{equation} 
\begin{cases}
\frac{d \bar{v}}{dt} &= \bar{F}_{ext}(\bar{x}) - \zeta \bar{v} + \delta \bar{F}, \\
\frac{d\bar{x}}{dt} &= \bar{v} 
\end{cases} 
\label{eq:chap3_langevin_generale_forze}
\end{equation}

Consideriamo il caso in cui $F_{ext} \neq 0$, ma facciamo un'assunzione semplificata: assumiamo che questa forza sia una costante e non dipenda dalla posizione.

Possiamo usare le tecniche che abbiamo adottato per derivare la soluzione dell'equazione completa in assenza di forza e poi sostituire $\delta F$ con $\delta F + F_{ext}$, in questo modo troviamo:

\begin{equation} 
\bar{v}(t)=v_0 e^{- \frac{\zeta}{m}t} + \int_0^t e^{-\frac{\zeta}{m} (t-t')} \left[ \frac{\delta \bar{F}(t')}{m} + \frac{\bar{F}_{ext}(t')}{m}\right]dt' 
\label{eq:chap3_sol_v_forze}
\end{equation}

Se $F_{ext}$ è costante, possiamo portarla fuori dall’integrale:

\begin{equation} 
\bar{v}(t)=v_0 e^{- \frac{\zeta}{m}t} + \frac{\bar{F}_{ext}}{m} \int_0^t e^{-\frac{\zeta}{m} (t-t')} dt' + \int_0^t e^{-\frac{\zeta}{m} (t-t')} \frac{\delta \bar{F}(t')}{m} dt' 
\label{eq:chap3_sol_v_forze_cost}
\end{equation}

Risolviamo il primo integrale:

\begin{equation} 
\frac{\bar{F}_{ext}}{m} \int_0^t e^{-\frac{\zeta}{m} (t-t')} dt' = \frac{\bar{F}_{ext}}{m}  \frac{m}{\zeta} e^{-\frac{\zeta}{m} t} \left( e^{+\frac{\zeta}{m} t }-1\right) = \frac{\bar{F}_{ext}}{\zeta} \left( 1-e^{-\frac{\zeta}{m} t }\right) 
\label{eq:chap3_integrale_forze}
\end{equation}

Sostituendo troviamo una soluzione completa in presenza di rumore e di una forza esterna costante:

\begin{equation} 
\bar{v}(t)=v_0 e^{- \frac{\zeta}{m}t} + \frac{\bar{F}_{ext}}{\zeta} \left( 1-e^{-\frac{\zeta}{m} t }\right) + \int_0^t e^{-\frac{\zeta}{m} (t-t')} \frac{\delta \bar{F}(t')}{m} dt' 
\label{eq:chap3_sol_v_completa}
\end{equation}

Ora, calcoliamo il valore medio di $v$. Se non avessi la forza esterna, otterrei solo il contributo dovuto alla condizione iniziale, poiché il valore medio di $\delta F$ è nullo. In presenza di forze esterne, invece, si ha:

\begin{equation} 
\langle \bar{v}(t) \rangle = v_0 e^{- \frac{\zeta}{m}t} + \frac{\bar{F}_{ext}}{\zeta} \left( 1-e^{-\frac{\zeta}{m} t }\right) 
\label{eq:chap3_media_v}
\end{equation}

Per $t \to \infty$ (ovvero $t>>\frac{m}{\zeta}=\tau$) si ha:

\begin{equation} 
\langle \bar{v} \rangle_{\infty} =  \frac{\bar{F}_{ext}}{\zeta}  
\label{eq:chap3_drift_velocity}
\end{equation}

L’espressione trovata coincide con quella della drift velocity del moto browniano.

Se guardo a scale temporali molto più grandi di quella microscopica, vedrò che particella browniana si muoverà in modo erratico, ma in media si muoverà lungo la direzione della forza esterna.

\subsection{Equazione di Fokker-Planck}

Analizziamo cosa succede a livello della distribuzione di probabilità, ovvero come evolve nel tempo la probabilità di trovare una particella nella posizione $x$ al tempo $t$. Per determinare il comportamento risultante, possiamo fare riferimento all'equazione della diffusione. 

\begin{equation} 
\frac{\partial P}{\partial t} + \nabla \cdot \bar{J} = 0 
\label{eq:chap3_continuita_fp}
\end{equation}

Questa equazione deriva dal bilancio tra l'afflusso e il deflusso di particelle in un intervallo dato, in assenza di forze esterne. 

Consideriamo ora l'effetto di una forza esterna che spinge le particelle in una direzione specifica.

Se consideriamo la forza esterna, le particelle acquisiscono una velocità di deriva $v = \frac{\bar{F}_{ext}}{\zeta}$ per cui $\delta x = v\delta t$.

Quindi si ha:

\begin{equation} 
\bar{J}_{drift} = v \cdot P =  \frac{\bar{F}_{ext}}{\zeta} \cdot P 
\label{eq:chap3_corrente_drift}
\end{equation} 

La corrente totale $\bar{J}= -D  \; \nabla \cdot P +\bar{J}_{drift}$ delle particelle sarà:

\begin{equation} 
\bar{J} = -D  \; \nabla \cdot P + \frac{\bar{F}_{ext}}{\zeta} P
\label{eq:chap3_corrente_totale}
\end{equation}

Imponendo la conservazione della probabilità ($\frac{\partial P}{\partial t} =- \nabla \cdot \bar{J}$), otteniamo l'\textbf{equazione di Fokker-Planck}:

\begin{equation} 
\frac{\partial P}{\partial t} = D  \nabla^2  P - \nabla \left( \frac{\bar{F}_{ext}}{\zeta} P \right)  
\label{eq:chap3_fokker_planck}
\end{equation}

\subsection{Potenziale di Membrana}

Consideriamo ora un'applicazione biologica dell'equazione di Fokker-Planck per stimare la differenza di potenziale elettrico attraverso la membrana di un neurone.

I neuroni contengono ioni come $Na^+$ e $Cl^-$ con concentrazioni diverse dentro e fuori la membrana. Supponiamo che la membrana sia permeabile al sodio ($Na^+$), con una concentrazione interna $C_{\text{int}}$ maggiore di quella esterna $C_{\text{out}}$. Gli ioni si muoveranno per diffusione dall'interno all'esterno, creando uno squilibrio di carica che genera un campo elettrico opposto al flusso diffusivo.

All'equilibrio, la corrente totale si annulla:

\begin{equation} 
\bar{J} = -D  \frac{dc}{dx} + \frac{\bar{F}_{ext}}{\zeta} c =0 \quad \longrightarrow \quad D  \frac{dc}{dx} = \frac{\bar{F}_{ext}}{\zeta} c
\label{eq:chap3_corrente_nulla}
\end{equation}

Integrando questa equazione e usando la relazione tra diffusione e temperatura ($D = \frac{k_B T}{\zeta}$), otteniamo la \textbf{relazione di Nernst}:

\begin{align} 
&K_b T \int_{in}^{out} \frac{dc}{c} = \int_{in}^{out} F_{ext} dx = e  \int_{in}^{out} E dx \label{eq:chap3_nernst_1} \\ 
&K_b T \left[ln(c_{out})-ln(c_{in})\right] = -e (V_{out}-V_{in}) \label{eq:chap3_nernst_2} \\ 
&c_{out}= c_{in} e^{-\beta e \Delta V} \label{eq:chap3_nernst_finale}
\end{align}

Usando valori tipici ($T \approx 300$ K, $C_{\text{int}} / C_{\text{out}} \approx 10$), otteniamo \textbf{$\Delta V \approx 58$ mV,} un valore in buon accordo con i dati sperimentali.
