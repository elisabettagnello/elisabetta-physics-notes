\chapter{Lezione 15}
\label{chap:lezione_15}

\begin{flushright}
\textit{Data: 12/11/2025}
\end{flushright}

\section{Funzioni di Correlazione nel Modello Gaussiano}

Abbiamo visto che nel Modello Gaussiano  la correlazione a due punti è data dall'inverso della matrice di interazione:
\begin{equation}
    \langle \sigma_i \sigma_k \rangle = A_{ik}^{-1}
\end{equation}
Tuttavia, se si conoscono le correlazioni a due corpi, in questo modello si conosce tutto.
Infatti, una funzione di correlazione a $2n$ punti può essere espressa come somma di prodotti delle funzioni a due punti (Teorema di Wick). Ad esempio, per la funzione a 4 punti:
\begin{equation}
    \langle \sigma_i \sigma_j \sigma_k \sigma_m \rangle = A_{ij}^{-1} A_{km}^{-1} + A_{ik}^{-1} A_{jm}^{-1} + A_{im}^{-1} A_{jk}^{-1}
\end{equation}


\subsubsection{Rappresentazione Diagrammatica}

Possiamo visualizzare questi termini usando dei diagrammi (precursori dei diagrammi di Feynman).
Se rappresentiamo $\langle \sigma_i \sigma_k \rangle$ con una linea che unisce $i$ e $k$, il termine a 4 punti è rappresentato da coppie di linee disgiunte (es. una linea $i-j$ e una linea $k-m$).
È fondamentale notare che le linee non si toccano, non c'è interazione o vertice dove si incrociano; sono semplicemente prodotti di propagatori indipendenti.
\begin{figure}[h!]
    \centering
    \includegraphics[width=0.7\textwidth]{pics/13.png}
    \caption{Rappresentazione Diagrammatica.}
\end{figure}

In questo modello, l'unico diagramma "connesso" è il propagatore stesso (la funzione a due punti). Tutte le funzioni di ordine superiore sono date da diagrammi "disconnessi".

\section{Funzioni di Correlazione Connesse}

Per trattare teorie interagenti (non gaussiane), è necessario definire le \textbf{funzioni di correlazione connesse} in modo ricorsivo, sottraendo le parti che si fattorizzano.
Definiamo:
\begin{align}
    \langle \sigma_i \rangle &\equiv \langle \sigma_i \rangle^c \\
    \langle \sigma_i \sigma_j \rangle &= \langle \sigma_i \sigma_j \rangle^c + \langle \sigma_i \rangle^c \langle \sigma_j \rangle^c \\
    \langle \sigma_i \sigma_j \sigma_k \rangle &= \langle \sigma_i \sigma_j \sigma_k \rangle^c + \langle \sigma_i \rangle^c \langle \sigma_j \sigma_k \rangle^c + 2 \text{ perm} + \langle \sigma_i \rangle^c \langle \sigma_j \rangle^c \langle \sigma_k \rangle^c
\end{align}
Da cui possiamo invertire per trovare la parte connessa a due punti:
\begin{equation}
    \langle \sigma_i \sigma_j \rangle^c = \langle \sigma_i \sigma_j \rangle - \langle \sigma_i \rangle^c \langle \sigma_j \rangle^c
\end{equation}

\subsubsection{Importanza delle Funzioni Connesse}
Le funzioni connesse sono cruciali per tre motivi:
\begin{enumerate}
    \item \textbf{Proprietà di Clustering:} Soddisfano la proprietà di decadimento a grandi distanze. Ad esempio, nella fase ordinata dove $\langle \sigma_i \sigma_j \rangle \to m^2$ per distanze grandi, la parte connessa $\langle \sigma_i \sigma_j \rangle^c \to 0$.
    \item \textbf{Diagrammi Connessi:} Negli approcci perturbativi, corrispondono esattamente alla somma dei soli diagrammi topologicamente connessi.
    \item \textbf{Densità di Energia Libera:} Sono legate alle derivate dell'energia libera rispetto al campo esterno.
    \begin{equation}
        \langle \prod_{a=1}^N \sigma_{i_a} \rangle^c = \beta^{-N} \prod_{a=1}^N \frac{\partial}{\partial h_{i_a}} \log Z[h] = -\beta^{-(N-1)} \prod_a \frac{\partial}{\partial h_{i_a}} F[h]
    \end{equation}
   
\end{enumerate}


\subsubsection{Riepilogo: Modello Gaussiano}
    \begin{itemize}
        \item È semplice, risolubile e possiede esponenti di campo medio.
        \item È esattamente risolubile per alte temperature, $\beta < \beta_c$.
        \item Ha una transizione a $\beta_c = 1/(2D)$.
        \item Il modello è mal definito per $\beta > \beta_c$ (bassa T), poiché la matrice di interazione  A non è più definita positiva.
        \item \textbf{Obiettivo:} Il GM è il nostro punto di partenza per sviluppare una teoria delle perturbazioni (come il modello di Landau-Ginzburg) per teorie più realistiche e \textit{interagenti}. Useremo il Gruppo di Rinormalizzazione (RG) per calcolare gli esponenti critici per queste teorie interagenti.
    \end{itemize}



\section{Espansione ad Alta Temperatura del GM}
\begin{tcolorbox}[colback=colorE!10, colframe=colorE!50!colorD, boxrule=1pt]
\textbf{Nota:} Il seguente calcolo non è parte del curriculum standard richiesto per l'esame, ma è riservato a chi aspira alla lode.
\end{tcolorbox}

\noindent Analizziamo ora l'espansione ad alta temperatura, un argomento utile per comprendere la genesi delle correlazioni e introdurre concetti diagrammatici.
A temperatura infinita ($\beta = 0$), il sistema è disordinato e non ci sono correlazioni: gli integrali si fattorizzano completamente.

Consideriamo la funzione di correlazione a due punti:
\begin{equation}
 \langle \sigma_a \sigma_b \rangle = \frac{\int (\prod d\sigma_i) e^{-\sum \sigma_i^2/2} e^{\beta \sum J_{ik} \sigma_i \sigma_k} \sigma_a \sigma_b}{\int (\prod d\sigma_i) e^{-\sum \sigma_i^2/2} e^{\beta \sum J_{ik} \sigma_i \sigma_k}}
\end{equation}
Per $\beta$ piccolo, espandiamo l'esponenziale dell'interazione (Termine di Hopping) in serie di potenze:
\begin{equation}
 e^{\beta \sum J_{ik} \sigma_i \sigma_k} = 1 + \frac{\beta}{2} \sum_{ ik } J_{ik} \sigma_i \sigma_k + \frac{\beta^2}{2} \left( \frac{1}{2}\sum_{ ik } J_{ik} \sigma_i \sigma_k \right)^2 + \dots
\end{equation}

\noindent Integriamo questa espansione termine per termine rispetto alla misura Gaussiana 
\begin{equation}
   d\mu = (\prod_i e^{-\sigma_i^2/2} d\sigma_i) 
\end{equation}

Vale la regola fondamentale:
\begin{tcolorbox}[colback=yellow!25, colframe=yellow!75!orange, coltitle=black, title=\textbf{Regola di Integrazione}]
L'integrale su $d\mu$ di qualsiasi termine contenente una potenza \textit{dispari} di qualsiasi $\sigma_i$ è zero. Un contributo non nullo si verifica solo quando tutte le variabili $\sigma$ nell'integrando sono "accoppiate" in potenze \textbf{pari}.
\end{tcolorbox}

\subsection{Analisi dei Termini}

\subsubsection{Il Denominatore (Funzione di Partizione $Z$)}
\begin{align*}
    Z &= \int d\mu \left[ 1 + \beta \sum_{\langle ik \rangle} J_{ik} \sigma_i \sigma_k + \frac{\beta^2}{2!} \sum_{\langle ik \rangle} \sum_{\langle i'k' \rangle} J_{ik} J_{i'k'} \sigma_i \sigma_k \sigma_{i'} \sigma_{k'} + \dots \right] \\
    Z &= Z_0 \left[ 1 + 0 + \frac{\beta^2}{2!} \sum_{\langle ik \rangle} \sum_{\langle i'k' \rangle} J_{ik} J_{i'k'} \langle \sigma_i \sigma_k \sigma_{i'} \sigma_{k'} \rangle_0 + \dots \right]
\end{align*}
dove $\langle \dots \rangle_0$ è l'integrale con la misura $H_0$ (cioè, il teorema di Wick).
\begin{enumerate}
    \item Il termine $O(\beta)$ è zero (potenze dispari).
    \item Il termine $O(\beta^2)$ è non nullo solo se le quattro variabili $\sigma$ sono accoppiate. Questo accade quando i due legami (link) $\{i,k\}$ e $\{i',k'\}$ sono gli stessi.
    Questo fornisce due possibilità:
    
        - $i=i'$ e $k=k'$
        
        - $i=k'$ e $k=i'$
        
    Questi due termini cancellano il fattore $1/2!$. Questo genera un diagramma a "doppia linea".
    \item A $O(\beta^4)$, appare un nuovo tipo di diagramma non nullo: un \textbf{loop} o \textbf{Plaquette}, dove quattro legami diversi formano un percorso chiuso (es. $J_{ij} J_{jk} J_{kl} J_{li}$).
\end{enumerate}
Quindi, il denominatore è $Z = Z_0 (1 + O(\beta^2))$.

\subsubsection{Il Numeratore}
Vogliamo trovare il contributo non nullo di ordine più basso per $\langle \sigma_a \sigma_b \rangle$.
\begin{align*}
    \text{Num.} &= \int d\mu \, \sigma_a \sigma_b \left[ 1 + \beta \sum_{\langle ik \rangle} J_{ik} \sigma_i \sigma_k + \dots \right] \\
    &= \int d\mu \, \sigma_a \sigma_b + \beta \sum_{\langle ik \rangle} J_{ik} \int d\mu \, \sigma_a \sigma_b \sigma_i \sigma_k + \dots
\end{align*}
\begin{enumerate}
    \item Il termine $O(\beta^0)$ $\int d\mu \, \sigma_a \sigma_b = \langle \sigma_a \sigma_b \rangle_0 = \delta_{ab}$. Questo è zero se $a \neq b$.
    \item Il termine $O(\beta^1)$ coinvolge $\langle \sigma_a \sigma_b \sigma_i \sigma_k \rangle_0$. Questo è non nullo se gli indici sono accoppiati. Gli accoppiamenti non nulli sono:
    
        - $a=i, b=k \implies \langle \sigma_i^2 \sigma_k^2 \rangle_0 = 1$
        
        - $a=k, b=i \implies \langle \sigma_k^2 \sigma_i^2 \rangle_0 = 1$
    
    L'integrale forza la somma $\sum_{\langle ik \rangle}$ a selezionare solo i termini dove $\{i,k\} = \{a,b\}$.
\end{enumerate}

    \begin{equation}
        \text{Num.} = \beta (J_{ab} \cdot \langle \sigma_a^2 \sigma_b^2 \rangle_0 + J_{ba} \cdot \langle \sigma_a^2 \sigma_b^2 \rangle_0) = \beta (J_{ab} + J_{ba})
    \end{equation}
    
\noindent  Se $a,b$ sono vicini (quindi $J_{ab} \neq 0$), questo termine è $\approx \beta J_{ab}$.

\noindent Il termine non nullo di ordine più basso per $\langle \sigma_a \sigma_b \rangle$ con $a \neq b$ è $O(\beta^1)$, corrispondente a un singolo legame $J_{ab}$. Per un percorso di 3 legami (es. $a \to 1 \to 2 \to b$), il contributo di ordine più basso sarebbe $O(\beta^3)$.



\section{Espansione ad Alta T nel Modello di Ising}

Passiamo ora al Modello di Ising, dove $\sigma_i = \pm 1$. Qui la matematica cambia a causa dell'\textbf{Algebra Booleana}:
\begin{equation}
 \sigma_i^2 = 1, \quad \sigma_i^3 = \sigma_i, \quad \dots
\end{equation}
Segue che qualsiasi polinomio in $\sigma$ è al massimo di grado 1 (lineare): 
\begin{equation}
    f(\sigma) = C_0 + C_1 \sigma
\end{equation}

\subsection{Character Expansion}
Sfruttando il fatto che $\sigma_i \sigma_j$ vale solo $\pm 1$, possiamo riscrivere l'esponenziale di Boltzmann in forma esatta (non è un'approssimazione per piccoli $\beta$, ma un'identità algebrica).
Il trucco centrale è riscrivere il fattore di Boltzmann per un singolo legame: $e^{\beta J \sigma_i \sigma_j}$ deve essere anch'esso un polinomio lineare della forma $A + B(\sigma_i \sigma_j)$.
Possiamo risolvere per $A$ e $B$ controllando i due possibili valori di $\sigma_i \sigma_j$:

\begin{itemize}
    \item \textbf{Caso 1:} $\sigma_i \sigma_j = +1 \implies A + B = e^{\beta J}$
    \item \textbf{Caso 2:} $\sigma_i \sigma_j = -1 \implies A - B = e^{-\beta J}$
\end{itemize}
Risolvendo il sistema otteniamo:
$A = \frac{e^{\beta J} + e^{-\beta J}}{2} = \cosh(\beta J)$
$B = \frac{e^{\beta J} - e^{-\beta J}}{2} = \sinh(\beta J)$

Questo fornisce l'identità esatta:
\begin{equation}
    e^{\beta J \sigma_i \sigma_j} = \cosh(\beta J) + \sinh(\beta J) \sigma_i \sigma_j
\end{equation}

\noindent Questa relazione è comunemente espressa come:
\begin{equation}
 e^{\beta \sigma_i \sigma_j} = c (1 + t \sigma_i \sigma_j)
 \label{f: char}
\end{equation}
dove definiamo i parametri fondamentali:
\begin{align*}
    c &= \cosh(\beta) \\
    s &= \sinh(\beta) \\
    t &= \tanh(\beta)
\end{align*}

La \ref{f: char} è un'\textbf{espansione dei caratteri}. Stiamo espandendo in $t = \tanh(\beta)$, che è piccolo ad alta T (piccolo $\beta$).

\subsection{Funzione di Partizione e Loop}

Ora scriviamo $Z$ usando questa identità per tutti i legami $\langle ij \rangle$ sul reticolo:
\begin{align}
    Z_{\text{Ising}} &= \sum_{\{\sigma = \pm 1\}} \prod_{\langle ij \rangle} e^{\beta J \sigma_i \sigma_j} \\
    &= \sum_{\{\sigma\}} \prod_{\langle ij \rangle} \left[ c (1 + t \sigma_i \sigma_j) \right] \\
    &= c^{N_L} \sum_{\{\sigma\}} \prod_{\langle ij \rangle} (1 + t \sigma_i \sigma_j) \label{eq:z_expand}
\end{align}
dove $N_L$ è il numero totale di legami sul reticolo.

Espandendo il prodotto nell'Eq. \ref{eq:z_expand}, otteniamo una somma di termini con potenze crescenti di $t$:
\begin{equation}
    \prod_{\langle ij \rangle} (1 + t \sigma_i \sigma_j) = 1 + \sum_{\langle ij \rangle} t (\sigma_i \sigma_j) + \sum_{\langle ij \rangle \neq \langle kl \rangle} t^2 (\sigma_i \sigma_j)(\sigma_k \sigma_l) + \dots
\end{equation}

\subsubsection*{Le Regole della Somma $\sum_{\{\sigma\}}$}

Per valutare quali termini sopravvivono, usiamo le proprietà della traccia sugli spin $\sigma_k~\in~\{-1, +1\}$:
\begin{itemize}
    \item $\sum_{\sigma_k} 1 = 2$;
    \item $\sum_{\sigma_k} \sigma_k = 0$ (poiché $1 + (-1) = 0$);
    \item $\sum_{\sigma_k} \sigma_k^n = 0$ se $n$ è dispari, mentre vale $2$ se $n$ è pari.
\end{itemize}

\noindent \textbf{Criterio di Sopravvivenza:} Un termine nell'espansione sopravvive alla somma su $\{\sigma\}$ solo se \textit{tutti} gli spin coinvolti in quel termine appaiono con una potenza pari (es. $\sigma_k^2, \sigma_k^4$). Se anche un solo spin appare con potenza dispari, la somma su quello spin azzera l'intero termine. \hfill \\

\noindent Analizziamo i contributi in base all'ordine di $t$ (numero di legami):

\begin{description}
    \item[Ordine $O(t^0)$:] Il termine costante \hfill \\
    Corrisponde al grafico vuoto (nessun legame). Nessuno spin è presente ($\sigma^0$). La somma su tutti gli $N$ spin fornisce semplicemente il fattore $2^N$.
    
    \item[Ordine $O(t^1)$:] Un singolo legame \hfill \\
    Il termine è $t \sigma_i \sigma_j$. Sia $\sigma_i$ che $\sigma_j$ appaiono una volta (potenza dispari). La somma è~$0$.
    
    \item[Ordine $O(t^2)$:] Due legami distinti \hfill \\
    Consideriamo $t^2 (\sigma_i \sigma_j)(\sigma_k \sigma_l)$. Tutti e quattro gli spin appaiono alla prima potenza. La somma è $0$.
    
    \item[Ordine $O(t^3)$:] Cammino aperto \hfill \\
    Consideriamo una catena di 3 legami: $t^3 (\sigma_1 \sigma_2)(\sigma_2 \sigma_3)(\sigma_3 \sigma_4)$. 
    Gli spin interni sono al quadrato ($\sigma_2^2, \sigma_3^2$), ma gli estremi $\sigma_1$ e $\sigma_4$ sono dispari. Il termine svanisce.
    
    \item[Ordine $O(t^4)$:] Il primo contributo non banale 
    Qui dobbiamo distinguere due casi geometrici:
    \begin{itemize}
        \item Cammini aperti: Svaniscono tutti perchè hanno estremi dispari.
        \item \textbf{Loop Chiuso (Placchetta):} Un quadrato formato dai siti 1, 2, 3, 4.
        Il termine è:
        \[
            t^4 (\sigma_1 \sigma_2)(\sigma_2 \sigma_3)(\sigma_3 \sigma_4)(\sigma_4 \sigma_1)
        \]
        Raggruppando gli spin, otteniamo:
        \[
            t^4 (\sigma_1^2) (\sigma_2^2) (\sigma_3^2) (\sigma_4^2) = t^4 (1)(1)(1)(1) = t^4
        \]
        Poiché non ci sono spin "spaiati", questo termine sopravvive. Sommando su tutte le configurazioni di spin otteniamo $t^4 \cdot 2^N$.
    \end{itemize}
\end{description}

\noindent In conclusione, la funzione di partizione è una somma su tutte le possibili configurazioni a \textbf{loop chiusi} (poligoni) sul reticolo:
\begin{equation}
    Z_{\text{Ising}} = c^{N_L} 2^N \left( 1 + N_{\text{sq}} t^4 + N_{\text{rect}} t^6 + \dots \right)
\end{equation}
dove $N_{\text{sq}}$ è il numero di quadrati elementari (placchette) e $N_{\text{rect}}$ è il numero di rettangoli $2\times1$.
