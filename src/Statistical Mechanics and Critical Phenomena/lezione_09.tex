\chapter{Lezione 9}
\label{chap:lezione_09} 

\begin{flushright}
\textit{Data: 22/10/2025}
\end{flushright}


\section{Risoluzione Esatta di un Modello con Forze Deboli a Lungo Raggio}

In questa sezione, risolveremo esattamente un modello in cui tutte le interazioni sono a lungo raggio, ma molto deboli. Dimostreremo che la soluzione esatta di questo modello coincide con le equazioni di campo medio (Mean Field, MF).

Definiamo l'Hamiltoniana del modello a lungo raggio (Long Range, LR) come:

\begin{equation}
    H_{LR}[\sigma] = -\frac{1}{2N} \sum_{i,k} \sigma_i \sigma_k - h \sum_i \sigma_i
\end{equation}

Dove:
\begin{itemize}
    \item La somma $\sum_{i,k}$ è estesa a tutte le coppie di spin $i, k$, con $i, k = 1, \dots, N$.
    \item $N$ è il numero totale di spin nel sistema, che coincide con il volume.
    \item $h$ è il campo magnetico esterno.
\end{itemize}


La prima somma include anche i termini con $i=k$. Il termine $\sigma_i \sigma_i = \sigma_i^2 = 1$ (poiché $\sigma_i = \pm 1$). La somma dei termini $i=k$ è quindi $\sum_i \sigma_i^2 = N$. Questo contribuisce all'energia totale con un termine $-\frac{1}{2N} \cdot N = -1/2$, che è una costante. Essendo una costante, è irrilevante per la fisica del sistema e possiamo trascurarla o includerla senza che cambi i risultati.

Il fattore $1/2$ nella formula serve a contare correttamente le coppie. La somma su $i,k$ conta due volte ogni coppia. Dividendo per 2, contiamo ogni coppia una sola volta.

\subsubsection{La necessità di un accoppiamento debole}

L'aspetto cruciale di questo modello è l'accoppiamento $J = 1/N$. Questa è un'interazione molto debole.

Consideriamo cosa succederebbe se $J$ fosse di ordine 1, $J = \mathcal{O}(1)$, nel limite termodinamico $N \to +\infty$.
L'energia è data dalla somma $\sum_{i,k} \sigma_i \sigma_k$, questa somma contiene $N^2$ termini.
La densità di energia (energia per spin) si ottiene normalizzando per $N$:
$$ \frac{E}{N} \propto \frac{1}{N} \sum_{i,k} \sigma_i \sigma_k \propto \frac{N^2}{N} \propto N $$
Se $J = \mathcal{O}(1)$, la densità di energia divergerebbe linearmente con $N$ quando $N \to \infty$. Il modello non sarebbe ben definito termodinamicamente, poiché la densità di energia non sarebbe una quantità intensiva (cioè di ordine 1).

Per avere un modello ben definito nel limite termodinamico, abbiamo bisogno che la densità di energia sia intensiva. Per ottenere ciò, l'accoppiamento $J$ deve scalare come $1/N$.
In questo modo, la densità di energia diventa:
$$ \frac{E}{N} \propto \frac{1}{N} \left( \frac{1}{N} \sum_{i,k} \sigma_i \sigma_k \right) \propto \frac{1}{N^2} \cdot N^2 \propto 1 $$
Con $J = 1/N$, la densità di energia è di ordine 1 e il modello è ben definito. Questo descrive un sistema in cui ogni spin interagisce con tutti gli altri, ma con un'interazione che diventa infinitamente debole al crescere del sistema.

\subsection{Trasformazione di Hubbard-Stratonovich}

Per risolvere il modello, useremo un'identità integrale nota (una versione della trasformazione di Hubbard-Stratonovich), basata sull'integrale Gaussiano:
\begin{tcolorbox}[colback=red!15,  
                  colframe=red!50!orange,  
                  boxrule=0.8pt, 
                  arc=3pt,  
                  top=4pt, bottom=4pt, left=6pt, right=6pt,
                  enhanced,
                  sharp corners=south]
\begin{equation}
    \int_{-\infty}^{+\infty} d x \, e^{-Ax^2 \pm Bx} = \sqrt{\frac{\pi}{A}} e^{\frac{B^2}{4A}}
\end{equation}
\end{tcolorbox}


Riscriviamo questa identità per isolare il termine esponenziale:
$$ e^{\frac{B^2}{4A}} = \sqrt{\frac{A}{\pi}} \int_{-\infty}^{+\infty} d x \, e^{-Ax^2 + Bx} $$

Applichiamo questo "trucco" al fattore di Boltzmann $e^{-\beta H}$. L'Hamiltoniana è 
\begin{equation}
   H = -\frac{1}{2N} \left (\sum_i \sigma_i \right)^2 - h \sum_i \sigma_i 
\end{equation}
dove abbiamo usato l'identità $(\sum_i \sigma_i)^2 = \sum_{i,k} \sigma_i \sigma_k$.

Il fattore di Boltzmann contiene il termine $e^{\frac{\beta}{2N} (\sum_i \sigma_i)^2}$.
Vogliamo usare l'identità per linearizzare questo termine quadratico. Identifichiamo i termini:
\begin{itemize}
    \item $\frac{B^2}{4A} = \frac{\beta}{2N} (\sum_i \sigma_i)^2$

    \item $B = \beta \sum_i \sigma_i$
    \item $A = \frac{N \beta}{2}$
\end{itemize}
Sostituendo questi valori nell'identità, otteniamo:
$$ e^{\frac{\beta}{2N} (\sum_i \sigma_i)^2} = \sqrt{\frac{N\beta}{2\pi}} \int_{-\infty}^{+\infty} d \lambda \, e^{-\frac{N\beta\lambda^2}{2} + \beta\lambda \sum_i \sigma_i} $$


Ora sostituiamo questa espressione nel fattore di Boltzmann completo $e^{-\beta H}$:
$$ e^{-\beta H} = e^{\frac{\beta}{2N} (\sum_i \sigma_i)^2 + \beta h \sum_i \sigma_i} = e^{\frac{\beta}{2N} (\sum_i \sigma_i)^2} \cdot e^{\beta h \sum_i \sigma_i} $$

\begin{equation}
    e^{-\beta H} = \sqrt{\frac{N\beta}{2\pi}} \int_{-\infty}^{+\infty} d \lambda \, e^{-\frac{N\beta\lambda^2}{2} + \beta\lambda \sum_i \sigma_i} \cdot e^{\beta h \sum_i \sigma_i}
\end{equation}

Combinando i termini nell'esponenziale dentro l'integrale:

\begin{equation}
    e^{-\beta H} = \sqrt{\frac{N\beta}{2\pi}} \int_{-\infty}^{+\infty} d \lambda \, e^{-\frac{N\beta\lambda^2}{2} + \beta(\lambda+h) \sum_i \sigma_i}
\end{equation}

Questa espressione sembra molto più complicata di quella da cui siamo partiti. Tuttavia, ha un vantaggio cruciale: l'esponente è ora \textbf{lineare} nelle variabili di spin $\sigma_i$. Il termine quadratico $(\sum_i \sigma_i)^2$ è sparito, a costo di introdurre una nuova variabile di integrazione $\lambda$.

Il termine $\sum_i \beta(\lambda+h)\sigma_i$ è \textbf{fattorizzato}: può essere scritto come un prodotto di termini che dipendono da un singolo spin, $\prod_i e^{\beta(\lambda+h)\sigma_i}$. Questo ci permetterà di calcolare la somma sugli stati degli spin.

\subsection{Calcolo della Funzione di Partizione Z}

La funzione di partizione è data dalla somma su tutte le configurazioni di spin $\{\sigma_i = \pm 1\}$:

\begin{equation}
    Z = \sum_{\{\sigma_i = \pm 1\}} e^{-\beta H[\sigma]}
\end{equation}

Sostituiamo l'espressione integrale per $e^{-\beta H}$:

\begin{equation}
    Z = \sum_{\{\sigma_i\}} \left[ \sqrt{\frac{N\beta}{2\pi}} \int_{-\infty}^{+\infty} d \lambda \, e^{-\frac{N\beta\lambda^2}{2}} \cdot e^{\beta(\lambda+h) \sum_i \sigma_i} \right]
\end{equation}

Poiché l'integrale in $d\lambda$ non dipende dalle $\sigma_i$ (e viceversa), possiamo scambiare l'ordine di somma e integrazione:

\begin{equation}
    Z = \sqrt{\frac{N\beta}{2\pi}} \int_{-\infty}^{+\infty} d \lambda \, e^{-\frac{N\beta\lambda^2}{2}} \left[ \sum_{\{\sigma_i = \pm 1\}} e^{\beta(\lambda+h) \sum_i \sigma_i} \right]
\end{equation}

Concentriamoci sulla somma tra parentesi quadre. Grazie alla fattorizzazione, possiamo scriverla come:
$$ \sum_{\{\sigma_i\}} \prod_{i=1}^N e^{\beta(\lambda+h) \sigma_i} = \prod_{i=1}^N \left[ \sum_{\sigma_i = \pm 1} e^{\beta(\lambda+h) \sigma_i} \right] $$

Calcoliamo la somma per un singolo spin $i$:
$$ \sum_{\sigma_i = \pm 1} e^{\beta(\lambda+h) \sigma_i} = e^{\beta(\lambda+h)(+1)} + e^{\beta(\lambda+h)(-1)} = 2 \cosh(\beta(\lambda+h)) $$
Questo risultato è identico per tutti gli $N$ spin. Il prodotto totale è quindi:

\begin{equation}
    \left[ \sum_{\sigma_i = \pm 1} e^{\beta(\lambda+h) \sigma_i} \right]^N = [2 \cosh(\beta(\lambda+h))]^N
\end{equation}

Sostituiamo questo risultato nell'espressione per $Z$:

\begin{equation}
    Z = \sqrt{\frac{N\beta}{2\pi}} \int_{-\infty}^{+\infty} d \lambda \, e^{-\frac{N\beta\lambda^2}{2}} \cdot [2 \cosh(\beta(\lambda+h))]^N
\end{equation}

Raggruppiamo tutti i termini che dipendono da $N$ nell'esponenziale:
$$ (2 \cosh(\beta(\lambda+h)))^N = e^{N \log(2 \cosh(\beta(\lambda+h)))} $$

\begin{equation}
    Z = \sqrt{\frac{N\beta}{2\pi}} \int_{-\infty}^{+\infty} d \lambda \, e^{-\frac{N\beta\lambda^2}{2} + N \log(2 \cosh(\beta(\lambda+h)))}
\end{equation}

Mettiamo in evidenza $-N\beta$ nell'esponente:

\begin{equation}
    Z = \sqrt{\frac{N\beta}{2\pi}} \int_{-\infty}^{+\infty} d \lambda \, e^{-N\beta \left[ \frac{\lambda^2}{2} - \frac{1}{\beta} \log(2 \cosh(\beta(\lambda+h))) \right]}
\end{equation}

\subsection{Metodo del Punto di Sella}

Abbiamo ridotto la funzione di partizione a un integrale della forma:
$$ \int d \lambda \, e^{-N\beta A(\lambda)} $$
dove abbiamo definito una "azione efficace" $A(\lambda)$:

\begin{equation}
    A(\lambda) = \frac{\lambda^2}{2} - \frac{1}{\beta} \log(2 \cosh(\beta(\lambda+h)))
\end{equation}

Siamo interessati al limite termodinamico $N \to \infty$. Il fattore $N$ che moltiplica l'esponente è molto grande. Questo ci permette di usare il \textbf{metodo del punto di sella}.

L'idea è che, per $N$ molto grande, l'integrale sarà completamente dominato dal valore $\lambda^*$ che \textbf{minimizza} la funzione $A(\lambda)$.
Qualsiasi valore $\lambda \neq \lambda^*$ darà $A(\lambda) > A(\lambda^*)$, e il suo contributo all'integrale sarà $e^{-N\beta A(\lambda)}$, che è esponenzialmente soppresso (tende a zero molto rapidamente) rispetto al contributo del minimo $e^{-N\beta A(\lambda^*)}$.

L'energia libera di Helmoltz per spin, $f$, è data da $f = -\frac{1}{\beta N} \log Z$.

Nel limite $N \to \infty$, possiamo approssimare $Z$ con il valore dell'integrando calcolato al punto di sella (minimo di $A(\lambda)$), trascurando il pre-fattore $\sqrt{N}$ e i contributi delle fluttuazioni quadratiche (che danno termini di ordine $\mathcal{O}(\log N)/N$ o $\mathcal{O}(1/N)$):

\begin{equation}
    f = \min_{\lambda} A(\lambda) + \mathcal{O} (1/N)
\end{equation}

\noindent Per trovare il minimo di $A(\lambda)$, dobbiamo calcolare la derivata prima e porla uguale a zero:

\begin{align}
    \frac{d}{d \lambda} & \left[ \frac{\lambda^2}{2} - \frac{1}{\beta} \log(2 \cosh(\beta(\lambda+h))) \right] = 0 \\
 \lambda &- \frac{1}{\beta} \cdot \frac{1}{2 \cosh(\beta(\lambda+h))} \cdot (2 \sinh(\beta(\lambda+h))) \cdot \beta = 0 \\
 \lambda& - \frac{\sinh(\beta(\lambda+h))}{\cosh(\beta(\lambda+h))} = 0
\end{align}  

Otteniamo quindi l'equazione:
\begin{tcolorbox}[colback=yellow!30,  
                  colframe=yellow!50!orange,  
                  boxrule=0.8pt, 
                  arc=3pt,  
                  top=4pt, bottom=4pt, left=6pt, right=6pt,
                  enhanced,
                  sharp corners=south]
\begin{equation}
    \lambda = \tanh(\beta(\lambda+h))
\end{equation}
\end{tcolorbox}


Questa è esattamente l'equazione di campo medio (M.F.E.) che abbiamo derivato in precedenza, a condizione di identificare la variabile di integrazione $\lambda$ con la magnetizzazione $m$:

\begin{equation}
    \lambda = m
\end{equation}

\paragraph{Conclusione:}
Abbiamo dimostrato che l'approssimazione di campo medio (M.F.A.) non è solo un'approssimazione, ma è la soluzione \textbf{esatta} per il modello con interazioni a raggio infinito ($J \propto 1/N$).

La ragione fisica profonda di questo risultato è che, in questo modello, le correlazioni spin-spin connesse sono soppresse. Si può dimostrare che:
\begin{equation}
    \langle \sigma_i \sigma_j \rangle^c = \langle \sigma_i \sigma_j \rangle - \langle \sigma_i \rangle \langle \sigma_j \rangle = \mathcal{O}\left (\frac{1}{N} \right)
\end{equation}
Poiché le interazioni sono deboli e "diluite" su tutto il sistema, ogni spin sente solo l'effetto medio (il campo medio) di tutti gli altri, e le correlazioni (fluttuazioni) tra coppie di spin diventano trascurabili nel limite termodinamico. Questo è lo stesso motivo per cui la M.F.A. diventa esatta in modelli a corto raggio nel limite di dimensionalità $D \to \infty$.

\section{Validità della M.F.A. e Funzioni di Correlazione}

Ora vogliamo capire quando la M.F.A. è un'approssimazione "sensata" per modelli più realistici (ad esempio, a corto raggio). Per farlo, studieremo le \textbf{funzioni di correlazione}.

\subsubsection{Il paradosso della stima delle correlazioni}

Sorge un problema concettuale: l'approssimazione di campo medio si basa sull'assunzione che gli spin non siano correlati, cioè che $\langle \sigma_i \sigma_j \rangle^c = 0$ per $i \neq j$.
Come possiamo allora usare un'approssimazione che per definizione impone correlazioni nulle per \emph{stimare} le correlazioni del modello originale?

La risposta è sottile. Dobbiamo distinguere tra due quantità:
\begin{enumerate}
    \item La funzione di correlazione calcolata \emph{all'interno} dell'approssimazione MF: $\langle \sigma_i \sigma_j \rangle_c^{MF}$. Questa è zero per definizione.
    \item Una stima della funzione di correlazione del modello \emph{originale}, ottenuta usando la soluzione MF.
\end{enumerate}

\subsubsection{Fluctuation-Dissipation Theorem (FDT)}

Nel modello esatto (teoria "True"), esiste una relazione fondamentale (il Teorema di Fluttuazione-Dissipazione, FDT) che lega le correlazioni alla risposta del sistema a una perturbazione (risposta lineare).

La funzione di correlazione connessa è la derivata seconda dell'energia libera:
\begin{equation}
    \langle \sigma_i \sigma_k \rangle_c^{(True)} = -\frac{1}{\beta} \frac{\partial}{\partial h_k} \left( \frac{\partial F}{\partial h_i} \right) = \frac{1}{\beta} \frac{\partial \langle \sigma_i \rangle}{\partial h_k} = \frac{1}{\beta} \frac{\partial m_i}{\partial h_k}
\end{equation}

Questa identità \textbf{non è vera all'interno dell'approssimazione di campo medio}. 

In M.F.A. si ha: $\langle \sigma_i \sigma_k \rangle_c^{MF} = 0$ (lato sinistro) , $\frac{1}{\beta} \frac{\partial m_i}{\partial h_k}\Big|_{MF} \neq 0$ (lato destro)



\subsubsection{Strategia di Stima}

La nostra strategia sarà quella di usare il termine di risposta lineare calcolato in M.F.A. come nostra \textbf{stima} della vera funzione di correlazione:

\begin{equation}
    \langle \sigma_i \sigma_k \rangle_c^{(True)} \approx \frac{\partial m_i}{\partial h_k}\Big|_{MF} \cdot \frac{1}{\beta}
\end{equation}

Questa è una stima migliore (e non banale) rispetto a $\langle \sigma_i \sigma_k \rangle_c^{MF} = 0$.

\subsubsection{Giustificazione Variazionale}

Perché questa è una stima migliore? Consideriamo un argomento variazionale.

Sia $P_{eq}^T$ la vera distribuzione di probabilità di equilibrio e $P_{eq}^A$ la distribuzione approssimata (es. quella fattorizzata di campo medio).
Definiamo la "distanza" tra le due $\delta P = |P_{eq}^T - P_{eq}^A|$. Assumiamo che l'approssimazione sia buona, quindi $\delta P = \mathcal{O}(\epsilon)$ con $\epsilon$ piccolo.

\begin{enumerate}
    \item Stima di $\langle \sigma_i \sigma_j \rangle_c$: Questa è un'aspettativa lineare sulla distribuzione $P$.
    $$ \langle \sigma_i \sigma_j \rangle_c^T = \langle \sigma_i \sigma_j \rangle_c^A + \mathcal{O}(\epsilon) $$
    L'errore sulla correlazione calcolata direttamente è $\mathcal{O}(\epsilon)$. 
    
    Nel nostro caso $\langle \sigma_i \sigma_j \rangle_c^A = 0$, quindi $\langle \sigma_i \sigma_j \rangle_c^T = \mathcal{O}(\epsilon)$.

    \item Stima dell'Energia Libera $F$:  La vera distribuzione $P_{eq}^T$ è il minimo di $F[P]$:
    \begin{equation}
        \frac{\delta F[P]}{\delta P} \Big|_{P = P_{eq}^T} = 0
    \end{equation}
    A causa di questa proprietà variazionale, quando calcoliamo $F$ usando la distribuzione approssimata $P_{eq}^A$ (che è "vicina" a $P_{eq}^T$), l'errore che commettiamo non è $\mathcal{O}(\epsilon)$, ma $\mathcal{O}(\epsilon^2)$:
    \begin{equation}
        F[P_{eq}^T] = F[P_{eq}^A] + \mathcal{O}(\epsilon^2)
    \end{equation}
    I termini lineari in $\epsilon$ si cancellano proprio perché siamo in un punto stazionario.
\end{enumerate}


La quantità fisica che ci interessa stimare è la funzione di correlazione del sistema reale $\langle \sigma_i \sigma_k \rangle_c^{(True)}$. Questa, nella teoria esatta, è legata alla derivata seconda dell'energia libera reale $F[P_{eq}^T]$.
La nostra stima (la risposta lineare calcolata in campo medio $\frac{\partial m_i}{\partial h_k}|_{MF}$) è invece legata alla derivata seconda dell'energia libera \emph{approssimata} $F[P_{eq}^A]$.

Il punto cruciale è che, grazie al principio variazionale, $F[P_{eq}^A]$ è una stima molto più accurata di $F[P_{eq}^T]$ (con un errore $\mathcal{O}(\epsilon^2)$) rispetto a quanto lo sia la stima diretta della correlazione $\langle \sigma_i \sigma_k \rangle_c^A$ (che ha un errore $\mathcal{O}(\epsilon)$).

Poiché $F[P_{eq}^A]$ è una stima così precisa dell'energia libera reale, ci aspettiamo che anche le sue derivate seconde (ovvero la nostra stima $\frac{\partial m_i}{\partial h_k}|_{MF}$) siano una stima della correlazione reale \emph{molto migliore} rispetto alla stima diretta (che è 0 e ha un errore maggiore).

Per questa ragione, procederemo calcolando $\frac{\partial m_i}{\partial h_k}|_{MF}$ e analizzeremo il suo comportamento per testare i limiti di validità dell'approssimazione di campo medio.