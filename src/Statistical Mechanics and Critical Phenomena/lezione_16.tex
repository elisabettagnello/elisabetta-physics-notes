\chapter{Lezione 16}
\label{chap:lezione_16}

\begin{flushright}
\textit{Data: 17/11/2025}
\end{flushright}

\section{Espansione ad Alta T e Variabili di Link}

Riprendiamo l'analisi del modello di Ising nell'espansione ad alta temperatura. Abbiamo visto che la funzione di partizione può essere scritta esattamente utilizzando la variabile duale $t = \tanh (\beta)$:

\begin{equation}
Z_{Ising} = \sum_{\{\sigma=\pm 1\}} e^{\beta \sum_{\langle i,j \rangle} \sigma_i \sigma_j}
\end{equation}

Utilizzando l'identità $e^{\beta \sigma_i \sigma_j} = \cosh (\beta) (1 + t \sigma_i \sigma_j)$, possiamo fattorizzare il termine $\cosh( \beta)$ per ogni legame ($N_L$ è il numero totale di legami):

\begin{equation}
Z_{Ising} = (\cosh( \beta))^{N_L} \sum_{\{\sigma=\pm 1\}} \prod_{\langle i,j \rangle} (1 + t \sigma_i \sigma_j)
\end{equation}

Quando espandiamo il prodotto, otteniamo una somma di termini. Poiché dobbiamo sommare sulle configurazioni di spin $\sigma = \pm 1$, sopravvivono solo i termini in cui ogni variabile di spin appare con una potenza pari. I termini con potenze dispari si annullano nella somma.
Questo ci porta a una rappresentazione grafica in termini di cammini chiusi.

Possiamo formalizzare questo concetto introducendo delle variabili di occupazione dei link, $n_{ik} \in \{0, 1\}$, per ogni coppia di vicini $\langle i,k \rangle$:
\begin{itemize}
    \item $n_{ik}=1$: il link è presente nel grafico (contributo $t$).
    \item $n_{ik}=0$: il link è assente (contributo $1$).
\end{itemize}

La funzione di partizione diventa una somma su queste variabili di link:

\begin{equation}
Z_{Ising} = (\cosh( \beta))^{N_L} \sum_{\{n_{ik}=0,1\}} t^{\sum_{\langle i,k \rangle} n_{ik}} \delta_{\text{constraint}}
\end{equation}

Il vincolo ($\delta_{\text{constraint}}$) impone che in ogni sito $i$, la somma delle variabili di link incidenti $n_i$ sia un numero pari:

\begin{equation}
n_i = \sum_{k} n_{ik} \quad \text{deve essere pari}
\end{equation}

Questo assicura che i grafici contribuenti siano formati da linee chiuse (loop).

\subsection{Contributi all'Energia Libera}

Il primo contributo non banale (diverso da 1) proviene dal loop chiuso più piccolo possibile. Su un reticolo ipercubico, non è possibile formare triangoli ($t^3$), quindi il primo termine\footnote{Nel caso monodimensionale ($D=1$), sebbene non esistano loop locali, se si assumono condizioni al contorno periodiche (spazio non piatto), esiste un unico cammino chiuso che avvolge l'intero volume $V$. Questo genera un contributo di ordine $t^V$, che tuttavia diventa trascurabile nel limite termodinamico.} è il quadrato (plaquette) di ordine $t^4$:

\begin{equation}
t^4 (\sigma_1 \sigma_2)(\sigma_2 \sigma_3)(\sigma_3 \sigma_4)(\sigma_4 \sigma_1) \to t^4 \sigma_1^2 \sigma_2^2 \sigma_3^2 \sigma_4^2 = t^4
\end{equation}

Calcoliamo l'energia libera per sito $f(\beta)$. Ricordiamo che $f = -\frac{1}{\beta N} \log Z$.
Separando il contributo del termine $(\cosh( \beta))^{N_L}$ (dove $N_L = DN$) e il contributo dei loop:

\begin{equation}
f(\beta) = -\frac{1}{\beta N} \log Z = -\frac{D}{\beta} \log(\cosh( \beta)) + \tilde{f}(\beta)
\end{equation}

Per valutare $\tilde{f}(\beta)$ all'ordine più basso, contiamo i modi per formare una placchetta quadrata di 4 passi ($t^4$) partendo da un sito dato e tornando all'origine:
\begin{enumerate}
    \item Scelta della prima direzione: abbiamo $D$ direzioni possibili.
    \item Scelta della seconda direzione: abbiamo $D-1$ direzioni (non possiamo tornare subito indietro).
    \item \textbf{Fattore di simmetria:} Dobbiamo dividere per 2. Questo fattore $1/2$ tiene conto del fatto che il cammino può essere percorso in due sensi (orario e antiorario), che rappresentano lo stesso oggetto geometrico nella somma sulla funzione di partizione.
\end{enumerate}
Il fattore combinatorio corretto risulta quindi $\frac{D(D-1)}{2}$. L'energia libera corretta all'ordine $t^4$ risulta:

\begin{equation}
f(\beta) \approx -\frac{1}{\beta} \left( D \log(\cosh( \beta)) + \frac{D(D-1)}{2} t^4 \right)
\end{equation}

Questo calcolo mostra come le fluttuazioni (i loop) modifichino l'energia libera rispetto alla semplice approssimazione di campo medio.


\noindent Per studiare il comportamento critico con precisione, si espande la funzione in serie di potenze di $t^2$ ad ordini molto elevati:

\begin{equation}
f(t^2) = \sum_{n=0}^{\infty} f_n t^2
\end{equation}

Assumiamo che la singolarità più vicina all'origine nel piano complesso $t^2$ sia la temperatura critica $t_c^2$, e che la funzione abbia una singolarità del tipo:

\begin{equation}
(t^2 - t_c^2)^\omega
\end{equation}

 
Utilizzando teoremi standard sulle serie di potenze, possiamo determinare la posizione della singolarità $t_c$ dal rapporto dei coefficienti:

\begin{equation}
t_c^2 = \lim_{n \to \infty} \frac{f_n}{f_{n+1}}
\end{equation}
 
Inoltre, possiamo stimare l'esponente critico $\omega$ analizzando la curvatura dei rapporti:

\begin{equation}
\omega = \lim_{n \to \infty} \left( -n^2 \left( \frac{f_n}{f_{n+1}} - \frac{f_{n-1}}{f_n} \right) \right) t_c^{-2} - 1
\end{equation}

Questa tecnica richiede il calcolo di un numero finito ma elevato di termini.
Per esempio, come citato nel libro di Parisi ("Statistical Field Theory"), per il reticolo cubico semplice ($D=3$) si sono calcolati termini fino al 18° ordine ($t^{18}$).


\section{Dal Reticolo alla Teoria di Campo (Continuo)}

Esaminiamo ora il passaggio dal reticolo discreto alla teoria di campo continuo. Questo passaggio è fisicamente giustificato solo vicino al punto critico, dove la lunghezza di correlazione $\xi$ diverge ($\xi \gg a$, con $a$ passo del reticolo).

Possiamo visualizzare questo limite immaginando la struttura del sistema in termini di domini di spin. Immaginiamo di avere delle "gocce" (droplets) di spin positivi in un mare di spin negativi, e viceversa.
\begin{itemize}
    \item Se la lunghezza di correlazione è piccola (ordine 1), avremo piccole gocce isolate.
    \item Avvicinandoci al punto critico, la dimensione tipica di queste gocce cresce.
    \item Nel limite del continuo, stiamo essenzialmente "sfocando" (smoothing) la visione microscopica: non guardiamo più il singolo spin che scatta su e giù, ma la magnetizzazione media locale su scale di lunghezza comparabili a $\xi$.
\end{itemize}
In questo limite, \textbf{Le variabili di spin discrete $\sigma_i$ possono essere sostituite da un campo continuo $\varphi(x)$ che varia lentamente nello spazio ("smooth"):}

\begin{equation}
\sigma_i \to \varphi(x)
\end{equation}

Analizziamo il propagatore (funzione di correlazione a due punti) nello spazio dei momenti (trasformata di Fourier):

\begin{equation}
\langle \sigma_j \sigma_l \rangle \sim \int d^D k \frac{e^{-i(j-l)\cdot k}}{1 - 2\beta J \sum_{\mu} \cos k_\mu}
\end{equation}

Espandiamo il denominatore per piccoli momenti $k$ (che corrispondono a grandi distanze):
Ricordando che $\cos k_\mu \approx 1 - \frac{k_\mu^2}{2}$, il denominatore diventa:

\begin{equation}
1 - 2\beta J D + \beta J k^2 + O(k^4)
\end{equation}

dove $k^2 = \sum k_\mu^2$.

Questa espressione ha la struttura tipica di un \textbf{propagatore massivo}:
\begin{tcolorbox}[colback=yellow!25, colframe=yellow!75!orange, coltitle=black, title=\textbf{Propagatore}]
\begin{equation}
 \sim \frac{1}{m^2 + k^2}
\end{equation}
\end{tcolorbox}


Qui identifichiamo un \textbf{termine di "massa"} $m^2$:
\begin{equation}
m^2 \propto \frac{1 - 2D\beta J}{\beta J} \propto (T - T_c)
\end{equation}
Si vede chiaramente che la massa si annulla alla temperatura critica di campo medio: $2D\beta_c J = 1$

Esattamente a $T=T_c$, il propagatore si comporta come: 

\begin{equation}
    \frac{1}{k^2}
\end{equation}


\subsection{Derivazione dell'Azione nel Continuo}

Costruiamo l'azione (o Hamiltoniana efficace) nel continuo espandendo i termini reticolari.
Il termine di somma sugli spin diventa un integrale spaziale:
\begin{equation}
\sum_i \varphi_i^2 \to \int d^D x \, \varphi^2(x)
\end{equation}

Il termine di interazione tra primi vicini $\sum_{\langle i,k \rangle} \sigma_i J_{ik} \sigma_k$ richiede un'attenzione particolare. Vogliamo capire come questo termine generi le derivate spaziali nel continuo.

Raggruppiamo i termini dei vicini di un sito in posizione $\vec{0}$ sottraendo il campo centrale $\varphi_{0,0}$:
\begin{equation}
(\varphi_{0,1} - \varphi_{0,0}) + (\varphi_{0,-1} - \varphi_{0,0}) + (\varphi_{1,0} - \varphi_{0,0}) + (\varphi_{-1,0} - \varphi_{0,0})
\end{equation}

Eseguiamo l'espansione di Taylor per i vicini lungo l'asse $y$ (siti $0,1$ e $0,-1$):
\begin{align}
\varphi_{0,1} &= \varphi_{0,0} + \frac{\partial \varphi}{\partial y} + \frac{1}{2} \frac{\partial^2 \varphi}{\partial y^2} + \dots \\
\varphi_{0,-1} &= \varphi_{0,0} - \frac{\partial \varphi}{\partial y} + \frac{1}{2} \frac{\partial^2 \varphi}{\partial y^2} + \dots
\end{align}

Sommando questi contributi, i termini di derivata prima si cancellano e rimangono le derivate seconde:
\begin{equation}
(\varphi_{0,1} - \varphi_{0,0}) + (\varphi_{0,-1} - \varphi_{0,0}) \approx 2 \left( \frac{1}{2} \frac{\partial^2 \varphi}{\partial y^2} \right) = \frac{\partial^2 \varphi}{\partial y^2}
\end{equation}

Sommando su tutte le $D$ direzioni spaziali, ricostruiamo l'operatore Laplaciano $\nabla^2$.

Sostituendo questo risultato nell'espressione della funzione di partizione, l'integrale funzionale assume la forma esatta:

L'azione gaussiana nel continuo assume quindi la forma:

\begin{equation}
S_{Gauss}[\varphi] =  \int d^D x \left( -(1-2D\beta) \frac{\varphi(x)^2}{2} + \frac{\beta}{2} \varphi(x) \nabla^2 \varphi(x) \right) 
\end{equation}


dove il termine $\varphi \nabla^2 \varphi$ rappresenta il termine cinetico che, integrando per parti, corrisponde al classico $-(\nabla \varphi)^2$


\section{Il Modello $\varphi^4$ (Ginzburg-Landau)}

Il modello gaussiano derivato sopra è un'approssimazione valida solo se le fluttuazioni sono puramente quadratiche (\textbf{teoria libera}). Questa teoria è "particolarmente noiosa" dal punto di vista della fisica critica. Essendo un'approssimazione quadratica, essa riproduce esattamente gli esponenti critici della teoria di Campo Medio, che sappiamo essere errati per dimensioni $D~<~4$. Sappiamo, infatti, che il modello di Ising ha una transizione di fase non banale che il modello gaussiano non cattura completamente in bassa dimensione ($D<4$).
Per descrivere una teoria fisica realistica, dobbiamo modificare l'azione aggiungendo un termine di interazione che rispetti la simmetria $\varphi \to -\varphi$. Aggiungiamo un termine quartico.  L'azione completa diventa:

\begin{equation}
S[\varphi] = \int d^D x \left( \frac{1}{2} (\nabla \varphi)^2 + \frac{1}{2} m^2 \varphi^2 + g \varphi^4 \right)
\end{equation}

Questo è il \textbf{modello di Ginzburg-Landau}. 

Il problema principale è che non sappiamo risolvere esattamente questa teoria interagente.
La nostra strategia sarà quella di trattare $g$ come un parametro piccolo ed eseguire un'\textbf{espansione perturbativa} in $g$.
\begin{itemize}
    \item Se $g=0$, torniamo alla teoria gaussiana risolubile.
    \item Espandendo per piccoli $g$, calcoliamo le correzioni agli esponenti di campo medio.
\end{itemize}
Questo approccio perturbativo sarà la base per l'applicazione del Gruppo di Rinormalizzazione.




