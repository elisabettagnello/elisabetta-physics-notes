\chapter{Lezione 7}
\label{chap:lezione_07} 

\begin{flushright}
\textit{Data: 15/10/2025}
\end{flushright}


\section{Fenomeni Critici e Lunghezza di Correlazione}

Quando la temperatura $T$ si avvicina alla temperatura critica $T_c$ da sopra o da sotto $T \to T_c^\pm$, la lunghezza di correlazione $\xi$ diverge $\xi \to +\infty$.

La lunghezza di correlazione è definita tramite il decadimento esponenziale della funzione di correlazione connessa $\langle\sigma_{i}\sigma_{J}\rangle$ per grandi distanze $|x_{i}-x_{J}| \to +\infty$:
\begin{equation}
   \langle\sigma_{i}\sigma_{J}\rangle_{c}\sim e^{-\frac{|x_{i}-x_{J}|}{\xi}}  
\end{equation}

All'avvicinarsi della temperatura critica, $\xi$ cresce come una potenza della differenza di temperatura:
\begin{equation}
\xi \sim \frac{1}{|T-T_c|^{\nu}} \quad \text{o} \quad \xi \sim |T-T_{c}|^{-\nu}
\end{equation}
Esattamente alla temperatura critica $T=T_c$, il decadimento non è più esponenziale ma a \textbf{legge di potenza}:
\begin{equation}
\langle\sigma_{i}\sigma_{J}\rangle \approx \frac{1}{|x_{i}-x_{J}|^{\eta}}
\end{equation}
L'esponente $\nu$ è un \textbf{esponente critico}. È cruciale perché indica il tipo di singolarità fisica che si verifica al punto critico. 

La fisica degli esponenti critici è universale. La temperatura critica invece non è universale,  $T_c$ dipende, ad esempio, dal tipo di reticolo (cubico semplice, cubico a facce centrate, esagonale, ecc.).

\newpage
\section{Equazioni del Campo Medio }

Le equazioni del campo medio possono essere derivate minimizzando l'energia libera di un sistema. L'energia libera $F$ ha un termine di energia interna e un termine di entropia.

L'entropia per una distribuzione di probabilità fattorizzata è data da:
\begin{equation}
S = -\sum_{i} \left\{ \frac{1+m_{i}}{2} \log\left(\frac{1+m_{i}}{2}\right) + \frac{1-m_{i}}{2} \log\left(\frac{1-m_{i}}{2}\right) \right\}
\end{equation}
La condizione per un punto stazionario è $\frac{dF}{dm_{i}}=0$, che è equivalente a:
\begin{equation}
-J \sum_{\langle k, i \rangle} m_{k} - h_{i} - \frac{1}{\beta} \frac{\partial S}{\partial m_{i}} = 0 \quad \text{o} \quad -\sum_{k} J_{ik} m_{k} - h_{i} - \frac{1}{\beta} \frac{\partial S}{\partial m_{i}} = 0
\end{equation}
dove $J_{ik}$ è l'accoppiamento tra spin $i$ e $k$, che è $J$ solo se $i$ e $k$ sono primi vicini, e $0$ altrimenti:
\begin{equation}
J_{ik} = \begin{cases} J & \text{se } |i-k|=1 \\ 0 & \text{altrimenti} \end{cases}
\end{equation}

\noindent Calcoliamo la derivata parziale dell'entropia rispetto a $m_i$ (considerando solo il contributo del sito $i$):
\begin{equation}
   - \frac{\partial S}{\partial m_{i}} = \frac{\partial}{\partial m_{i}} \left[ \frac{1+m_{i}}{2} \log\left(\frac{1+m_{i}}{2}\right) + \frac{1-m_{i}}{2} \log\left(\frac{1-m_{i}}{2}\right) \right] 
\end{equation}


Usando la regola di derivazione del prodotto :
\begin{align}
- \frac{\partial S}{\partial m_{i}}
&= \frac{1}{2} \log\left(\frac{1+m_{i}}{2}\right) + \frac{1+m_{i}}{2} \cdot \frac{1}{2 \cdot \frac{1+m_{i}}{2}}   -\frac{1}{2} \log\left(\frac{1-m_{i}}{2}\right) + \frac{1-m_{i}}{2} \cdot \frac{1}{(-2) \cdot \frac{1-m_{i}}{2}} \nonumber  \\
&= \frac{1}{2} \left[ \log\left(\frac{1+m_{i}}{2}\right) - \log\left(\frac{1-m_{i}}{2}\right) \right] \\ 
& =\frac{1}{2} \log\frac{1+m_{i}}{1-m_{i}} \\
&= \operatorname{arctanh}(m_{i}) \quad \quad \quad \quad \quad \quad \quad \quad\quad \quad \quad \quad \text{(vedi la sezione \ref{appendice1})}
\end{align}




Quindi, l'equazione per $m_i$ diventa:
\begin{equation}
-\sum_{k} J_{ik} m_{k} - h_{i} + \frac{1}{\beta} \operatorname{arctanh}(m_{i}) = 0
\end{equation}
Risolvendo per $m_i$, otteniamo le \textbf{equazioni del campo medio}:

\begin{tcolorbox}[colback=yellow!30,  
                  colframe=yellow!50!orange,  
                  boxrule=0.8pt, 
                  arc=3pt,  
                  top=4pt, bottom=4pt, left=6pt, right=6pt,
                  enhanced,
                  sharp corners=south]
\begin{equation}
m_{i} = \tanh\left[ \beta \left( \sum_{k} J_{ik} m_{k} + h_{i} \right) \right]
\end{equation}
\end{tcolorbox} 

In campo medio, lo spin $\sigma_i$ sente un campo magnetico efficace dato da:

\begin{equation}
    h_{\text{eff}, i} = h_{i} + \sum_{k} J_{ik} m_{k}
\end{equation}

Questo campo effettivo dipende dalla magnetizzazione media dei vicini, non dagli spin stessi.

\subsection{Dimostrazione dell'equivalenza con arctanh()} \label{appendice1}
L'equivalenza con la funzione arcotangente iperbolica $\operatorname{arctanh}(m_i)$ si basa sulla definizione di $\tanh(x)$ e sulla sua inversione.
Sia $y = \tanh(x)$:
$$ y = \tanh(x) = \frac{e^{x}-e^{-x}}{e^{x}+e^{-x}}  $$
Ponendo $z \equiv e^{x}$, l'espressione diventa:
$$ y = \frac{z-\frac{1}{z}}{z+\frac{1}{z}}  $$
Moltiplicando numeratore e denominatore per $z$  si ottiene:
$$ y = \frac{z^2-1}{z^2+1} \implies y(z^2+1) = z^2-1  $$
Risolvendo per $z^2$:
$$ z^2 y + y = z^2 - 1 \implies z^2(1-y) = 1+y \implies z^2 = \frac{1+y}{1-y}  $$
Ricordando che $z = e^x$, si ha $z^2 = e^{2x}$.
$$ e^{2x} = \frac{1+y}{1-y}  $$
Prendendo il logaritmo naturale di entrambi i lati:
$$ 2x = \log\left(\frac{1+y}{1-y}\right) \implies x = \frac{1}{2} \log\left(\frac{1+y}{1-y}\right)  $$
Poiché $x = \operatorname{arctanh}(y)$, ponendo $y = m_i$, si ha:
\begin{equation}
-\frac{\partial S}{\partial m_{i}} = \frac{1}{2} \log\frac{1+m_{i}}{1-m_{i}} = \operatorname{arctanh}(m_{i}) 
\end{equation}

\newpage
\section{Relazioni Esatte di Dobrushin, Lanford e Ruelle (DLR)}
Il metodo DLR (Dobrushin, Lanford, Ruelle) è un metodo noto in fisica matematica, utile per dimostrare che sistemi statistici esistono nel limite termodinamico (infinito) e per controllare le interazioni su regioni sempre più ampie. Consente di collegare i valori di aspettazione di osservabili in una piccola regione con i valori di aspettazione in una regione più grande.


Si parte dal valore di aspettazione dello spin $\sigma_i$:
\begin{equation}
\langle \sigma_{i} \rangle = \frac{1}{Z} \sum_{\{\sigma=\pm1\}} e^{-\beta H[\sigma]} \sigma_{i}
\end{equation}
Si divide l'Hamiltoniana $H[\sigma]$ in due parti:
\begin{equation}
H[\sigma] = H^{-}[\sigma^{-}] + \tilde{H}(\sigma_{i} + \text{vicini di } \sigma_{i} \equiv \tilde\sigma)
\end{equation}
\begin{itemize}
    \item $\mathbf{H^{-}[\sigma^{-}]}$: Contiene tutti i contributi che \textbf{non contengono} $\sigma_i$. 
    $\sigma^{-}$ indica l'insieme di tutti gli spin tranne $\sigma_i$
    \item $\mathbf{\tilde{H}}$: Contiene tutti i contributi che \textbf{contengono} $\sigma_i$.
\end{itemize}
L'Hamiltoniana è $H = -\sum_{i, k} J_{ik} \sigma_i \sigma_k - \sum_i h_i \sigma_i$.
Il termine $\tilde{H}$ sarà della forma:
\begin{equation}
     \tilde{H} = -\sigma_i \left( \sum_{k} J_{ik} \sigma_{k} + h_{i} \right) 
\end{equation}

dove la somma su $k$ è ristretta ai vicini di $i$.

\noindent Sostituendo nell'espressione del valore di aspettazione e separando la somma su $\sigma_i = \pm 1$:
\begin{equation}
    \langle \sigma_{i} \rangle = \frac{1}{Z} \sum_{\{\sigma^{-} = \pm 1\}} e^{-\beta H^{-}[\sigma^{-}]} \left[ \sum_{\sigma_{i}=\pm 1} \sigma_{i} e^{-\beta \tilde{H}(\tilde\sigma)} \right]
\end{equation}  
Concentrandosi sul termine $\sum_{\sigma_{i}=\pm 1} \sigma_{i} e^{-\beta \tilde{H}(\sigma_{i}, \ldots)}$:

\begin{equation}
  \sum_{\sigma_{i}=\pm 1} \sigma_{i} \exp\left( \beta \sigma_{i} \left( \sum_{k} J_{ik} \sigma_{k} + h_{i} \right) \right)  
  \label{eq: sum}
\end{equation}
 
Posto $X = \beta (\sum_{k} J_{ik} \sigma_{k} + h_{i})$, la somma in \ref{eq: sum} diventa:
\begin{equation}
    (+1) e^{X} + (-1) e^{-X} = 2 \sinh(X)
\end{equation}  
Similmente, la somma sui $\sigma_i$ nel denominatore $Z$ è:
\begin{equation}
    \sum_{\sigma_{i}=\pm 1} e^{-\beta \tilde{H}(\tilde \sigma)} = e^{X} + e^{-X} = 2 \cosh(X)
\end{equation}

Il rapporto $\frac{\sum_{\sigma_{i}=\pm 1} \sigma_{i} e^{-\beta \tilde{H}}}{\sum_{\sigma_{i}=\pm 1} e^{-\beta \tilde{H}}}$ è:
\begin{equation}
    \frac{2 \sinh(X)}{2 \cosh(X)} = \tanh(X)
\end{equation}

Quindi, l'espressione per $\langle \sigma_{i} \rangle$ diventa:
\begin{equation}
\langle \sigma_{i} \rangle = \left\langle \tanh\left( \beta \left( \sum_{k} J_{ik} \sigma_{k} + h_{i} \right) \right) \right\rangle
\end{equation}
Questa è una \textbf{relazione esatta}. Mette in relazione il valore di aspettazione dello spin $\sigma_i$ con una funzione del valore di aspettazione degli spin $\sigma_k$.

\section{Approssimazione di Campo Medio}
L'approssimazione di campo medio consiste nel sostituire gli spin $\sigma_k$ con i loro valori di aspettazione, ovvero le magnetizzazioni $m_k = \langle \sigma_k \rangle$:
$$ \sigma_{k} \to \langle \sigma_{k} \rangle = m_{k} $$
Questo è equivalente a fattorizzare la probabilità ($P[\sigma] \approx \prod_i P[\sigma_i]$). In questa approssimazione, la funzione $\tanh(\ldots)$ dipende ora solo da valori medi (numeri) e non da variabili stocastiche, quindi il valore di aspettazione $\langle \ldots \rangle$ non è più necessario.

Sostituendo $\sigma_k \to m_k$:
\begin{equation}
\langle \sigma_{i} \rangle = m_{i} = \tanh\left( \beta \left( \sum_{k} J_{ik} m_{k} + h_{i} \right) \right)
\end{equation}
Si ottengono di nuovo le equazioni del campo medio.


\subsection{Caso omogeneo e con campo magnetico costante }
Nel caso omogeneo e isotropo, con campo magnetico costante ($h_i = h$) e accoppiamento solo con i primi vicini ($J_{ik} = 1$ per $k$ vicino a $i$ e $0$ altrimenti), la magnetizzazione media è uniforme:

\begin{equation}
   \langle \sigma_{i} \rangle = m_i = m
\end{equation}

Sia $D$ la dimensionalità del reticolo. Per un reticolo cubico semplice, ogni sito ha $2D$ primi vicini.

L'energia interna media per sito è:
\begin{equation}
    \frac{\langle H \rangle}{N} = u = -D  m^2 - h m
\end{equation}

L'energia libera per unità di volume è:
\begin{tcolorbox}[colback=yellow!30,  
                  colframe=yellow!50!orange,  
                  boxrule=0.8pt, 
                  arc=3pt,  
                  top=4pt, bottom=4pt, left=6pt, right=6pt,
                  enhanced,
                  sharp corners=south]
\begin{equation}
f = -D m^{2} - h m + \frac{1}{\beta} \left( \frac{1+m}{2} \log\left(\frac{1+m}{2}\right) + \frac{1-m}{2} \log\left(\frac{1-m}{2}\right) \right)
\end{equation}
\end{tcolorbox} 


L'equazione del campo medio diventa:
\begin{equation}
m = \tanh\left( \beta (2 D  m + h) \right)
\end{equation}


\subsection{Caso con campo magnetico nullo}
Nel caso di campo magnetico esterno nullo ($h=0$):
\begin{equation}
m = \tanh(2 \beta D  m)
\end{equation}
Una soluzione è sempre \textbf{$m=0$}. La stabilità di $m=0$ dipende dalla temperatura.

\subsection{Risoluzione grafica}
Si intersecano le due curve $y=m$  e $y=\tanh(2 \beta D  m)$


\begin{itemize}
    \item \textbf{Alta Temperatura} ($T > T_c$, $\beta$ piccolo): 
    
    La pendenza $2 \beta D  < 1$. L'unica intersezione è in $m=0$.

    \item \textbf{Temperatura Critica} ($T = T_c$, $\beta_c$): 
    
    La pendenza $2 \beta_c D  = 1$. Questo definisce $\beta_c$ e quindi $T_c$:
    \begin{equation}
    \beta_{c} = \frac{1}{2 D } \implies T_{c} = 2 D 
    \end{equation}


    \item \textbf{Bassa Temperatura} ($T < T_c$, $\beta$ grande): 
    
    La pendenza $2 \beta D  > 1$. 
    Ci sono \textbf{tre intersezioni}: $m=0$ e due soluzioni non banali $m=m^+$ e $m=m^-$.
\end{itemize}

\begin{figure}[h!]
    \centering
    \begin{minipage}[b]{0.45\textwidth}
        \centering
        \includegraphics[width=\textwidth]{pics/06.png}
    \end{minipage}
    \hfill
    \begin{minipage}[b]{0.45\textwidth}
        \centering
        \includegraphics[width=\textwidth]{pics/07.png}
    \end{minipage}
    \hfill  
    \begin{minipage}[b]{0.45\textwidth}
        \centering
        \includegraphics[width=\textwidth]{pics/08.png}
    \end{minipage}
    \caption{Soluzioni per m al variare della temperatura.}
  
\end{figure}


\newpage
\subsection{Stabilità  dell'Energia Libera}
La stabilità delle soluzioni è determinata dal segno della derivata seconda dell'energia libera $f$. La soluzione stabile (minimo) richiede $f'' > 0$.
La derivata prima è:
$$ f' = \frac{df}{dm} = -2 D  m + \frac{1}{2\beta} \log\frac{1+m}{1-m} $$
La derivata seconda ($f''$) è:
\begin{equation}
f'' = -2 D  + \frac{1}{\beta} \cdot \frac{1}{1-m^{2}}
\end{equation}
Valutando in $m=0$ (che è sempre una soluzione):
\begin{equation}
f''(m=0) = -2 D  + \frac{1}{\beta}
\end{equation}
\begin{itemize}
    \item Per $T > T_c$ ($\beta < \beta_c$, $\frac{1}{\beta} > 2 D $): $f''(0) > 0$
    
    $m=0$ è un minimo (soluzione stabile)
    
    \item Per $T < T_c$ ($\beta > \beta_c$, $\frac{1}{\beta} < 2 D $): $f''(0) < 0$
    
    $m=0$ diventa un massimo (soluzione instabile)
    
    Le due nuove soluzioni $m^+$ e $m^-$ sono i minimi.
\end{itemize}

