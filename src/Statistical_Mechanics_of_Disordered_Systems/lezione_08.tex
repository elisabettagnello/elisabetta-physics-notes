\chapter{Lezione 8}
\label{chap:lezione_08}

\begin{flushright}
\textit{Data: 22/10/2025}
\end{flushright}

\section{Modelli Ferromagnetici Diluiti}

Fino ad ora abbiamo trattato modelli "puri", definiti da reticoli perfetti e interazioni costanti. Tuttavia, i sistemi reali presentano spesso disordine. Un esempio classico in materia condensata è dato dalle leghe magnetiche, dove avviene una sostituzione casuale di atomi magnetici con atomi non magnetici (impurità).
La domanda teorica fondamentale è se l'introduzione di questa perturbazione (il disordine) modifichi il comportamento critico del sistema (classe di universalità, esponenti critici) o se il sistema rimanga robusto.

\noindent Analizziamo due modelli principali di disordine geometrico:
\begin{itemize}
    \item \textbf{Site Dilution:} Si rimuovono randomicamente alcuni siti dal reticolo.
    \item \textbf{Link Dilution:} Si rimuovono randomicamente alcuni legami tra gli spin.
\end{itemize}

\subsubsection{Definizione delle Hamiltoniane}



Nel modello di \textbf{Site Dilution}, introduciamo una variabile casuale $\epsilon_i$ associata ad ogni sito $i$:

\begin{equation}
\mathcal{H}(\vec{S}) = -J \sum_{\langle i,j \rangle} \epsilon_i S_i \epsilon_j S_j - h \sum_i \epsilon_i S_i
\end{equation}

\noindent dove la variabile $\epsilon_i$ è definita come:

\begin{equation}
\epsilon_i = 
\begin{cases} 
1 & \text{con probabilità } p \\
0 & \text{con probabilità } 1-p 
\end{cases}
\end{equation}

\noindent Qui $p$ rappresenta la concentrazione di siti magnetici attivi (densità). Se $\epsilon_i=0$, il sito è vuoto e non interagisce.

\hfill 

\noindent Nel modello di \textbf{Link Dilution} (o Bond Dilution), i siti sono tutti presenti, ma il disordine è sui legami $J_{ij}$:

\begin{equation}
\mathcal{H}(\vec{S}) = - \sum_{\langle i,j \rangle} J_{ij} S_i S_j - h \sum_i S_i
\end{equation}

\noindent con la distribuzione:

\begin{equation}
J_{ij} = 
\begin{cases} 
1 & \text{con probabilità } p \\
0 & \text{con probabilità } 1-p 
\end{cases}
\end{equation}

\noindent Sebbene microscopicamente diversi, questi modelli condividono molte proprietà a livello macroscopico e possono spesso essere trattati con le stesse tecniche di teoria di campo (analogia con la percolazione di sito e di legame). Convenzionalmente, $p=1$ corrisponde al sistema puro (senza disordine).

\subsection{Diagramma di Fase}

Per studiare la possibile esistenza di una fase ferromagnetica, poniamo $h=0$ per permettere la rottura spontanea della simmetria $\mathbb{Z}_2$.
Costruiamo il diagramma di fase.

\begin{figure}[h!]
    \centering
    \includegraphics[width=0.6\textwidth]{pics/18.png}
    \caption{Diagramma di fase nel piano $(p, T)$  per il modello di Ising diluito. La linea rossa è la temperatura critica $T_c(p)$ che separa la fase P da quella F. La linea tratteggiata è la temperatura critica del modello omogeneizzato $T_c^{homo}(p)$, che funge da limite superiore (bound) per la temperatura critica reale.}
    \label{fig: fase ferromag}
\end{figure}



Possiamo fissare due punti chiave del diagramma:
\begin{enumerate}
    \item \textbf{Limite Puro ($p=1$):} Ritroviamo il modello di Ising puro. Esiste una temperatura critica ben definita $T_c(p=1)$ che separa la fase paramagnetica (P) da quella ferromagnetica (F).
    \item Limite a $T=0$ \textbf{(Percolazione):} A temperatura nulla, il sistema cerca di ordinarsi localmente. Tuttavia, l'ordine a lungo raggio è possibile solo se esiste un cluster infinito di spin connessi. Esiste una soglia critica $p_c$ (soglia di percolazione):
    \begin{itemize}
        \item Per $p < p_c$: Il sistema è frammentato in cluster finiti. Non c'è ordine a lungo raggio. Fase Non Percolativa o Paramagnetica.
        \item Per $p > p_c$: Esiste un cluster gigante che permette l'ordine ferromagnetico. Fase Percolativa o Ferromagnetica.
    \end{itemize}
\end{enumerate}

\section{Disuguaglianze di Griffiths}

Per analizzare analiticamente la regione della transizione, introduciamo le disuguaglianze di Griffiths. Queste si applicano a modelli ferromagnetici generalizzati con interazioni a più corpi, purché tutti i coefficienti di accoppiamento siano non negativi ($J, K, \dots \ge 0$).

L'Hamiltoniana generica è:

\begin{equation}
H = - \sum_i h_i S_i - \sum_{i<j} J_{ij} S_i S_j - \sum_{i<j<k} K_{ijk} S_i S_j S_k - \dots
\end{equation}

Le disuguaglianze fondamentali sono due:
\begin{enumerate}
    \item \textbf{Positività delle Correlazioni Totali:}
    \begin{equation}
    \langle S_i \rangle \ge 0 \quad \forall i \subset \{1, \dots, N\}
    \end{equation}
    \begin{equation}
    \langle S_i S_j\rangle \ge 0 \quad \forall i, j \subset \{1, \dots, N\}
    \end{equation}
     \begin{equation}
    \langle S_i S_j S_k\rangle \ge 0 \quad \forall i, j, k \subset \{1, \dots, N\}
    \end{equation}
     \begin{equation*}
    \dots 
    \end{equation*}
    Questa si dimostra facilmente espandendo la funzione di partizione (espansione di alta temperatura): essendo tutti i coefficienti positivi, ogni termine della somma è positivo.
    
    \item \textbf{Positività delle Correlazioni Connesse:}
    \begin{equation}
    \langle S_i S_j \rangle_c = \langle S_i S_j \rangle - \langle S_i \rangle \langle S_j \rangle \ge 0 \quad \forall i, j \subset \{1, \dots, N\}
    \end{equation}
\end{enumerate}

\subsection{Dimostrazione della Prima Disuguaglianza}

La dimostrazione si basa sull'espansione in serie ad alta temperatura, utilizzando l'identità per le variabili di Ising ($S = \pm 1$):
\begin{equation}
e^{\lambda S} = \cosh(\lambda) + S \sinh(\lambda) = \cosh(\lambda) [1 + S \tanh(\lambda)]
\end{equation}

\noindent Applichiamo questa identità a ogni termine dell'Hamiltoniana 

$H(\vec{S}) = - \sum h_i S_i - \sum J_{ij} S_i S_j - \dots$ all'interno della funzione di partizione:

\begin{equation}
e^{\beta h_i S_i} = \cosh(\beta h_i) [1 + S_i \tanh(\beta h_i)]
\end{equation}

\begin{equation}
e^{\beta J_{ij} S_i S_j} = \cosh(\beta J_{ij}) [1 + S_i S_j \tanh(\beta J_{ij})]
\end{equation}

\begin{equation}
e^{\beta K_{ijk} S_i S_j S_k} = \cosh(\beta K_{ijk}) [1 + S_i S_j S_k \tanh(\beta K_{ijk})]
\end{equation}

La funzione di partizione (o il numeratore di una media) può essere scritta come il prodotto di questi termini sommato su tutte le configurazioni:

\begin{equation}
Z = \sum_{\{S\}} \prod_i \cosh(\beta h_i) [1 + S_i \tanh(\beta h_i)] \prod_{\langle ij \rangle} \cosh(\beta J_{ij}) [1 + S_i S_j\tanh(\beta J_{ij})] \dots
\end{equation}


Osserviamo che:
\begin{itemize}
    \item I termini $\cosh(\dots)$ sono costanti positive e possono essere portati fuori.
    \item Poiché stiamo considerando un sistema ferromagnetico, le costanti di accoppiamento sono non negative ($J \ge 0, h \ge 0$), quindi $\tanh(\beta J) \ge 0$.
\end{itemize}

Sviluppando il prodotto, otteniamo un \textbf{polinomio nelle variabili di spin con coefficienti tutti positivi}.
Quando calcoliamo il valore di aspettazione $\langle S_i \rangle$, stiamo essenzialmente selezionando alcuni termini di questo polinomio e sommando su tutte le configurazioni $\{S\}$.
Poiché la somma su $\{S\}$ di un qualsiasi monomio di spin $\prod S_k$ è:
\begin{equation}
\sum_{S=\pm 1} S_{k_1} S_{k_2} \dots = 
\begin{cases} 
2^N & \text{se ogni spin appare una potenza pari (o zero)} \\
0 & \text{altrimenti}
\end{cases}
\end{equation}
Il risultato è una somma di termini positivi (i coefficienti del polinomio) moltiplicati per $2^N$ o $0$. Non c'è alcun termine negativo che possa rendere la media minore di zero.

\begin{equation}
\implies \langle S_i \rangle \ge 0
\end{equation}


\subsection{Dimostrazione della Seconda Disuguaglianza: Positività delle Correlazioni Connesse}

Vogliamo dimostrare che per un sistema ferromagnetico, le funzioni di correlazione connesse sono non negative:
\begin{equation}
\langle S_i S_j \rangle_c = \langle S_i S_j \rangle - \langle S_i \rangle \langle S_j \rangle \ge 0
\end{equation}

Per dimostrarlo, utilizziamo il trucco del raddoppio del sistema. Consideriamo due copie indipendenti del sistema, descritte dalle variabili di spin $S$ e $T$, entrambe soggette alla stessa Hamiltoniana $H$ con coefficienti positivi ($J, K, \dots \ge 0$).
Poiché le due copie sono indipendenti e identicamente distribuite, vale:
\begin{equation}
\langle S_i \rangle \langle S_j \rangle = \langle S_i \rangle \langle T_j \rangle = \langle S_i T_j \rangle_{S+T}
\end{equation}
dove la media $\langle \cdot \rangle_{S+T}$ è fatta sull'Hamiltoniana totale $H_{tot} = H(S) + H(T)$.
Possiamo quindi riscrivere la correlazione connessa come:
\begin{equation}
\langle S_i S_j \rangle - \langle S_i \rangle \langle S_j \rangle = \langle S_i S_j \rangle - \langle S_i T_j \rangle = \langle S_i (S_j - T_j) \rangle
\end{equation}

\noindent Introduciamo ora un cambio di variabili:
\begin{equation}
T_k = S_k \sigma_k \quad \text{con } \sigma_k \in \{+1, -1\}
\end{equation}
Questa trasformazione è invertibile poiché $S_k^2 = 1$.
L'Hamiltoniana totale del sistema combinato diventa:
\begin{equation}
H_{tot}(S, \sigma) = H(S) + H(S\sigma)
\end{equation}

\noindent Analizziamo i termini dell'Hamiltoniana trasformata. Per un'interazione generica a $p$ corpi con coefficiente $K_p \ge 0$:
\begin{equation}
- K_p S_{i_1} \dots S_{i_p} - K_p T_{i_1} \dots T_{i_p} = - K_p S_{i_1} \dots S_{i_p} (1 + \sigma_{i_1} \dots \sigma_{i_p})
\end{equation}
Il nuovo coefficiente efficace per l'interazione degli spin $S$ è $K^{eff}_p = K_p (1 + \sigma_{i_1} \dots \sigma_{i_p})$.
Poiché il prodotto delle $\sigma$ vale $\pm 1$, il termine in parentesi vale $0$ o $2$.
Dunque, per ogni configurazione fissata di $\sigma$, l'Hamiltoniana efficace per $S$ ha coefficienti non negativi ($K^{eff} \ge 0$).

\noindent Possiamo ora valutare il numeratore della differenza delle medie:
\begin{equation}
\sum_{S, T} e^{-\beta (H(S) + H(T))} (S_i S_j - S_i T_j) = \sum_{S, \sigma} e^{-\beta (H(S) + H(S\sigma))} S_i S_j (1 - \sigma_j)
\end{equation}
Riordinando la somma:
\begin{equation}
\sum_{\sigma} (1 - \sigma_j) \left[ \sum_{S} S_i S_j e^{-\beta H_{eff}(S; \sigma)} \right]
\end{equation}
Il termine tra parentesi quadre è proporzionale al valore di aspettazione $\langle S_i S_j \rangle$ calcolato con l'Hamiltoniana efficace $H_{eff}$. Poiché $H_{eff}$ ha tutti i coefficienti positivi, per la prima disuguaglianza di Griffiths (correlazioni totali), questo valore è non negativo.
Inoltre, il termine $(1 - \sigma_j)$ è sempre non negativo (vale $0$ o $2$).
Pertanto, l'intera somma è una somma di termini non negativi.

\begin{equation}
\implies \langle S_i S_j \rangle - \langle S_i \rangle \langle S_j \rangle \ge 0
\end{equation}

\section{Bound sulla Temperatura Critica}

Usiamo le disuguaglianze di Griffiths per confrontare il modello disordinato con un modello "omogeneizzato" (tipo campo medio o annealed), dove ogni accoppiamento $J_{ij}$ è sostituito dal suo valor medio:

\begin{equation}
J_{homo} = \overline{J_{ij}} = 1 \cdot p + 0 \cdot (1-p) = p
\end{equation}

\noindent La temperatura critica di questo modello omogeneo è semplicemente un riscalamento di quella del modello puro ($p=1$):

\begin{equation}
T_c^{homo}(p) = J_{homo} \cdot T_c(p=1) = p \cdot T_c(p=1)
\end{equation}

\noindent Questa relazione descrive la bisettrice nel piano $(T, p)$ della Figura \ref{fig: fase ferromag}.

\begin{tcolorbox}[colback=yellow!10, colframe=yellow!60!colorG, , coltitle=black, title=\textbf{Teorema}]
La temperatura critica del modello disordinato è sempre inferiore o uguale a quella del modello omogeneizzato:
\begin{equation}
T_c(p) \le T_c^{homo}(p) = p \cdot T_c(p=1)
\end{equation}
\end{tcolorbox}

\subsection{Dimostrazione:}

Vogliamo dimostrare che la magnetizzazione del modello disordinato è limitata superiormente da quella del modello omogeneo.
Innanzitutto, definiamo rigorosamente cosa intendiamo per magnetizzazione $m$ nel contesto di un sistema disordinato. Questa grandezza coinvolge due livelli di media:
\begin{enumerate}
    \item La \textbf{media termica} $\langle \cdot \rangle$, che è il valore di aspettazione calcolato sulla distribuzione di Boltzmann-Gibbs per una specifica configurazione congelata (quenched) del disordine $\{J_{ij}\}$.
    \item La \textbf{media sul disordine} $\overline{\dots}$ (o $\mathbb{E}_J[\dots]$), che è il valore di aspettazione calcolato sulla distribuzione di probabilità $P(J_{ij})$ dei coupling.
\end{enumerate}

La magnetizzazione media del sistema è quindi definita come la media sul disordine della media termica di uno spin generico (ad esempio $S_0$):
\begin{equation}
m = \overline{\langle S_0 \rangle}
\end{equation}

\noindent Tuttavia, c'è una sottigliezza fondamentale: in assenza di campo esterno ($h=0$), l'Hamiltoniana è invariante per l'inversione globale di tutti gli spin ($S_i \to -S_i$). Di conseguenza, la media termica $\langle S_0 \rangle$ calcolata sull'ensemble canonico completo sarebbe identicamente nulla ($m=0$) per simmetria, sia nella fase paramagnetica che in quella ferromagnetica.
Per definire un parametro d'ordine non nullo nella fase a simmetria rotta, dobbiamo restringere la media a uno dei due stati puri (ad esempio quello positivo). Operativamente, questo si ottiene applicando un piccolo campo esterno $h$ e prendendo il limite $h \to 0^+$:
\begin{equation}
\langle S_0 \rangle^+ = \lim_{h \to 0^+} \langle S_0 \rangle_h
\end{equation}
La disuguaglianza che vogliamo dimostrare confronta quindi la magnetizzazione media nello stato positivo del sistema disordinato con quella del sistema omogeneo :
\begin{equation}
m = \overline{\langle S_0 \rangle^+} \le m^{homo} = \langle S_0 \rangle^+_{homo}
\end{equation}
dove $\langle S_0 \rangle^+_{homo}$ è calcolata nel sistema in cui tutti i coupling sono sostituiti dal loro valor medio $\overline{J_{ij}} = p$.

\noindent Per dimostrare questa relazione, utilizziamo la \textbf{disuguaglianza di Jensen}. Dobbiamo mostrare che la funzione $\langle S_0 \rangle^+(\{J_{ij}\})$ è una \textbf{funzione concava} rispetto a ogni variabile di accoppiamento $J_{ij}$. Calcoliamo le derivate rispetto a un generico $J_{ij}$ (assorbendo $\beta$ nella derivazione per semplicità):

\begin{enumerate}
    \item \textbf{Derivata Prima:}
    Derivando la media termica, scende un fattore $S_i S_j$ dalla misura di Boltzmann. Si ottiene la funzione di correlazione connessa:
    \begin{equation}
    \frac{\partial \langle S_0 \rangle^+}{\partial J_{ij}} = \langle S_0 S_i S_j \rangle^+ - \langle S_0 \rangle^+ \langle S_i S_j \rangle^+ \ge 0
    \end{equation} 
    Per la seconda disuguaglianza di Griffiths, le correlazioni connesse sono non negative. Dunque la derivata prima è positiva, indicando che la magnetizzazione cresce all'aumentare dell'interazione.

    \item \textbf{Derivata Seconda:}
    Deriviamo nuovamente l'espressione precedente rispetto a $J_{ij}$. Dobbiamo derivare un prodotto di termini. Ricordando che per spin di Ising $S_i^2 = 1$.
    
    \begin{equation}
    \begin{split}
    \frac{\partial^2 \langle S_0 \rangle^+}{\partial J_{ij}^2} &= \frac{\partial}{\partial J_{ij}} \left( \langle S_0 S_i S_j \rangle^+ - \langle S_0 \rangle^+ \langle S_i S_j \rangle^+ \right) \\
    &= \left[ \langle S_0 (S_i S_j)^2 \rangle - \langle S_0 S_i S_j \rangle \langle S_i S_j \rangle \right] \\
    &\quad - \left[ \left( \langle S_0 S_i S_j \rangle - \langle S_0 \rangle \langle S_i S_j \rangle \right) \langle S_i S_j \rangle + \langle S_0 \rangle \left( \langle (S_i S_j)^2 \rangle - \langle S_i S_j \rangle^{2} \right) \right]
    \end{split}
    \end{equation}
    
    Sostituendo $(S_i S_j)^2 = S_i^2 S_j^2 = 1$:
    \begin{equation}
    \begin{split}
    \frac{\partial^2 \langle S_0 \rangle}{\partial J_{ij}^2} &= \left[ \langle S_0 \rangle - \langle S_0 S_i S_j \rangle \langle S_i S_j \rangle \right] \\
    &\quad - \left[ \langle S_0 S_i S_j \rangle \langle S_i S_j \rangle - \langle S_0 \rangle \langle S_i S_j \rangle^{2} + \langle S_0 \rangle - \langle S_0 \rangle \langle S_i S_j \rangle^{2} \right]
    \end{split}
    \end{equation}
    
    Semplificando i termini simili ($\langle S_0 \rangle$ si cancella) e raccogliendo $-2 \langle S_i S_j \rangle^+$, rimane:
   
    \begin{equation}
    \frac{\partial^2 \langle S_0 \rangle^+}{\partial J_{ij}^2} = -2 \langle S_i S_j \rangle^+ \left[ \langle S_0 S_i S_j \rangle^+ - \langle S_0 \rangle^+ \langle S_i S_j \rangle^+ \right]
    \end{equation}
    
    Analizziamo il segno di questa espressione:
    \begin{itemize}
        \item $\langle S_i S_j \rangle^+ \ge 0$ per la prima disuguaglianza di Griffiths.
        \item Il termine in parentesi quadre è la correlazione connessa $\langle S_0 S_i S_j \rangle^+_c$, che è $\ge 0$ per la seconda disuguaglianza di Griffiths.
        \item C'è un segno meno davanti a tutto.
    \end{itemize}
    
\noindent Concludiamo che la derivata seconda è non positiva ($\le 0$), quindi la funzione è \textbf{concava}.
\end{enumerate}

\begin{tcolorbox}[colback=colorD!10, colframe=colorD!60!colorE, , coltitle=white, title=\textbf{Esito della dimostrazione}]
\noindent Poiché $\langle S_0 \rangle^+$ è una funzione concava delle variabili $\{J_{ij}\}$, possiamo applicare la disuguaglianza di Jensen:
\begin{equation}
\mathbb{E}_J \left[ \langle S_0 \rangle^+(\{J_{ij}\}) \right] \le \langle S_0 \rangle^+(\{\mathbb{E}_J[J_{ij}]\})
\end{equation}
Ovvero:
\begin{equation}
\overline{\langle S_0 \rangle^+} \le \langle S_0 \rangle^+_{homo}
\end{equation}
Questo dimostra che, a parità di temperatura, l'ordine nel sistema disordinato è sempre inferiore a quello del sistema omogeneo medio. Di conseguenza, la temperatura critica del modello disordinato deve essere più bassa: $T_c(p) \le T_c^{homo}(p) = p T_c(p=1)$ 
\end{tcolorbox}


\begin{figure}[h!]
    \centering
    \includegraphics[width=0.55\textwidth]{pics/19.png}
    \caption{Confronto tra le magnetizzazioni in funzione della temperatura per $p_c~<~p~<~1$.}
    \label{fig: fase ferromag}
\end{figure}


\section{Criterio di Harris Generalizzato}

Il Criterio di Harris ci permette di capire se la presenza di un piccolo disordine (vicino al punto critico del sistema puro, $p \approx 1$) modifichi la classe di universalità della transizione, ovvero se gli esponenti critici cambino rispetto al caso puro.
Se il punto fisso del sistema puro è stabile rispetto al disordine, allora il sistema disordinato su grande scala si comporta come quello puro. Se invece è instabile, il flusso del gruppo di rinormalizzazione si sposta verso un nuovo punto fisso (ad esempio un "Random Fixed Point"), implicando nuovi esponenti critici.

Per applicare questo criterio al modello diluito, dobbiamo mappare il disordine geometrico in una fluttuazione di un parametro termodinamico locale.
Nel modello di Link Dilution, le quantità fisiche dipendono dal prodotto $\beta J_{ij}$.
Invece di considerare la temperatura inversa $\beta$ costante e gli accoppiamenti $J_{ij}$ come variabili casuali ($0$ o $1$), possiamo cambiare punto di vista: fissiamo l'interazione costante $J=1$ e consideriamo una \textbf{temperatura inversa locale} $\beta_{ij}$ che fluttua spazialmente:

\begin{equation}
\beta_{ij} = 
\begin{cases} 
\beta & \text{con probabilità } p \\
0 & \text{con probabilità } 1-p 
\end{cases}
\end{equation}

Questo approccio ci permette di interpretare il \textbf{disordine come una fluttuazione locale della temperatura critica} $T_c(\vec{x})$. Ogni regione dello spazio $\vec{x}$ avrà una sua temperatura critica locale che dipende dalla densità locale di legami attivi (o equivalentemente dal valore medio locale di $\beta_{ij}$).

Assumiamo che le fluttuazioni della temperatura critica locale siano correlate spazialmente secondo una funzione $g(\vec{r})$:

\begin{equation}
g(\vec{x}-\vec{y}) = \overline{T_c(\vec{x}) T_c(\vec{y})} - \overline{T_c}^2 \sim |\vec{x}-\vec{y}|^{-a} \quad \text{per } |\vec{x}-\vec{y}| \gg 1
\end{equation}

Dove l'esponente $a$ caratterizza il tipo di correlazione del disordine:
\begin{itemize}
    \item $a$ grande ($a \ge D$): disordine a corto raggio (uncorrelated).
    \item $a$ piccolo ($a < D$): disordine a lungo raggio (long-range correlated).
\end{itemize}

Introduciamo la temperatura ridotta locale $t(\vec{x})$, normalizzata rispetto alla temperatura critica media $\overline{T_c}$ (che nel regime perturbativo $p \approx 1$ è molto vicina a quella del sistema puro):
\begin{equation}
t(\vec{x}) = \frac{T - T_c(\vec{x})}{\overline{T_c}}
\end{equation}

Sappiamo che in un sistema puro, la lunghezza di correlazione $\xi$ diverge come una potenza della temperatura ridotta:
\begin{equation}
\xi \sim t^{-\nu} \implies t \sim \xi^{-1/\nu}
\end{equation}

Vogliamo capire se le fluttuazioni locali della temperatura ridotta sono rilevanti su una scala di lunghezza $\xi$. Per farlo, consideriamo un volume di correlazione $V \sim \xi^D$ e calcoliamo la media spaziale della temperatura ridotta in questo volume (\textbf{coarse graining}):
\begin{equation}
t_V = \frac{1}{V} \int_V d^D x \, t(\vec{x})
\end{equation}

Ci interessa confrontare il valor medio di questa quantità (che guida la transizione) con la sua varianza (fluttuazione) dovuta al disordine. Calcoliamo la fluttuazione quadratica media $\Delta^2$ su tutto il sistema:

\begin{equation}
\Delta^2 = \overline{t_V^2} - \overline{t_V}^2 = \frac{1}{V^2} \int_V d^D \vec{x} \int_V d^D \vec{y} \left( \overline{t(\vec{x})t(\vec{y})} - \overline{t(\vec{x})} \; \overline{t(\vec{y})} \right)
\end{equation}

Il termine tra parentesi è proporzionale alla funzione di correlazione connessa $g(\vec{x}-\vec{y})$. Sostituendo l'andamento asintotico $g(r) \sim r^{-a}$ e integrando:

\begin{equation}
\Delta^2 \propto \frac{1}{V} \int_V d^D r \, g(r) \sim \xi^{-D} \int_0^\xi dr \, r^{D-1} r^{-a} = \xi^{-D} \int_0^\xi dr \, r^{D-1-a}
\end{equation}

L'integrale si comporta diversamente a seconda del valore di $a$ rispetto alla dimensione spaziale $D$, determinando se le fluttuazioni mediano a zero o divergono:

\begin{enumerate}
    \item \textbf{Caso $a > D$ (Decadimento veloce/Corto raggio):}
    L'integrale $\int_0^\xi r^{D-1-a} dr$ converge per grandi $\xi$ (il contributo dominante viene da $r$ piccoli). L'integrale è asintoticamente una costante indipendente da $\xi$.
    \begin{equation}
    \Delta^2 \sim \xi^{-D}
    \end{equation}
    
    \item \textbf{Caso $a < D$ (Decadimento lento/Lungo raggio):}
    L'integrale diverge per grandi $\xi$ e il suo valore è dominato dal limite superiore di integrazione.
    \begin{equation}
    \int_0^\xi dr \, r^{D-1-a} \sim \xi^{D-a}
    \end{equation}
    Sostituendo nell'espressione per la varianza:
    \begin{equation}
    \Delta^2 \sim \xi^{-D} \cdot \xi^{D-a} = \xi^{-a}
    \end{equation}
    
    \item \textbf{Caso $a = D$ (Marginale):}
    L'integrale dà un contributo logaritmico:
    \begin{equation}
    \Delta^2 \sim \xi^{-D} \log \xi
    \end{equation}
\end{enumerate}

Riassumendo, la fluttuazione del disordine $\Delta$ scala come:

\begin{tcolorbox}[colback=yellow!10, colframe=yellow!60!colorG, , coltitle=black, title=\textbf{Fluttuazione del Disordine}]
\begin{equation}
\Delta \sim 
\begin{cases} 
\xi^{-D/2} & \text{se } a > D \\
\xi^{-D} \log \xi & \text{se } a = D \\
\xi^{-a/2} & \text{se } a < D
\end{cases}
\end{equation}
\end{tcolorbox}


\noindent Ora applichiamo il criterio di stabilità. Il disordine è una perturbazione \textbf{irrilevante} se, avvicinandosi al punto critico ($\xi \to \infty$), le fluttuazioni del disordine $\Delta$ diventano trascurabili rispetto alla distanza media dal punto critico $t$. Ovvero il "segnale" termico deve vincere sul "rumore" del disordine.
Ricordando che $t \sim \xi^{-1/\nu}$, la condizione di stabilità è:
\begin{equation}
\frac{\Delta}{t} \to 0 \quad \text{per } \xi \to \infty
\end{equation}
Ovvero, l'esponente con cui decade $\Delta$ deve essere maggiore di $1/\nu$.

\begin{itemize}
    \item Per $a > D$ (Criterio di Harris Standard):
    Richiediamo $D/2 > 1/\nu \implies D\nu > 2$.
    Usando la relazione di iperscaling $\alpha = 2 - D\nu$, questa condizione equivale a richiedere $\alpha < 0$. Se il calore specifico del sistema puro non diverge (o diverge molto debolmente), il disordine è irrilevante.
    
    \item Per $a < D$ (Criterio di Weinrib-Halperin):
    Richiediamo $a/2 > 1/\nu \implies a\nu > 2$.
    In questo caso, la correlazione a lungo raggio del disordine rende più difficile soddisfare la condizione di stabilità, poiché le fluttuazioni decadono più lentamente.
\end{itemize}

\noindent Se queste disuguaglianze sono violate, il disordine è rilevante: il punto fisso puro diventa instabile e il sistema fluisce verso un nuovo punto fisso disordinato con nuovi esponenti critici.

